\documentclass[9pt]{beamer}

\usepackage[T1]{fontenc}
\usepackage{amsmath}
\usepackage{graphicx}
\usepackage[export]{adjustbox}
\usepackage{algorithm}
\usepackage[citestyle=authortitle-ibid,backend=bibtex]{biblatex} 
\setbeamerfont{footnote}{size=\tiny}
%\usepackage{algorithm2e}
\usepackage{algpseudocode}
\graphicspath{{figures/}}


\usepackage[absolute,overlay]{textpos}

\usepackage{tikz}
\usetikzlibrary{automata,positioning}

%\usepackage{longtable}
%\usepackage{fancyhdr}
%\usepackage{color}
%\usepackage{array}
%\usepackage{mdwmath}
%\usepackage{mdwtab}
%\usepackage{amsmath,amssymb}
%\usepackage{cite}
%\usepackage{graphicx}
%\usepackage{listings}
\usepackage{subfig}
%\usepackage{booktabs}
%\usepackage{latexsym}
%\usepackage{color}
%\usepackage{rotating}
%\usepackage{multirow}
%\usepackage{paralist}
%\usepackage{bibentry}
%%\usepackage[algochapter]{algorithm2e}
%\PassOptionsToPackage{hyphens}{url}\usepackage[bookmarks=true,bookmarksopen=true]{hyperref}
%\usepackage{url}
%\usepackage{lscape}
%%\usepackage{algorithmic}
%\usepackage{algorithm}
%\usepackage{longtable}
%\usepackage[T1]{fontenc} 
%\usepackage[latin1]{inputenc}


%\usepackage{tikz}
\usepackage{multirow}
\usepackage{graphicx}
\usepackage{mathrsfs}
\usepackage{framed}
\usepackage{amssymb}
\usepackage[figuresright]{rotating}
\usepackage{comment}
\usepackage{subfig}
\usepackage{listings}
\usepackage{xcolor}
\usepackage{algpseudocode}

%\usepackage{flushend}
\usepackage{times}
\usepackage{url}
\usepackage{comment}
\usepackage{paralist}
\usepackage[multiple]{footmisc}
\usepackage{multicol}
%\usepackage{pbox}
\usepackage{textcomp}
\usepackage{stfloats}
\usepackage{multirow}
\usepackage{hhline}

\usepackage{longtable}

\usepackage{amsthm,thmtools}


\makeatletter
\def\th@scalabilityEvaluationStyle{%
    \normalfont % body font
    \setbeamertemplate{blocks}[rounded][shadow=false]
    \setbeamercolor{block title example}{bg=POLIMIEvaluationGreen,fg=white}
    \setbeamercolor{block body example}{bg=POLIMILightgray,fg=black}
    \def\inserttheoremblockenv{exampleblock}
  }
\theoremstyle{scalabilityEvaluationStyle}
\newtheorem*{scalabilityEvaluation}{}
\makeatother

\makeatletter
\def\th@complexityEvaluationStyle{%
    \normalfont % body font
    \setbeamertemplate{blocks}[rounded][shadow=false]
    \setbeamercolor{block title example}{bg=POLIMIComplexity,fg=black}
    \setbeamercolor{block body example}{bg=POLIMILightgray,fg=black}
    \def\inserttheoremblockenv{exampleblock}
  }
\theoremstyle{complexityEvaluationStyle}
\newtheorem*{complexityEvaluation}{}
\makeatother


\setbeamertemplate{blocks}[rounded][shadow=true]
\setbeamertemplate{theorems}[numbered]
 
 
 \newcommand{\backupbegin}{
   \newcounter{framenumberappendix}
   \setcounter{framenumberappendix}{\value{framenumber}}
}
\newcommand{\backupend}{
   \addtocounter{framenumberappendix}{-\value{framenumber}}
   \addtocounter{framenumber}{\value{framenumberappendix}} 
}
% usage: \tikzpic{<x percent>}{<y percent>}{<image size>}{<image name>}
\newcommand*{\tikzpic}[4]{%
\begin{tikzpicture}[remember picture,overlay]
\node at (current page.south west) [xshift=#1\paperwidth,yshift=#2\paperheight] {\includegraphics[width=#3\linewidth]{#4}};
\end{tikzpicture}%
}

\title[]%
{Esercitazione 2}
\author[C. Menghi]{\underline{Claudio Menghi,}\\
{\small \href{mailto:claudio.menghi@polimi.it}{\nolinkurl{claudio.menghi@polimi.it}}\\
}}
%\institute[POLIMI,Kennesaw State University]{Politecnico di Milano, Italy, Kennesaw State University, USA}

\usetheme{POLIMI}

\AtBeginSubsection[]{\frame[plain,noframenumbering]{\frametitle{Outline}%
                  \usebeamerfont{myTOC}\tableofcontents[currentsection,currentsubsection]}}

\student{Claudio Menghi}

\ExecuteBibliographyOptions{firstinits=true, isbn=false, url=false, doi=false, uniquename=init}

\AtEveryCitekey{%
\ifentrytype{article}{
    \clearfield{pages}%
    \clearfield{volume}%
}{}
}
\bibliography{bibliography}

\input{Sections/ExperimentValues.tex}

\newcounter{EsercizioCounter}
 \setcounter{EsercizioCounter}{1}


\newcommand{\Esercizio}[1]{
%\setlength{\fboxsep}{2pt}
\fbox{
   
  \parbox[t][]{\textwidth}{
   \vspace{2ex}
   \textbf{Esercizio \arabic{EsercizioCounter}}: #1
    \vspace{2ex}
    \refstepcounter{EsercizioCounter}
  }
}
}


%----------------------------------------------------------------------------------------
%	TITLE PAGE
%----------------------------------------------------------------------------------------

\title{
\vspace{2in}
\textmd{\textbf{\hmwkClass\\ \vspace{1cm} \hmwkTitle \vspace{1cm}}}\\
\normalsize\vspace{0.1in}\small{\hmwkDueDate}\\
\vspace{0.1in}\large{\textit{\hmwkClassInstructor\ \hmwkClassTime}}
\vspace{3in}
}
\begin{document}
%\color[RGB]{0,51,102}

\frame
{
\frametitle{Esercizi}

\Esercizio{Progettare e implementare una classe che rappresenti un LettoreMP3. Ogni LettoreMP3 ha un \emph{unico} numero di serie. Un numero di serie (o numero seriale) \`e un numero identificativo assegnato in maniera univoca per distinguere un esemplare di una serie. I numeri di serie sono associati in modo incrementale. Il LettoreMP3 ha una meroria e puo' essere ascoltato.}
}

\frame
{
\frametitle{Esercizi}

\Esercizio{Progettare e implementare una classe che rappresenti un \emph{insieme} di interi di dimensione massima fissata}
}

\frame
{
\frametitle{Esercizi}

\Esercizio{Modificare l'insieme di interi per consentire la \emph{creazione} di sets di dimensioni variabili ma minore della dimensione massima fissata al momento della creazione dell'insieme}
}

\frame
{
\frametitle{Esercizi}

\Esercizio{Progettare e implementare una classe che rappresenti un numero complesso}
}

\frame
{
\frametitle{Esercizi}

\Esercizio{Aggiungere un costruttore che permette di creare un numero complesso da module e fase}
}

\frame
{
\frametitle{Esercizi}

\Esercizio{Scrivere una classe Enum per modellizzare i pianeti del sistema solare}
}

\frame
{
\frametitle{Esercizi}

\Esercizio{Modificare il codice del secondo esercizio affinch\`e mantenga un insieme
di numeri complessi. Nella costruzione di tale insieme viene definita un certo valore di precisione
che è utilizzato per determinare l'uguaglianza tra due numeri}
}


\frame
{
\frametitle{Esercizi}

\Esercizio{Progettare e implementare una classe che rappresenti un CD. Ogni CD ha un \emph{unico} numero di serie. Un numero di serie (o numero seriale) è un numero identificativo assegnato in maniera univoca per distinguere un esemplare di una serie. Un cd ha un titolo, un autore e un prezzo.}
}

\frame
{
\frametitle{Esercizi}

\Esercizio{Implementare il gioco della morra cinese (carta, forbice, sasso).
L'applicazione deve chiedere la scelta al giocatore, generare una scelta e stampare a video il risultato del 
giocatore (vittoria, pareggio, sconfitta).}
}

%\section*{Bibliography}
%\begin{frame}[allowframebreaks]{Bibliography}
%\footnotesize{
%\bibliographystyle{abbrv}
%\bibliography{bibliography} }
%\end{frame}

\backupbegin

%\section{Tool demo}
\backupend




\end{document}
