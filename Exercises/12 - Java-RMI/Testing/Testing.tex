\documentclass{article}

\usepackage{fancyhdr} % Required for custom headers
\usepackage{lastpage} % Required to determine the last page for the footer
\usepackage{extramarks} % Required for headers and footers
\usepackage[usenames,dvipsnames]{color} % Required for custom colors
\usepackage{graphicx} % Required to insert images
\usepackage{listings} % Required for insertion of code
\usepackage{courier} % Required for the courier font
\usepackage{lipsum} % Used for inserting dummy 'Lorem ipsum' text into the template
\usepackage{hyperref}
\usepackage{multirow}
\usepackage{tabularx}
\usepackage{longtable}
\usepackage{listings}
\usepackage{subfigure}
\usepackage{afterpage}
\usepackage{amsmath,amssymb}            
\usepackage{rotating}  
\usepackage{fancyhdr}
\usepackage{graphicx}
\usepackage{amsthm}
\usepackage[scriptsize]{caption} 
\hyphenation{a-gen-tiz-za-zio-ne}
% Margins
\topmargin=-0.45in
\evensidemargin=0in
\oddsidemargin=0in
\textwidth=6.5in
\textheight=9.0in
\headsep=0.25in

\linespread{1.1} % Line spacing

\lstset{
  numbers=left,
  stepnumber=5,    
  firstnumber=1,
  numberfirstline=true
}

% Set up the header and footer
\pagestyle{fancy}
\lhead{\hmwkAuthorName} % Top left header
\chead{\hmwkClass\ (\hmwkClassInstructor\ \hmwkClassTime): \hmwkTitle} % Top center head
\rhead{\firstxmark} % Top right header
\lfoot{\lastxmark} % Bottom left footer
\cfoot{} % Bottom center footer
\rfoot{Page\ \thepage\ of\ \protect\pageref{LastPage}} % Bottom right footer
\renewcommand\headrulewidth{0.4pt} % Size of the header rule
\renewcommand\footrulewidth{0.4pt} % Size of the footer rule

\setlength\parindent{0pt} % Removes all indentation from paragraphs

\usepackage{listings}
\usepackage{color}

\definecolor{dkgreen}{rgb}{0,0.6,0}
\definecolor{gray}{rgb}{0.5,0.5,0.5}
\definecolor{mauve}{rgb}{0.58,0,0.82}

\lstset{frame=tb,
  language=Java,
  aboveskip=3mm,
  belowskip=3mm,
  showstringspaces=false,
  columns=flexible,
  basicstyle={\small\ttfamily},
  numbers=none,
  numberstyle=\tiny\color{gray},
  keywordstyle=\color{blue},
  commentstyle=\color{dkgreen},
  stringstyle=\color{mauve},
  breaklines=true,
  breakatwhitespace=true
  tabsize=3
}

%----------------------------------------------------------------------------------------
%	DOCUMENT STRUCTURE COMMANDS
%	Skip this unless you know what you're doing
%----------------------------------------------------------------------------------------

% Header and footer for when a page split occurs within a problem environment
\newcommand{\enterProblemHeader}[1]{
\nobreak\extramarks{#1}{#1 continued on next page\ldots}\nobreak
\nobreak\extramarks{#1 (continued)}{#1 continued on next page\ldots}\nobreak
}

% Header and footer for when a page split occurs between problem environments
\newcommand{\exitProblemHeader}[1]{
\nobreak\extramarks{#1 (continued)}{#1 continued on next page\ldots}\nobreak
\nobreak\extramarks{#1}{}\nobreak
}




%----------------------------------------------------------------------------------------
%	NAME AND CLASS SECTION
%----------------------------------------------------------------------------------------

\newcommand{\hmwkTitle}{Testing} % Assignment title
\newcommand{\hmwkDueDate}{Martedi,\ Aprile 15,\ 2014} % Due date
\newcommand{\hmwkClass}{Ingegneria del Software 1} % Course/class
\newcommand{\hmwkClassTime}{} % Class/lecture time
\newcommand{\hmwkClassInstructor}{Carlo Ghezzi} % Teacher/lecturer
\newcommand{\hmwkAuthorName}{} % Your name

%----------------------------------------------------------------------------------------
%	TITLE PAGE
%----------------------------------------------------------------------------------------

\title{
\vspace{2in}
\textmd{\textbf{\hmwkClass:\ \hmwkTitle}}\\
\normalsize\vspace{0.1in}\small{Due\ on\ \hmwkDueDate}\\
\vspace{0.1in}\large{\textit{\hmwkClassInstructor\ \hmwkClassTime}}
\vspace{3in}
}

\author{\textbf{\hmwkAuthorName}}
\date{} % Insert date here if you want it to appear below your name

%----------------------------------------------------------------------------------------

\begin{document}

\maketitle

%----------------------------------------------------------------------------------------
%	TABLE OF CONTENTS
%----------------------------------------------------------------------------------------

%\setcounter{tocdepth}{1} % Uncomment this line if you don't want subsections listed in the ToC

\newpage
\tableofcontents
\newpage



\section{Introduction}
\begin{itemize}
\item Software is buggy
\end{itemize}
Testing: concerns the execution a program on a sample of the input domain and checking the correctness of the program over these inputs.

Testing is 
\begin{itemize}
\item a Dynamic technique: the software must be executed for the testing activity to be performed
\item optimistic approximation: the program under test is performed in a subset of the input data. We are optimistic since we assume that the behavior of the program with the other input value is consistent with the input data we choose for testing it.
\end{itemize}
\subsection{Basic concepts}

\begin{itemize}
\item \emph{Failure}: is an incorrect behavior of the software
\end{itemize}
It is an observable behavior of the software which \emph{is not related to its code}.
\begin{itemize}
\item \emph{Fault or a bug}: incorrect code 
\end{itemize}
Is a necessary but not sufficient condition from the presence of a failure.
\begin{itemize}
\item \emph{Error}: is the cause of a fault
\end{itemize}

\subsection{Black-box testing or functional testing}
The test cases are determined in relation with the behavior the component has to exhibit.\\
Is used to identify \emph{failures} in the software.

The black box testing
\begin{itemize}
\item it is not necessary to have the code to determine the data to be used in the testing activity
\item these tests can be identified in the design phase
\item usually use the specification of the software to identify the input value to be used in the test. Usually by partitioning the input values of the component 
\end{itemize}

\subsection{White-box testing or structural testing}
The test cases are determined in relation with the source code to be tested.\\
Is used to identify\emph{bugs} in the software

\begin{itemize}
\item is complementary to the white box testing
\item the goal is to ``execute all the parts" of the code
\item the developers tries to find data that allow to execute all the program
\end{itemize}

There are three different criteria for the white box testing: \emph{instruction coverage}, \emph{branch coverage} and \emph{path coverage}.

\subsubsection{Instruction coverage (copertura delle istruzioni)}
The goal is to find a set $V=\left \{ v_1, v_2, \ldots v_n\right \}$ of test values that allows to cover every instruction at least one for some value $v_i \in V$. 
\subsubsection{Branch coverage (copertura delle diramazioni)}

\subsubsection{Path coverage (copertura dei cammini)}


\newpage

\section{Excercises: for the theory part}

\section{Excercises: for the lab}

\clearpage

% ---- Bibliography ----




\addcontentsline{toc}{chapter}{Bibliography}
\bibliographystyle{alpha}
\bibliography{bib}
\nocite{*}


\end{document}

