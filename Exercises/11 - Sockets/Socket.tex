\documentclass{article}

\usepackage{fancyhdr} % Required for custom headers
\usepackage{lastpage} % Required to determine the last page for the footer
\usepackage{extramarks} % Required for headers and footers
\usepackage[usenames,dvipsnames]{color} % Required for custom colors
\usepackage{graphicx} % Required to insert images
\usepackage{listings} % Required for insertion of code
\usepackage{courier} % Required for the courier font
\usepackage{lipsum} % Used for inserting dummy 'Lorem ipsum' text into the template
\usepackage{hyperref}
\usepackage{multirow}
\usepackage{tabularx}
\usepackage{longtable}
\usepackage{listings}
\usepackage{subfigure}
\usepackage{afterpage}
\usepackage{amsmath,amssymb}            
\usepackage{rotating}  
\usepackage{fancyhdr}
\usepackage{graphicx}
\usepackage{amsthm}
\usepackage[scriptsize]{caption} 
\hyphenation{a-gen-tiz-za-zio-ne}
% Margins
\topmargin=-0.45in
\evensidemargin=0in
\oddsidemargin=0in
\textwidth=6.5in
\textheight=9.0in
\headsep=0.25in

\linespread{1.1} % Line spacing

\lstset{
  numbers=left,
  stepnumber=5,    
  firstnumber=1,
  numberfirstline=true
}

% Set up the header and footer
\pagestyle{fancy}
\lhead{\hmwkAuthorName} % Top left header
\chead{\hmwkClass\ (\hmwkClassInstructor\ \hmwkClassTime): \hmwkTitle} % Top center head
\rhead{\firstxmark} % Top right header
\lfoot{\lastxmark} % Bottom left footer
\cfoot{} % Bottom center footer
\rfoot{Page\ \thepage\ of\ \protect\pageref{LastPage}} % Bottom right footer
\renewcommand\headrulewidth{0.4pt} % Size of the header rule
\renewcommand\footrulewidth{0.4pt} % Size of the footer rule

\setlength\parindent{0pt} % Removes all indentation from paragraphs

\usepackage{listings}
\usepackage{color}

\definecolor{dkgreen}{rgb}{0,0.6,0}
\definecolor{gray}{rgb}{0.5,0.5,0.5}
\definecolor{mauve}{rgb}{0.58,0,0.82}

\lstset{frame=tb,
  language=Java,
  aboveskip=3mm,
  belowskip=3mm,
  showstringspaces=false,
  columns=flexible,
  basicstyle={\small\ttfamily},
  numbers=none,
  numberstyle=\tiny\color{gray},
  keywordstyle=\color{blue},
  commentstyle=\color{dkgreen},
  stringstyle=\color{mauve},
  breaklines=true,
  breakatwhitespace=true
  tabsize=3
}

%----------------------------------------------------------------------------------------
%	DOCUMENT STRUCTURE COMMANDS
%	Skip this unless you know what you're doing
%----------------------------------------------------------------------------------------

% Header and footer for when a page split occurs within a problem environment
\newcommand{\enterProblemHeader}[1]{
\nobreak\extramarks{#1}{#1 continued on next page\ldots}\nobreak
\nobreak\extramarks{#1 (continued)}{#1 continued on next page\ldots}\nobreak
}

% Header and footer for when a page split occurs between problem environments
\newcommand{\exitProblemHeader}[1]{
\nobreak\extramarks{#1 (continued)}{#1 continued on next page\ldots}\nobreak
\nobreak\extramarks{#1}{}\nobreak
}




%----------------------------------------------------------------------------------------
%	NAME AND CLASS SECTION
%----------------------------------------------------------------------------------------



%----------------------------------------------------------------------------------------
%	TITLE PAGE
%----------------------------------------------------------------------------------------
\newcommand{\hmwkTitle}{Socket} % Assignment title
\newcommand{\hmwkDueDate}{Marted\`i,\ Maggio 17,\ 2016} % Due date
\newcommand{\hmwkClass}{Ingegneria del Software 1} % Course/class
\newcommand{\hmwkClassTime}{} % Class/lecture time
\newcommand{\hmwkClassInstructor}{Claudio Menghi, Alessandro Rizzi} % Teacher/lecturer
\newcommand{\hmwkAuthorName}{} % Your name

%----------------------------------------------------------------------------------------
%	TITLE PAGE
%----------------------------------------------------------------------------------------
\newcounter{EsercizioCounter}
 \setcounter{EsercizioCounter}{1}


\newcommand{\Esercizio}[1]{
%\setlength{\fboxsep}{2pt}
\fbox{
   
  \parbox[t][]{\textwidth}{
   \vspace{2ex}
   \textbf{Esercizio \arabic{EsercizioCounter}}: #1
    \vspace{2ex}
    \refstepcounter{EsercizioCounter}
  }
}
}




%----------------------------------------------------------------------------------------

\begin{document}

\maketitle

%----------------------------------------------------------------------------------------
%	TABLE OF CONTENTS
%----------------------------------------------------------------------------------------

%\setcounter{tocdepth}{1} % Uncomment this line if you don't want subsections listed in the ToC

\newpage
\tableofcontents
\newpage



\section{Introduzione}
La comunicazione di rete mediante socket pu\`o essere eseguita come specificato in seguito:
\begin{itemize}
\item il \emph{server}  attende che i client si connettano rimanendo in ascolto su una specifica porta (\emph{listening}) 
\item un \emph{client} conosce il nome dell'host (l'indirizzo ip della macchina) sul quale il server \`e eseguito e il numero di porta sulla quale il server \`e in ascolto. Il client prova a contattare il server su quella porta. 
\item il \emph{server} crea un nuovo canale di comunicazione.
\end{itemize} 

Un socket consente una comunicazione \emph{bidirezionale}, e per mezzo della porta il layer TCP riesce a identificare l'applicazione a cui vanno consegnati i paccketti ricevuti. Un socket ha due end-points: \texttt{PORT}, \texttt{IP} del server e \texttt{PORT}, \texttt{IP} del client.

Le classi di Java che consentono di gestire i socket sono contenuti all'interno del package \texttt{java.net}. La classe \texttt{ServerSocket} consente al server di creare un ascoltatore (listener) che accetta connessioni dal client.



\section{Esercizi}


\subsection{client-server (per un client singolo)} 
La comunicazione tra client e server passa per i seguenti punti:
\begin{itemize}
\item il processo server viene eseguito e attende (wait) una connessione dal client per mezzo della classe \texttt{Server Socket};
\item il processo client viene lanciato (start) e si connette al server sfruttando la classe \texttt{Socket}
\item il server crea un socket che viene utilizzato per comunicare con il client
\end{itemize}

\subsubsection{The server class}
\begin{itemize}
\item La classe server deve creare un istanza della classe \texttt{java.net.ServerSocket} attraverso l'istruzione
\begin{itemize}
\item \texttt{ServerSocket serverSocket=new ServerSocket(29999);}
\item il valore  29999 indica la porta sulla quale il server attende la connessione.
\end{itemize} 
\item Il server chiama il metodo  \texttt{accept()} di \texttt{serverSocket}. Questo metodo specifica che il server attende una determinata richiesta di connessione da parte del client\footnote{Attenzione assicurati che la porta non sia gi\`a in uso sul tuo server (per esempio che il tuo server non sia gi\`a in esecuzione.}.
\begin{itemize}
\item Socket socket=serverSocket.accept();
\end{itemize}
\item Quando un client si connette al server uno nuovo socket viene ritornato \texttt{Socket}.
\end{itemize}

\begin{lstlisting}[language=Java,escapechar=|]
package socket;

import java.io.IOException;
import java.io.PrintWriter;
import java.net.ServerSocket;
import java.net.Socket;
import java.util.Scanner;

/**
 * contains the Server class
 * This server is able to manage only one client
 * @author Claudio Menghi
 *
 */
public class Server {

	/**
	 * is the port that the Server uses to wait connections
	 */
	private final static int PORT = 29999;

	/**
	 * starts the server
	 * @throws IOException
	 */
	public void startServer() throws IOException {
		//creates a new Server socket on the specified port
		ServerSocket serverSocket = new ServerSocket(PORT);
		System.out.println("Server socket ready on port: " + PORT);

		// puts the serverSocket in a state where is waiting for client connections
		// the server stops on this instruction until a connection arrives
		Socket socket = serverSocket.accept();
		System.out.println("client connection received");

		// creates a new scanner that is able to read characters from the socket
		// for more detail drag your mouse over the Scanner class
		Scanner socketIn = new Scanner(
				socket.getInputStream());
		
		// creates a new PrintWriter that is able to write characters on the socket
		// for more details drag your mouse over the PrintWriter class
		PrintWriter socketOut=new PrintWriter(socket.getOutputStream());
		
		boolean end=false;
		
		// reads and write characters from the socket until the String quit is received
		while(!end){
			// line contains the String sent by the client
			String line=socketIn.nextLine();
			System.out.println("SERVER: received the string: "+line);
			if(line.equals("quit")){
				end=true;
			}
			else{
				// the Server sends to the client the following string
				socketOut.println("Well done client, the server has received the string: "+line);
				socketOut.flush();
			}
		}
		System.out.println("Closing the socket");
		// closes the scanner
		socketIn.close();
		// closes the printWriter
		socketOut.close();
		// closes the socket
		socket.close();
		// closes the ServerSocket
		serverSocket.close();
	}
	
	/**
	 * Runs the Server 
	 * @param args
	 */
	public static void main(String[] main){
		// creates the server
		Server echoServer=new Server();
		try {
			// starts the server
			echoServer.startServer();
		} catch (IOException e) {			
			e.printStackTrace();
		}
	}
}
\end{lstlisting}


\subsubsection{The client class}
\begin{itemize}
\item apre un socket verso lo specifico indirizzo IP e la data porta 
\begin{itemize}
\item \texttt{Socket socket = new Socket(“127.0.0.1”, 29999)}
\end{itemize}
\item ATTENZIONE: controllare che allo specifico indirizzo IP e alla data PORTA sia presente un server in attesa di connessioni (eseguite il server prima del client)
\item se la connessione \`e stata stabilita \`e possibile creare un \texttt{Scanner} per leggere dal socket e un \texttt{PrintWriter} per scrivere
\begin{itemize}
\item \texttt{Scanner in = new Scanner(socket.getInputStream());}
\item \texttt{PrintWriter out = new PrintWriter(socket.getOutputStream());}
\end{itemize}
\end{itemize}

\begin{lstlisting}[language=Java,escapechar=|]


import java.io.BufferedReader;
import java.io.IOException;
import java.io.InputStreamReader;
import java.io.PrintWriter;
import java.net.Socket;
import java.util.Scanner;

/**
 * contains the Client class
 * 
 * @author Claudio Menghi
 * 
 */
public class Client {

	/**
	 * is the port that the Server uses to wait connections
	 */
	private final static int PORT = 29999;

	/**
	 * starts the Client
	 * 
	 * @throws IOException
	 */
	public void startClient() throws IOException {

		/*
		 * creates a new socket that refers to the machine with the specified IP
		 * and PORT 127.0.0.1 refers to your local machine, change this address
		 * if you want to use your program in the network (in case of problems
		 * check your firewall preferences)
		 */
		Socket socket = new Socket("127.0.0.1", PORT);
		System.out.println("Connection Established");

		// creates a new stream that is able to read characters from the socket
		// for more detail drag your mouse over the Scanner class
		Scanner socketIn = new Scanner(socket.getInputStream());
		// creates a new stream that is able to write characters on the socket
		// for more details drag your mouse over the PrintWriter class
		PrintWriter socketOut = new PrintWriter(socket.getOutputStream());

		// creates a new scanner to read input from standard input
		Scanner stdin = new Scanner(new InputStreamReader(System.in));

		while (true) {
			// read the string from the standard input
			String inputLine = stdin.nextLine();
			// sends the read string to the server
			socketOut.println(inputLine);
			socketOut.flush();
			// checks if there is a message sent from the server
			if (socketIn.hasNextLine()) {
				String socketLine = socketIn.nextLine();
				// print the message to the standard output
				System.out.println(socketLine);
			} else {
				break;
			}
		}

		// closes the scanner
		socketIn.close();
		// closes the printWriter
		socketOut.close();
		// closes the standard input scanner
		stdin.close();
		// closes the socket
		socket.close();

	}
	/**
	 * Runs the Client 
	 * @param args
	 */
	public static void main(String[] args) {
		// creates a new Client
		Client client = new Client();
		try {
			// runs the Client
			client.startClient();
		} catch (IOException e) {
			e.printStackTrace();
		}
	}
}
\end{lstlisting}


\subsection{client-server (multi client)} 
La comunicazione client server multi-client si basa sui seguenti passi:
\begin{itemize}
\item il server viene eseguito e attende una connessione sul suo \texttt{ServerSocket}
\item il client viene eseguito e si connette al server mediante la classe \texttt{Socket}
\item il server \emph{crea il socket} da utilizzare per comunicare con il  client
\item il server \emph{crea un nuovo thread} che \`e responsabile di gestire il socket e la comunicazione con il client specifico
\end{itemize}
ATTENZIONE!!! all'interno del server ogni socket viene gestito da un thread distinto. Nel seguente esempio il behavior del thread \`e contenuto nella classe \texttt{ServerClientHandler}.

\subsubsection{The multi-client server class}
\begin{lstlisting}[language=Java,escapechar=|]
import java.io.IOException;
import java.net.ServerSocket;
import java.net.Socket;
import java.util.concurrent.ExecutorService;
import java.util.concurrent.Executors;

/**
 * contains the Server class This server is able to manage multiple clients
 * 
 * @author Claudio Menghi
 * 
 */
public class Server {

	/**
	 * is the port that the Server uses to wait connections
	 */
	private final static int PORT = 29999;

	/**
	 * starts the server
	 * 
	 * @throws IOException
	 */
	public void startServer() throws IOException {
		/*
		 * ExecutorService represents an asynchronous execution mechanism which
		 * is capable of executing tasks in the background.
		 * Executors.newCachedThreadPool() Creates a thread pool that creates
		 * new threads as needed, but will reuse previously constructed threads
		 * when they are available.
		 */
		ExecutorService executor = Executors.newCachedThreadPool();

		// creates a new Server socket on the specified port
		ServerSocket serverSocket = new ServerSocket(PORT);
		System.out.println("Server socket ready on port: " + PORT);

		System.out.println("Server ready");
		while (true) {
			try {
				// puts the serverSocket in a state where is waiting for client
				// connections
				// the server stops on this instruction until a connection
				// arrives
				Socket socket = serverSocket.accept();
				// Submits a Runnable task for execution
				executor.submit(new ClientHandler(socket));
			} catch (IOException e) {
				break;
			}
		}
		// shutdown the executor
		executor.shutdown();

		// closes the ServerSocket
		serverSocket.close();
	}

	/**
	 * Runs the Server
	 * 
	 * @param args
	 */
	public static void main(String[] args) {
		Server server = new Server();

		// starts the server
		try {
			server.startServer();
		} catch (IOException e) {
			e.printStackTrace();
		}
	}
}
\end{lstlisting}

\subsubsection{The ClientHandler class}
\begin{lstlisting}[language=Java,escapechar=|]

import java.io.IOException;
import java.io.PrintWriter;
import java.net.Socket;
import java.util.Scanner;


/**
 * contains the Handler of a Client
 * @author Claudio Menghi
 *
 */
public class ClientHandler implements Runnable{
	
	/**
	 * contains the Socket which i managed by the Handler
	 */
	private Socket socket;
	
	/**
	 * creates a new {@link ServerClientHandler} with the specified socket
	 * @param socket is the socket that is managed by the handler
	 */
	public ClientHandler(Socket socket){
		this.socket = socket;
	}
	
	/**
	 * executes the client handler
	 */
	public void run() {
		try {
			// creates a new stream that is able to read characters from the socket
			// for more detail drag your mouse over the BufferedReader class
			Scanner socketIn = new Scanner(socket.getInputStream());
			
			// creates a new stream that is able to write characters on the socket
			// for more details drag your mouse over the PrintWriter class
			PrintWriter socketOut = new PrintWriter(socket.getOutputStream());
			
			while (true){				
				// reads a new Line from the Socket
				String line = socketIn.nextLine();
				
				// writes the line in the screen
				System.out.println(line);
				
				// if the line is equal to quit
				if (line.equals("quit")){
					// exits from the while
					break;
				}
				else {
					// the Server sends to the client the following string
					socketOut.println("Well done client, the server has reveived the String: " + line);
					socketOut.flush();
				}
			}
			// closes the scanner
			socketIn.close();
			// closes the printWriter
			socketOut.close();
			// closes the socket
			socket.close();
			
		} catch (IOException e) {
			System.err.println(e.getMessage());
		}
	}
}
\end{lstlisting}

\subsection{client-server (gioco condiviso) Client}
\label{clientservershared}
Il lettore pu\`o utilizzare il client precedentemente descritto


\subsubsection{Client}
\begin{lstlisting}[language=Java,escapechar=|]
package game.client;

import java.io.IOException;
import java.io.PrintWriter;
import java.net.Socket;
import java.util.Scanner;
import java.util.concurrent.ExecutorService;
import java.util.concurrent.Executors;

/**
 * contains the Client implementation
 * 
 * @author Claudio Menghi
 * 
 */
public class Client {

	/**
	 * is the port that the Server uses to wait connections
	 */
	private final static int PORT = 29999;
	private final static String IP="127.0.0.1";

	/**
	 * starts the Client
	 * 
	 * @throws IOException
	 */
	public void startClient() throws IOException {

		/*
		 * creates a new socket that refers to the machine with the specified IP
		 * and PORT 127.0.0.1 refers to your local machine, change this address
		 * if you want to use your program in the network (in case of problems
		 * check your firewall preferences)
		 */
		Socket socket = new Socket(IP, PORT);
		System.out.println("Connection Established");

		// creates the thread pool
		ExecutorService executor = Executors.newFixedThreadPool(2);
		
		// runs the client handler in charge of getting messages from the server
		executor.submit(new ClientInHandler(new Scanner(socket.getInputStream())));
		
		// runs the client handler in charge of reading the mosse from the command line and sending it to the server
		executor.submit(new ClientOutHandler(new PrintWriter(socket.getOutputStream())));
		
	}
	/**
	 * Runs the Client 
	 * @param args
	 */
	public static void main(String[] args) {
		// creates a new Client
		Client client = new Client();
		try {
			// runs the Client
			client.startClient();
		} catch (IOException e) {
			e.printStackTrace();
		}
	}
}
\end{lstlisting}

\subsubsection{ClientInHandler}
\begin{lstlisting}[language=Java,escapechar=|]
/**
 * manages the incoming connection of the client, that is the messages sent from the server to the client
 * @author Claudio Menghi
 *
 */
public class ClientInHandler implements Runnable {

	/**
	 * contains the Scanner which is used to read messages from the server
	 */
	private Scanner socketIn;
	
	/**
	 * creates a new Handler for the incoming connections
	 * @param socketIn is the scanner which is used to read messages that are sent from the server
	 */
	public ClientInHandler(Scanner socketIn) {
		this.socketIn=socketIn;
	}
	
	/**
	 * executes the client handler
	 */
	public void run() {
			
			while (true) {
				// reads a new Line from the Scanner
				String line = socketIn.nextLine();
				// print the message sent by the server
				System.out.println(line);
			}
	}
}
\end{lstlisting}

\subsubsection{ClientOutHandler}
\begin{lstlisting}[language=Java,escapechar=|]
/**
 * contains the Handler which is responsible of reading the mosse from the
 * command line and sending them to the server
 * 
 * @author Claudio Menghi
 * 
 */
public class ClientOutHandler implements Runnable {

	/**
	 * contains the writer which is used to send messages to the server
	 */
	private PrintWriter socketOut;

	/**
	 * creates a new Handler responsible to read the mosse from command line and
	 * send them to the server
	 * 
	 * @param socketOut
	 */
	public ClientOutHandler(PrintWriter socketOut) {
		this.socketOut = socketOut;
	}

	/**
	 * executes the client handler
	 */
	public void run() {
		// creates a new scanner to read input from standard input

		Scanner stdin = new Scanner(System.in);

		while (true) {
			// read the string from the standard input
			String inputLine = stdin.nextLine();
			// sends the read string to the server
			socketOut.println(inputLine);
			socketOut.flush();
		}
	}
}
\end{lstlisting}

\subsection{client-server (gioco condiviso) Server}

\subsubsection{Model}
\begin{lstlisting}[language=Java,escapechar=|]
/**
 * contains the possible states of the game
 * @author Claudio Menghi
 *
 */
public enum GameState {	
	
	WAITINGTWOPLAYERS  {
		@Override
		public GameState nextState() {
			
			return WAITINGONEPLAYER;
		}
	},
	WAITINGONEPLAYER  {
		@Override
		public GameState nextState() {
			
			return RUNNING;
		}
	},
	RUNNING  {
		@Override
		public GameState nextState() {
			
			return ONE_MOSSA_RECEIVED;
		}
	},
	ONE_MOSSA_RECEIVED{
		@Override
		public GameState nextState() {
			
			return ENDED;
		}
	},
	ENDED  {
		@Override
		public GameState nextState() {
			
			return ENDED;
		}
	};
	
	public abstract GameState nextState();
}
\end{lstlisting}
\begin{lstlisting}[language=Java,escapechar=|]
/**
 * contains the different actions the user can perform
 * @author Claudio Menghi
 *
 */
public enum Mossa{	
	
	FORBICE  {
		@Override
		public boolean batte(Mossa other) {
			
			return other == CARTA;
		}
	},	
	CARTA {
		@Override
		public boolean batte(Mossa other){
			return other == PIETRA;
		}
	},
	PIETRA { 
		@Override
		public boolean batte(Mossa other){
			return other == FORBICE;
		} 
	};
	
	public abstract boolean batte(Mossa other);
}
\end{lstlisting}

\begin{lstlisting}[language=Java,escapechar=|]
/**
 * contains the model of the game
 * 
 * @author Claudio Menghi
 * 
 */
public class Partita extends Observable{

	// contains the number of player of the game
	private static final int nPlayers = 2;
	
	private Player winner;
	
	/**
	 * mappa che contiene per ogni giocatore la mossa scelta
	 */
	private Map<Player, Mossa> mossa;

	// contains the current state of the game
	private GameState gameState;

	public Partita() {
		this.gameState = GameState.WAITINGTWOPLAYERS;
		this.mossa = new HashMap<Player, Mossa>();
		this.mossa.put(new Player("player1"), null);
		this.mossa.put(new Player("player2"), null);
	}

	public void addMossa(Player player, Mossa mossa){
		this.mossa.put(player, mossa);
	}
	
	/**
	 * returns the number of players
	 * 
	 * @return the number of players
	 */
	public static int getNplayers() {
		return nPlayers;
	}

	/**
	 * returns the state of the game
	 * 
	 * @return the current state of the game
	 */
	public GameState getGameState() {
		return gameState;
	}

	/**
	 * sets the state of the game
	 * 
	 * @param gameState
	 *            sets the state of the game to the gameState
	 */
	public void setGameState(GameState gameState) {
		this.gameState = gameState;
		this.setChanged();
		this.notifyObservers("Game state: "+gameState);
	}
	
	/**
	 * ritorna l'insieme dei giocatori
	 * @return the set of the players
	 */
	public Set<Player> getPlayers(){
		return this.mossa.keySet();
	}
	
	public boolean hasPerformedAnAction(Player player){
		return this.mossa.get(player)!=null;
	}
	
	
	public Map<Player, Mossa> getPlayersActions(){
		return Collections.unmodifiableMap(this.mossa);
	}



	public Player getWinner() {
		return winner;
	}

	public void setWinner(Player winner) {
		
		this.winner = winner;
		this.setChanged();
		this.notifyObservers("The winner is "+winner);
	}

}
\end{lstlisting}

\begin{lstlisting}[language=Java,escapechar=|]
public class Player{

	
	private String name;
	
	public Player(String name){
		this.name=name;
	}
	
	@Override
	public String toString() {
		return "Player [name=" + name + "]";
	}

	public String getName(){
		return this.name;
	}

	@Override
	public int hashCode() {
		final int prime = 31;
		int result = 1;
		result = prime * result + ((name == null) ? 0 : name.hashCode());
		return result;
	}

	@Override
	public boolean equals(Object obj) {
		if (this == obj)
			return true;
		if (obj == null)
			return false;
		if (getClass() != obj.getClass())
			return false;
		Player other = (Player) obj;
		if (name == null) {
			if (other.name != null)
				return false;
		} else if (!name.equals(other.name))
			return false;
		return true;
	}
}
\end{lstlisting}

\subsubsection{Actions}
\begin{lstlisting}[language=Java,escapechar=|]
import game.server.model.Partita;
import game.server.model.Player;

public abstract class Action{

	private Player player;

	public Action(Player player) {
		this.setPlayer(player);
	}

	public Player getPlayer() {
		return player;
	}

	public void setPlayer(Player player) {
		this.player = player;
	}
	public abstract void esegui(Partita model);
}
\end{lstlisting}

\begin{lstlisting}[language=Java,escapechar=|]
import java.util.Iterator;
import java.util.Set;

import game.server.model.GameState;
import game.server.model.Mossa;
import game.server.model.Partita;
import game.server.model.Player;

public class EseguiMossa extends Action {

	private final Mossa mossa;
	
	public EseguiMossa(Mossa mossa, Player player) {
		super( player);
		this.mossa=mossa;
		
	}

	@Override
	public void esegui(Partita model) {
		System.out.println("Server: Performing the action");
		synchronized (model) {
			System.out.println(model.getGameState());
			if(
					(model.getGameState().equals(GameState.RUNNING)|| model.getGameState().equals(GameState.ONE_MOSSA_RECEIVED)) && 
					!model.hasPerformedAnAction(this.getPlayer())){
				

				model.addMossa(this.getPlayer(), mossa);
				GameState gameState=model.getGameState();
				GameState nextState=gameState.nextState();
				
				System.out.println("updating the game state");
				model.setGameState(nextState);
				if(nextState==GameState.ENDED){
					Set<Player> players=model.getPlayers();
					Iterator<Player> it=players.iterator();
					Player player1=it.next();
					Player player2=it.next();
					if(model.getPlayersActions().get(player1).batte(model.getPlayersActions().get(player2))){
						model.setWinner(player1);
					}
					else{
						model.setWinner(player2);
					}
				}	
			}
		}
		
		System.out.println("Server: Action performed");
	}
}
\end{lstlisting}



\subsubsection{View}
\begin{lstlisting}[language=Java,escapechar=|]
import game.server.actions.Action;
import game.server.actions.EseguiMossa;
import game.server.model.Mossa;
import game.server.model.Player;

import java.io.IOException;
import java.io.PrintWriter;
import java.net.Socket;
import java.util.Observable;
import java.util.Scanner;
import java.util.StringTokenizer;

/**
 * contains the Handler of a Client
 * 
 * @author Claudio Menghi
 * 
 */
public class ServerSocketView extends View implements Runnable{

	/**
	 * contains the Socket which i managed by the Handler
	 */
	private Socket socket;
	private Scanner socketIn;
	private PrintWriter socketOut;
	
	/**
	 * creates a new {@link ServerSocketView} with the specified socket
	 * 
	 * @param socket
	 *            is the socket that is managed by the handler
	 * @throws IOException
	 */
	public ServerSocketView(Socket socket) throws IOException {
		this.socket = socket;
		// creates a new stream that is able to read characters from the socket
		// for more detail drag your mouse over the Scanner class
		socketIn = new Scanner(socket.getInputStream());

		// creates a new stream that is able to write characters on the socket
		// for more details drag your mouse over the PrintWriter class
		socketOut = new PrintWriter(socket.getOutputStream());
	}
	

	/**
	 * executes the client handler
	 */
	public void run() {
		try {

			while (true) {
				// reads a new Line from the Socket
				String line = socketIn.nextLine();

				System.out.println("SERVER: getting the command "+line);
				StringTokenizer tokenizer=new StringTokenizer((String) line);
				
				Player player=new Player(tokenizer.nextToken());
				System.out.println("SERVER: "+player);
				
				Mossa mossa=Mossa.valueOf(tokenizer.nextToken().toUpperCase());
				System.out.println("SERVER: "+mossa);
				
				Action action=new EseguiMossa(mossa, player);
				
				this.setChanged();
				this.notifyObservers(action);
				if (line.equals("quit")) {
					// exits from the while
					break;
				}
			}
			// closes the scanner
			socketIn.close();
			// closes the printWriter
			socketOut.close();
			// closes the socket
			socket.close();

		} catch (IOException e) {
			System.err.println(e.getMessage());
		}
	}

	@Override
	public void update(Observable o, Object arg) {
		System.out.println("SERVER-VIEW-SOCKET: sending the client the message: "+
				arg);
		socketOut
		.println("SERVER: "
				+ arg);
		socketOut.flush();
	}
}
\end{lstlisting}
\begin{lstlisting}[language=Java,escapechar=|]
public abstract class View extends Observable implements Observer {


}
\end{lstlisting}
\subsubsection{Server}
\begin{lstlisting}[language=Java,escapechar=|]
import game.server.model.GameState;
import game.server.model.Partita;
import game.server.view.ServerSocketView;
import game.server.view.View;

import java.io.IOException;
import java.net.ServerSocket;
import java.net.Socket;
import java.rmi.AlreadyBoundException;
import java.rmi.RemoteException;
import java.util.concurrent.ExecutorService;
import java.util.concurrent.Executors;

/**
 * contains the implementation of the Server class. This server is able to manage multiple clients
 * 
 * @author Claudio Menghi
 * 
 */
public class Server {

	/**
	 * is the port that the Server uses to wait connections
	 */
	private final static int PORT = 29999;
	
	private Partita partita;
	private Controller controller;
	
	public Server() throws RemoteException{
		partita=new Partita();
		controller=new Controller(partita);
	}
	
	public void start() throws AlreadyBoundException, IOException{
		this.startSocket();
	}
	
	
	/**
	 * starts the server
	 * 
	 * @throws IOException
	 */
	private void startSocket() throws IOException {
		/*
		 * ExecutorService represents an asynchronous execution mechanism which
		 * is capable of executing tasks in the background.
		 * Executors.newCachedThreadPool() Creates a thread pool that creates
		 * new threads as needed, but will reuse previously constructed threads
		 * when they are available.
		 */
		ExecutorService executor = Executors.newCachedThreadPool();

		// creates a new Server socket on the specified port
		ServerSocket serverSocket = new ServerSocket(PORT);
		System.out.println("Server socket ready on port: " + PORT);
		System.out.println("Server ready");
		
		while (true) {

			try {
				// puts the serverSocket in a state where is waiting for client
				// connections
				// the server stops on this instruction until a connection
				// arrives
				Socket socket = serverSocket.accept();
				ServerSocketView view=new ServerSocketView(socket);
				this.addClient(view);
				executor.submit(view);
				

			} catch (IOException e) {
				break;
			}
		}
		// shutdown the executor
		executor.shutdown();

		// closes the ServerSocket
		serverSocket.close();
	}
	

	public void addClient(View view){
		view.addObserver(controller);
		partita.addObserver(view);
		// Submits a Runnable task for execution
		partita.setGameState(partita.getGameState().nextState());
		if(partita.getGameState().equals(GameState.RUNNING)){
			System.out.println("creo una nuova partita");
			partita=new Partita();
			System.out.println(partita.getGameState());
			controller=new Controller(partita);
		}	
	}
	/**
	 * Runs the Server
	 * 
	 * @param args
	 * @throws AlreadyBoundException 
	 * @throws RemoteException 
	 */
	public static void main(String[] args) throws AlreadyBoundException, RemoteException {
		Server server = new Server();
		// starts the server
		try {
			server.start();
		} catch (IOException e) {
			e.printStackTrace();
		}
	}
}
\end{lstlisting}

\begin{lstlisting}[language=Java,escapechar=|]

import game.server.actions.Action;
import game.server.model.Partita;

import java.util.Observable;
import java.util.Observer;

/**
 * contains the controller of the game
 * 
 * @author Claudio Menghi
 * 
 */
public class Controller implements Observer{

	
	private Partita partita;

	/**
	 * creates a new Controller
	 */
	public Controller(Partita partita) {
		this.partita=partita;
	}

	
	@Override
	public void update(Observable o, Object arg) {
		Action action=(Action) arg;
		System.out.println("Server: the thread that is performing the action have been lanched ");
		
		action.esegui(partita);
	}
}
\end{lstlisting}

\clearpage

% ---- Bibliography ----




\addcontentsline{toc}{chapter}{Bibliography}
\bibliographystyle{alpha}
\bibliography{bib}
\nocite{*}


\end{document}

