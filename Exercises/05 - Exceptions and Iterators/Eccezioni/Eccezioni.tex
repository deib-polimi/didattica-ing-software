%%%%%%%%%%%%%%%%%%%%%%%%%%%%%%%%%%%%%%%%%
% Beamer Presentation
% LaTeX Template
% Version 1.0 (10/11/12)
%
% This template has been downloaded from:
% http://www.LaTeXTemplates.com
%
% License:
% CC BY-NC-SA 3.0 (http://creativecommons.org/licenses/by-nc-sa/3.0/)
%
%%%%%%%%%%%%%%%%%%%%%%%%%%%%%%%%%%%%%%%%%

%----------------------------------------------------------------------------------------
%	PACKAGES AND THEMES
%----------------------------------------------------------------------------------------

\documentclass{beamer}

\mode<presentation> {

% The Beamer class comes with a number of default slide themes
% which change the colors and layouts of slides. Below this is a list
% of all the themes, uncomment each in turn to see what they look like.

%\usetheme{default}
%\usetheme{AnnArbor}
%\usetheme{Antibes}
%\usetheme{Bergen}
%\usetheme{Berkeley}
%\usetheme{Berlin}
%\usetheme{Boadilla}
%\usetheme{CambridgeUS}
%\usetheme{Copenhagen}
%\usetheme{Darmstadt}
%\usetheme{Dresden}
%\usetheme{Frankfurt}
%\usetheme{Goettingen}
%\usetheme{Hannover}
%\usetheme{Ilmenau}
%\usetheme{JuanLesPins}
%\usetheme{Luebeck}
\usetheme{Madrid}
\usepackage{pbox}
\usepackage{listings}
\lstset{language=Java,
                basicstyle=\footnotesize\ttfamily,
                keywordstyle=\footnotesize\color{blue}\ttfamily,
}

%\usetheme{Malmoe}
%\usetheme{Marburg}
%\usetheme{Montpellier}
%\usetheme{PaloAlto}
%\usetheme{Pittsburgh}
%\usetheme{Rochester}
%\usetheme{Singapore}
%\usetheme{Szeged}
%\usetheme{Warsaw}

% As well as themes, the Beamer class has a number of color themes
% for any slide theme. Uncomment each of these in turn to see how it
% changes the colors of your current slide theme.

%\usecolortheme{albatross}
%\usecolortheme{beaver}
%\usecolortheme{beetle}
%\usecolortheme{crane}
%\usecolortheme{dolphin}
%\usecolortheme{dove}
%\usecolortheme{fly}
%\usecolortheme{lily}
%\usecolortheme{orchid}
%\usecolortheme{rose}
%\usecolortheme{seagull}
%\usecolortheme{seahorse}
%\usecolortheme{whale}
%\usecolortheme{wolverine}

%\setbeamertemplate{footline} % To remove the footer line in all slides uncomment this line
%\setbeamertemplate{footline}[page number] % To replace the footer line in all slides with a simple slide count uncomment this line

%\setbeamertemplate{navigation symbols}{} % To remove the navigation symbols from the bottom of all slides uncomment this line
}

\usepackage{graphicx} % Allows including images
\usepackage{booktabs} % Allows the use of \toprule, \midrule and \bottomrule in tables
\usepackage{framed}

%----------------------------------------------------------------------------------------
%	TITLE PAGE
%----------------------------------------------------------------------------------------

\title[Eccezioni]{Eccezioni e Iteratori} % The short title appears at the bottom of every slide, the full title is only on the title page

\author{Claudio Menghi} % Your name
\institute[Polimi] % Your institution as it will appear on the bottom of every slide, may be shorthand to save space
{
Politecnico di Milano \\ % Your institution for the title page
\medskip
\textit{claudio.menghi@polimi.it} % Your email address
}
\date{\today} % Date, can be changed to a custom date

\begin{document}

\begin{frame}
\titlepage % Print the title page as the first slide
\end{frame}

\begin{frame}
\frametitle{Overview} % Table of contents slide, comment this block out to remove it
\tableofcontents % Throughout your presentation, if you choose to use \section{} and \subsection{} commands, these will automatically be printed on this slide as an overview of your presentation
\end{frame}

%----------------------------------------------------------------------------------------
%	PRESENTATION SLIDES
%----------------------------------------------------------------------------------------


%------------------------------------------------


\section{Eccezioni}

\begin{frame}
\frametitle{Eccezioni}
\begin{framed}
Le eccezioni sono eventi che accadono durante l'esecuzione del programma  che ``disturbano" il normale flusso di esecuzione delle istruzioni. 
\end{framed}
\end{frame}

\begin{frame}
\frametitle{Eccezioni}
\begin{itemize}
\item quando un eccezione ``avviene" viene \textbf{creato} un oggetto di tipo eccezione che contiene le informazioni riguardanti l'errore includendone il tipo e lo stato dell'applicazione al momento della generazione dell'eccezione.
\item dopo essere creata l'eccezione viene \textbf{lanciata}
\item l'eccezione viene \textbf{propagata} finch\`e non \`e presente un metodo che consente di gestirla.
\begin{itemize}
\item L'insieme dei metodi che sono stati chiamati prima che l'eccezione \`e stata lanciata (\emph{call stack} costituisce l'insieme di possibili elementi che possono gestire l'eccezione). 
\item La ricerca avviene dall'ultimo metodo eseguito risalendo lo stack delle chiamate fino al \texttt{main}
\item il blocco di codice in grado di gestire l'eccezione viene detto \emph{exception handler}.
\end{itemize}
\item se non ci sono metodi capaci di catchare l'eccezione l'applicazione termina
\end{itemize}
\end{frame}



\subsection{Tipi di eccezioni}
\begin{frame}
\frametitle{Tipi di eccezioni}
Ci sono tre tipi di eccezioni
\begin{itemize}
\item \textbf{checked}: sono condizioni eccezionali che \emph{l'applicazione deve essere capace di gestire}. Per esempio se l'applicazione richiede all'utente di inserire il nome di un file non corretto la classe \texttt{FileReader} genera un \texttt{FileNotFoundException}. Una applicazione ben scritta deve notificare all'utente l'errore e permettegli di inserire il nome corretto. Le eccezioni checked possono essere catchate mediante il \texttt{Catch} statement.
\item \textbf{unchecked}: (o run-time): queste condizioni sono interne all'applicazione e, in genere, non \`e possibile anticiparle ne eseguire un recover. In genere corrispondono a \emph{bug} di programmazione, per esempio un incorretto uso delle API. Supponiamo per esempio di passare al costruttore di \texttt{FileReader} un valore \texttt{null}, stiamo utilizzando impropriamente le sue API.
\end{itemize}
\end{frame}


\subsection{Eccezioni unchecked}
\begin{frame}[fragile]
\frametitle{Eccezioni unchecked}
\begin{itemize}
\item sono sottoclasse di \textbf{RuntimeException}
\item Non devono essere gestite dal client 
\item non devono essere dichiarate.
\end{itemize}
\begin{framed}
\begin{lstlisting}[language=Java]
public boolean mossaValida(Scacchiera scacchiera, 
        Casella casellaFinale) {
    if(scacchiera==null){
        throw new NullPointerException(
            "La scacchiera non puo' essere nulla");
    }
    if(casellaFinale==null){
        throw new NullPointerException(
            "La casella finale non puo' essere nulla");
    }
}
\end{lstlisting}
\end{framed}
\end{frame}




\subsection{Eccezioni checked}
\begin{frame}[fragile]
\frametitle{Eccezioni checked}
\begin{itemize}
\item sono condizioni eccezionali che \emph{l'applicazione deve essere capace di gestire} 
\item  per esempio se l'applicazione richiede all'utente di inserire il nome di un file non corretto la classe \texttt{FileReader} genera un \texttt{FileNotFoundException}
\end{itemize}
\begin{framed}
\begin{lstlisting}[language=Java]
public boolean mossaValida(Scacchiera scacchiera, 
        Casella casellaFinale) {
    if(!casellaFinale.isEmpty()){
        throw new MossaNonValidaException(
            "Non e' possibile muovere una pedina in una casella gia' occupata");
    }
}
\end{lstlisting}
L'actual type parameter di \texttt{b1} \`e \texttt{String}.
\end{framed}
\end{frame}


\begin{frame}[fragile]
\frametitle{Eccezioni checked}
Le eccezioni checked possono essere gestite in due modi:
\begin{itemize}
\item utilizzando il costrutto \texttt{try}, \texttt{catch}
\item propaganco l'eccezioni
\end{itemize}
\end{frame}

\begin{frame}[fragile]
\frametitle{Eccezioni checked: try catch}
\begin{lstlisting}[language=Java]
try{
    FileInputStream fileInputStream=new 
        FileInputStream("file.txt");        
}
catch(FileNotFoundException e){
    System.err.println("File not found");
    // codice per gestire l'eccezione
}
\end{lstlisting}
\end{frame}


\begin{frame}[fragile]
\frametitle{Eccezioni checked: propagazione}
\begin{lstlisting}[language=Java]
public void doSomething() throws FileNotFoundException{
    ....
    
    FileInputStram file=new FileInputStream("file.txt");
    ...
}
\end{lstlisting}
\end{frame}

\begin{frame}[fragile]
\frametitle{Creare un eccezione checked}
\begin{lstlisting}[language=Java]
public class MossaNonValida extends Exception {
    /*
     * identifica unicamente la versione della classe 
     */
     private static final long serialVersionUID = 
	                -5605493852164516896L;

     public MossaNonValida(String message){
          super(message);
     }
}
\end{lstlisting}
\end{frame}

\begin{frame}[fragile]
\frametitle{Eccezione unchecked}
Considerato che le eccezioni \textbf{unchecked} in genere causano il crash dell'applicazione come posso evitarle?
\begin{itemize}
\item Ask for forgiveness
\begin{lstlisting}[language=Java]
try{
    set.add(new Complex(1.0,1.0));
}catch(FullStackException e){
    System.err.println("The stack is full");
}
\end{lstlisting}
\item Ask for permission
\begin{lstlisting}[language=Java]
if(!set.isFull()){
    set.add(new Complex(1.0, 1.0));
}
\end{lstlisting}
\item In java si utilizza la seconda convenzione
\end{itemize}
\end{frame}

\begin{frame}[fragile]
\frametitle{Creare un eccezione unchecked}
\begin{lstlisting}[language=Java]
public class FullStackException extends RuntimeException {
    /*
     * identifica unicamente la versione della classe 
     */
     private static final long serialVersionUID = 
	                -5605493852164516896L;

}
\end{lstlisting}
\end{frame}

\subsection{Il blocco finally}
\begin{frame}[fragile]
\frametitle{Pillole di buona programmazione}
Oltre al blocco catch \`e possibile specificare un blocco finally
\begin{itemize}
\item contiene le operazione che sono eseguite in ogni caso, sia che le operazioni vanno a buon fine sia se \`e generata un eccezione
\item solitamente viene usato per rilasciare le risorse utilizzate (per esempio i file aperti)
 \end{itemize}
\end{frame}

\subsection{Pillole di buona programmazione}
\begin{frame}[fragile]
\frametitle{Pillole di buona programmazione}
\begin{itemize}
\item utilizzare le eccezioni solo per condizioni eccezionali
\item non scrivere il seguente codice
\begin{lstlisting}[language=Java]
try{
        int i=0;
        while(true){
            range[i++].climb();
        }
    }catch(ArrayIndexOutOfBoundException e){
    }
}
\end{lstlisting}
\end{itemize}
\end{frame}


\begin{frame}[fragile]
\frametitle{Pillole di buona programmazione}
\begin{itemize}
\item utilizzare le eccezioni checked per condizioni recuperabili 
\item utilizzare eccezioni unchecked per errori di programmazione
\end{itemize}
\end{frame}

\begin{frame}[fragile]
\frametitle{Pillole di buona programmazione}
\begin{itemize}
\item utilizzare il pi\`u possibile le eccezioni standard di Java
\item le pi\`u comuni sono:
\begin{itemize}
\item \texttt{IllegalArgumentException}: valori non nulli ma parametri incorretti
\item \texttt{IllegalStateException}: lo stato dell'oggetto non \`e appropriato per l'invocazione del metodo 
\item \texttt{NullPointerException}: il parametro \`e null quando \`e proibito
\item \texttt{IndexOutOfBoundException}: valore fuori dal range consentito
\item \texttt{ConcurrentModificationException}: modifica concorrente eseguita quando proibito
\item \texttt{UnsupportedOperationException}: l'ogggetto non supporta il metodo
\end{itemize}
\end{itemize}
\end{frame}

\begin{frame}[fragile]
\frametitle{Pillole di buona programmazione}
\begin{itemize}
\item  documentare sempre le eccezioni di ogni metodo (sia checked che unchecked)
 \end{itemize}
\end{frame}


\begin{frame}[fragile]
\frametitle{Pillole di buona programmazione}
\begin{itemize}
\item  evitare i blocchi catch vuoti, almeno documentare il perch\`e un eccezione \`e ignorata
 \end{itemize}
\end{frame}



\section{Esercizi}


\begin{frame}[fragile]
\frametitle{Esercizio 1}
\begin{lstlisting}[language=Java]
public static void f() throws Exception{
    System.out.println("Throw new Exception f()");
    throw new Exception("Generated in f()");
}
public static void g() throws Exception{
    try {
        f();
    } catch (Exception e) {
     System.out.println("Exception catched in g(). Rethrow");
     throw e;
    }
}
public static void main(String[] args) {
    try{    g();
    }
    catch(Exception ex){
        System.out.println("Exception catched in main()"+
            "Just stay");
        System.out.println(ex.getMessage());
    }
}
\end{lstlisting}
\end{frame}

\begin{frame}[fragile]
\frametitle{Esercizio 1}
Throw new Exception f()\\
Exception catched in g(). Rethrow\\
Exception catched in main(). Just stay\\
Generated in f()
\end{frame}

\begin{frame}[fragile]
\frametitle{Esercizio 2}
\begin{lstlisting}[language=Java]
public static void f() throws Exception{
    System.out.println("Throw new Exception f()");
    throw new Exception("Generated in f()");
}
public static void g() throws Exception{
    try {
        f();
    } catch (Exception e) {
     System.out.println("Exception catched in g(). Rethrow");
     throw new Exception("Generated in g()");
    }
}
public static void main(String[] args) {
    try{    g();
    }
    catch(Exception ex){
        System.out.println("Exception catched in main()"+
            "Just stay");
        System.out.println(ex.getMessage());
    }
}
\end{lstlisting}
\end{frame}

\begin{frame}[fragile]
\frametitle{Esercizio 2}
Throw new Exception f()\\
Exception catched in g(). Rethrow\\
Exception catched in main(). Just stay\\
Generated in g()
\end{frame}


\begin{frame}[fragile]
\frametitle{Esercizio 3}
Specificare la causa di un eccezione
\begin{lstlisting}[language=Java]
public static void f() throws Exception{
    System.out.println("Throw new Exception f()");
    throw new Exception("Generated in f()");
}
public static void g() throws Exception{
    try {
        f();
    } catch (Exception e) {
     System.out.println("Exception catched in g(). Rethrow");
     throw new Exception("Generated in g()", e);
    }
}
public static void main(String[] args) {
    try{    g(); }
    catch(Exception ex){
        System.out.println("Exception catched in main()"+
            "Just stay");
        System.out.println(ex.getMessage());
    }
}
\end{lstlisting}
\end{frame}

\begin{frame}[fragile]
\frametitle{Esercizio 4}
Ordine dei catch
\begin{lstlisting}[language=Java]
public class MyException extends Exception {
    private static final long serialVersionUID =
         3112454329854888478L;
    public MyException(String message){
        super("This is an instance of MyExeption"+message);
    }
}
\end{lstlisting}
\end{frame}


\begin{frame}[fragile]
\frametitle{Esercizio 4}
Ordine dei catch
\begin{lstlisting}[language=Java]
public static void f() throws Exception{
    System.out.println("Throw new Exception f()");
    throw new MyException("Generated in f()");
}
public static void g() throws Exception{
    try{ f(); }
    catch(MyException ex){
        System.out.println("My exception catched in "+
            "g(). Rethrow");
    }
    catch (Exception e) {
        System.out.println("Exception catched in g()."+
            " Rethrow");
        throw new Exception("Generated in g()", e);
    }
}
\end{lstlisting}
\texttt{catch(MyException ex)} consuma l'eccezione, il ramo \texttt{catch (Exception e)} non viene eseguito
\end{frame}


\begin{frame}[fragile]
\frametitle{Esercizio 5}
Ordine dei catch
\begin{lstlisting}[language=Java]
public static void f() throws Exception{
    System.out.println("Throw new Exception f()");
    throw new MyException("Generated in f()");
}
public static void g() throws Exception{
    try{ f(); }
    catch (Exception e) {
        System.out.println("Exception catched in g()."+
            " Rethrow");
        throw new Exception("Generated in g()", e);
    }
    catch(MyException ex){
        System.out.println("My exception catched in "+
            "g(). Rethrow");
    }
    
}
\end{lstlisting}
Errore di compilazione
\end{frame}


\begin{frame}[fragile]
\frametitle{Esercizio 6}
\begin{lstlisting}[language=Java]
private void printFileContent(FileInputStream fileInputStream) {
 byte[] buffer = new byte[1024];
 int readBytes = 0;
 try {
   while ((readBytes = fileInputStream.read(buffer)) != 0) {
    for (int i = 0; i < readBytes; i++){
        System.out.print(buffer[i]);
    }
   }
 } catch (IOException e) {
   System.err.println("Error reading file content");
 } finally {
   try {
        fileInputStream.close();
   } catch (IOException e) {
       System.err.println("Error reading file");
   }
} 
}
\end{lstlisting}
\end{frame}

\section{Iteratori}

\section{Esercizi}

%-------------

%------------------------------------------------


%----------------------------------------------------------------------------------------

\end{document} 
%
%\documentclass{beamer}
%\usepackage[english]{babel}
%\usepackage{xcolor}
%\usepackage[utf8]{inputenc}
%%\usepackage[ngerman]{babel}
%\usepackage{hyperref}
%\usepackage{graphicx}
%\usepackage{listings} % Required for insertion of code
%%\usepackage[pdftex]{color}
%%\usepackage{colortbl}
%%\usepackage{listings}
%\usepackage{xcolor}
%\usepackage[lf]{venturis}
%\usepackage{amsmath}
%\usepackage{amsfonts}
%\usepackage{amssymb}
%\usepackage{subcaption}
%%\usepackage{multirow}
%\usepackage[version=3]{mhchem}
%%\usepackage{longtable}
%\usepackage{pgf}
%\usepackage{tikz}
%\usepackage[citetracker=true,sorting=none,backend=bibtex,autocite=footnote]{biblatex}
%{\fontsize{2}{2} \bibliography{bibliography}}
%
%\renewcommand{\bibfont}{\tiny}
%
%\definecolor{kugreen}{RGB}{6,76,149}
%\definecolor{kugreenlys}{RGB}{132,158,139}
%\definecolor{kugreenlyslys}{RGB}{173,190,177}
%\definecolor{kugreenlyslyslys}{RGB}{214,223,216}
%\setbeamercovered{transparent}
%
%
%
%
%
%
%\setbeamertemplate{navigation symbols}{}
%
%  
%\mode<presentation>
%{  \usetheme{Madrid}
%  \usecolortheme[named=kugreen]{structure}
%  \useinnertheme{circles}
%  \usefonttheme[onlymath]{serif}
%  \setbeamercovered{transparent}
%  \setbeamertemplate{blocks}[rounded][shadow=true]
%}
%
%
%
%
%%\setbeamerfont{myTOC}{series=\bfseries}
%\AtBeginSection[]{\frame{\frametitle{Outline}%
%                  \usebeamerfont{myTOC}\tableofcontents[current]}}
%                  
%\title[Agile Verification]{Agile Verification}
%\author[C. Menghi]{Student: Claudio Menghi \\ Supervisor: Carlo Ghezzi}
%\institute[]{Politecnico di Milano}
%\date{29 October 2014}
%
%\begin{document}
%\frame{
%\titlepage \vspace{-0.5cm}
% }
%
%
%\frame
%{
%\frametitle{Overview}
%\tableofcontents[]
%}
%

%
%
%%\section*{Bibliography}
%%\begin{frame}[allowframebreaks]{Bibliography}
%%\footnotesize{
%%\bibliographystyle{abbrv}
%%\bibliography{bibliography} }
%%\end{frame}
%
%\end{document}
