\documentclass{article}

\usepackage{fancyhdr} % Required for custom headers
\usepackage{lastpage} % Required to determine the last page for the footer
\usepackage{extramarks} % Required for headers and footers
\usepackage[usenames,dvipsnames]{color} % Required for custom colors
\usepackage{graphicx} % Required to insert images
\usepackage{listings} % Required for insertion of code
\usepackage{courier} % Required for the courier font
\usepackage{lipsum} % Used for inserting dummy 'Lorem ipsum' text into the template
\usepackage{hyperref}
\usepackage{multirow}
\usepackage{tabularx}
\usepackage{framed}
\usepackage{longtable}
\usepackage{listings}
\usepackage{subfigure}
\usepackage{afterpage}
\usepackage{amsmath,amssymb}            
\usepackage{rotating}  
\usepackage{fancyhdr}
\usepackage{graphicx}
\usepackage{amsthm}
\usepackage[scriptsize]{caption} 
\hyphenation{a-gen-tiz-za-zio-ne}
% Margins
\topmargin=-0.45in
\evensidemargin=0in
\oddsidemargin=0in
\textwidth=6.5in
\textheight=9.0in
\headsep=0.25in

\linespread{1.1} % Line spacing

\lstset{
  numbers=left,
  stepnumber=5,    
  firstnumber=1,
  numberfirstline=true
}

% Set up the header and footer
\pagestyle{fancy}
\lhead{\hmwkAuthorName} % Top left header
\chead{\hmwkClass\ (\hmwkClassInstructor\ \hmwkClassTime): \hmwkTitle} % Top center head
\rhead{\firstxmark} % Top right header
\lfoot{\lastxmark} % Bottom left footer
\cfoot{} % Bottom center footer
\rfoot{Page\ \thepage\ of\ \protect\pageref{LastPage}} % Bottom right footer
\renewcommand\headrulewidth{0.4pt} % Size of the header rule
\renewcommand\footrulewidth{0.4pt} % Size of the footer rule

\setlength\parindent{0pt} % Removes all indentation from paragraphs

\usepackage{listings}
\usepackage{color}

\definecolor{dkgreen}{rgb}{0,0.6,0}
\definecolor{gray}{rgb}{0.5,0.5,0.5}
\definecolor{mauve}{rgb}{0.58,0,0.82}

\lstset{frame=tb,
  language=Java,
  aboveskip=3mm,
  belowskip=3mm,
  showstringspaces=false,
  columns=flexible,
  basicstyle={\small\ttfamily},
  numbers=none,
  numberstyle=\tiny\color{gray},
  keywordstyle=\color{blue},
  commentstyle=\color{dkgreen},
  stringstyle=\color{mauve},
  breaklines=true,
  breakatwhitespace=true
  tabsize=3
}

%----------------------------------------------------------------------------------------
%	DOCUMENT STRUCTURE COMMANDS
%	Skip this unless you know what you're doing
%----------------------------------------------------------------------------------------

% Header and footer for when a page split occurs within a problem environment
\newcommand{\enterProblemHeader}[1]{
\nobreak\extramarks{#1}{#1 continued on next page\ldots}\nobreak
\nobreak\extramarks{#1 (continued)}{#1 continued on next page\ldots}\nobreak
}

% Header and footer for when a page split occurs between problem environments
\newcommand{\exitProblemHeader}[1]{
\nobreak\extramarks{#1 (continued)}{#1 continued on next page\ldots}\nobreak
\nobreak\extramarks{#1}{}\nobreak
}




\newcommand{\hmwkTitle}{05 - Eccezioni} % Assignment title
\newcommand{\hmwkDueDate}{Mercoledì,\ Aprile 5,\ 2016} % Due date
\newcommand{\hmwkClass}{Ingegneria del Software 1} % Course/class
\newcommand{\hmwkClassTime}{} % Class/lecture time
\newcommand{\hmwkClassInstructor}{Claudio Menghi, Alessandro Rizzi} % Teacher/lecturer
\newcommand{\hmwkAuthorName}{} % Your name

\author{\textbf{\hmwkAuthorName}}
\date{} % Insert date here if you want it to appear below your name

\newcounter{EsercizioCounter}
 \setcounter{EsercizioCounter}{1}


\newcommand{\Esercizio}[1]{
%\setlength{\fboxsep}{2pt}
\fbox{
   
  \parbox[t][]{\textwidth}{
   \vspace{2ex}
   \textbf{Esercizio \arabic{EsercizioCounter}}: #1
    \vspace{2ex}
    \refstepcounter{EsercizioCounter}
  }
}
}


%----------------------------------------------------------------------------------------
%	TITLE PAGE
%----------------------------------------------------------------------------------------

\title{
\vspace{2in}
\textmd{\textbf{\hmwkClass\\ \vspace{1cm} \hmwkTitle \vspace{1cm}}}\\
\normalsize\vspace{0.1in}\small{\hmwkDueDate}\\
\vspace{0.1in}\large{\textit{\hmwkClassInstructor\ \hmwkClassTime}}
\vspace{3in}
}

%----------------------------------------------------------------------------------------

\begin{document}

\maketitle

%----------------------------------------------------------------------------------------
%	TABLE OF CONTENTS
%----------------------------------------------------------------------------------------

%\setcounter{tocdepth}{1} % Uncomment this line if you don't want subsections listed in the ToC

\newpage
\tableofcontents
\newpage



\section{Introduzione}


\subsection{Eccezioni}
\begin{itemize}
\item Le eccezioni sono eventi che accadono durante l'esecuzione del programma  che ``disturbano" il normale flusso di esecuzione delle istruzioni. 
\item quando un eccezione ``avviene" viene \textbf{creato} un oggetto di tipo eccezione che contiene le informazioni riguardanti l'errore includendone il tipo e lo stato dell'applicazione al momento della generazione dell'eccezione.
\item dopo essere creata l'eccezione viene \textbf{lanciata}
\item l'eccezione viene \textbf{propagata} finch\`e non \`e presente un metodo che consente di gestirla.
\begin{itemize}
\item L'insieme dei metodi che sono stati chiamati prima che l'eccezione \`e stata lanciata (\emph{call stack} costituisce l'insieme di possibili elementi che possono gestire l'eccezione). 
\item La ricerca avviene dall'ultimo metodo eseguito risalendo lo stack delle chiamate fino al \texttt{main}
\item il blocco di codice in grado di gestire l'eccezione viene detto \emph{exception handler}.
\end{itemize}
\item se non ci sono metodi capaci di catchare l'eccezione l'applicazione termina
\end{itemize}

Ci sono due tipi di eccezioni
\begin{itemize}
\item \emph{checked}: sono condizioni eccezionali che \emph{l'applicazione deve essere capace di gestire}. Per esempio se l'applicazione richiede all'utente di inserire il nome di un file non corretto la classe \texttt{FileReader} genera un \texttt{FileNotFoundException}. Una applicazione ben scritta deve notificare all'utente l'errore e permettegli di inserire il nome corretto. Le eccezioni checked possono essere catchate mediante il \texttt{Catch} statement.
\item \emph{unchecked}: (o run-time): queste condizioni sono interne all'applicazione e, in genere, non \`e possibile anticiparle ne eseguire un recover. In genere corrispondono a \emph{bug} di programmazione, per esempio un incorretto uso delle API. Supponiamo per esempio di passare al costruttore di \texttt{FileReader} un valore \texttt{null}, stiamo utilizzando impropriamente le sue API.
%\item \emph{Error}: sono condizioni eccezionali che sono esterne all'applicazione, ma che l'applicazione deve prevedere e possibilmente deve poter ``riprendersi". Per esempio, nel caso di una lettura file, nel caso di un malfunzionamento del sistema viene generato un \texttt{IOError}. Non \`e possibile effettuare un \texttt{catch} degli  errori.  
\end{itemize}


\subsection{Eccezioni unchecked}
\begin{itemize}
\item sono sottoclasse di \textbf{RuntimeException}
\item Non devono essere gestite dal client 
\item non devono essere dichiarate.
\item in genere corrispondono a bug di programmazione
\end{itemize}

Le eccezioni unchecked sono utilizzate per esempio per controllare che i parametrei passati a un metodo non sono nulli. Per esempio, il metodo \texttt{mossaValida} della classe scacchiera dovrebbe controllare che la scacchiera e la casellaFinale non siano nulle. In tal caso, lo sviluppatore sta utilizzando il metodo mossaValida in una maniera impropria.
\begin{lstlisting}[language=Java]
public boolean mossaValida(Scacchiera scacchiera, 
        Casella casellaFinale) {
    if(scacchiera==null){
        throw new NullPointerException(
            "La scacchiera non puo' essere nulla");
    }
    if(casellaFinale==null){
        throw new NullPointerException(
            "La casella finale non puo' essere nulla");
    }
}
\end{lstlisting}


Considerato che le eccezioni \textbf{unchecked} in genere causano il crash dell'applicazione lo sviluppatore che utilizza la nostra classe vorrebbe evitare di generare tali eccezioni. A tal fine pu\`o utilizzare due strategie:
\begin{itemize}
\item Ask for forgiveness
\begin{lstlisting}[language=Java]
try{
    set.add(new Complex(1.0,1.0));
}catch(FullStackException e){
    System.err.println("The stack is full");
}
\end{lstlisting}
\item Ask for permission
\begin{lstlisting}[language=Java]
if(!set.isFull()){
    set.add(new Complex(1.0, 1.0));
}
\end{lstlisting}

\end{itemize}
 In java si utilizza la seconda convenzione.
\`E possibile dichiarare delle eccezioni \texttt{Unchecked} mediante una classe che estende \texttt{RuntimeException}

\begin{lstlisting}[language=Java]
public class FullStackException extends RuntimeException {
    /*
     * identifica unicamente la versione della classe 
     */
     private static final long serialVersionUID = 
	                -5605493852164516896L;

}
\end{lstlisting}



\subsection{Eccezioni checked}
\begin{itemize}
\item sono condizioni eccezionali che \emph{l'applicazione deve essere capace di gestire} 
\item  per esempio se l'applicazione richiede all'utente di inserire il nome di un file non corretto la classe \texttt{FileReader} genera un \texttt{FileNotFoundException}
\end{itemize}

\begin{lstlisting}[language=Java]
public boolean mossaValida(Scacchiera scacchiera, 
        Casella casellaFinale) {
    if(!casellaFinale.isEmpty()){
        throw new MossaNonValidaException(
            "Non e' possibile muovere una pedina in una casella gia' occupata");
    }
}
\end{lstlisting}
L'actual type parameter di \texttt{b1} \`e \texttt{String}.


Le eccezioni checked possono essere gestite in due modi:
\begin{itemize}
\item utilizzando il costrutto \texttt{try}, \texttt{catch}
\begin{lstlisting}[language=Java]
try{
    FileInputStream fileInputStream=new 
        FileInputStream("file.txt");        
}
catch(FileNotFoundException e){
    System.err.println("File not found");
    // codice per gestire l'eccezione
}
\end{lstlisting}
\item propagando le eccezioni
\begin{lstlisting}[language=Java]
public void doSomething() throws FileNotFoundException{
    ....
    
    FileInputStram file=new FileInputStream("file.txt");
    ...
}
\end{lstlisting}
\end{itemize}

\`E possibile dichiarare delle eccezioni checked mediante una classe che estende \texttt{Exception}

\begin{lstlisting}[language=Java]
public class MossaNonValida extends Exception {
    /*
     * identifica unicamente la versione della classe 
     */
     private static final long serialVersionUID = 
	                -5605493852164516896L;

     public MossaNonValida(String message){
          super(message);
     }
}
\end{lstlisting}


\subsection{Il blocco finally}
Oltre al blocco catch \`e possibile specificare un blocco finally
\begin{itemize}
\item contiene le operazione che sono eseguite in ogni caso, sia che le operazioni vanno a buon fine sia se \`e generata un eccezione
\item solitamente viene usato per rilasciare le risorse utilizzate (per esempio i file aperti)
 \end{itemize}

\subsection{Pillole di buona programmazione}
\begin{itemize}
\item utilizzare le eccezioni solo per condizioni eccezionali (non scrivere il seguente)
\begin{lstlisting}[language=Java]
try{
        int i=0;
        while(true){
            range[i++].climb();
        }
    }catch(ArrayIndexOutOfBoundException e){
    }
}
\end{lstlisting}
\end{itemize}

\begin{itemize}
\item utilizzare le eccezioni checked per condizioni recuperabili 
\item utilizzare eccezioni unchecked per errori di programmazione
\end{itemize}
\begin{itemize}
\item utilizzare il pi\`u possibile le eccezioni standard di Java
\item le pi\`u comuni sono:
\begin{itemize}
\item \texttt{IllegalArgumentException}: valori non nulli ma parametri incorretti
\item \texttt{IllegalStateException}: lo stato dell'oggetto non \`e appropriato per l'invocazione del metodo 
\item \texttt{NullPointerException}: il parametro \`e null quando \`e proibito
\item \texttt{IndexOutOfBoundException}: valore fuori dal range consentito
\item \texttt{ConcurrentModificationException}: modifica concorrente eseguita quando proibito
\item \texttt{UnsupportedOperationException}: l'ogggetto non supporta il metodo
\end{itemize}
\end{itemize}
\begin{itemize}
\item  documentare sempre le eccezioni di ogni metodo (sia checked che unchecked)
 \end{itemize}


\begin{itemize}
\item  evitare i blocchi catch vuoti, almeno documentare il perch\`e un eccezione \`e ignorata
 \end{itemize}






\section{Esercizi}

\subsection{Esercizio}
\Esercizio{che cosa viene stampato quando viene invocato il metodo \texttt{main}?}

\begin{lstlisting}[language=Java]
public static void f() throws Exception{
    System.out.println("Throw new Exception f()");
    throw new Exception("Generated in f()");
}
public static void g() throws Exception{
    try {
        f();
    } catch (Exception e) {
     System.out.println("Exception catched in g(). Rethrow");
     throw e;
    }
}
public static void main(String[] args) {
    try{    g();
    }
    catch(Exception ex){
        System.out.println("Exception catched in main()"+
            "Just stay");
        System.out.println(ex.getMessage());
    }
}
\end{lstlisting}

Quando il metodo \texttt{f} viene stampato: \textit{Throw new Exception f()}\\
Subito dopo viene creata una nuova eccezione con messaggio ``Generated in f()"  che viene lanciata. L'eccezione risale lo stack trace e viene catchata dal \texttt{catch} di \texttt{g} 
Il \texttt{catch} di \texttt{g} stampa \textit{Exception catched in g(). Rethrow} e rilancia l'eccezione utilizzando la keyword \texttt{throw}.
Il metodo main esegue il catch dell'eccezione, stampa \textit{Exception catched in main(). Just stay} e il messaggio contenuto all'interno dell'eccezione: \textit{Generated in f()}.

\subsection{Esercizio}
\Esercizio{che cosa viene stampato quando viene invocato il metodo \texttt{main}?}

\begin{lstlisting}[language=Java]
public static void f() throws Exception{
    System.out.println("Throw new Exception f()");
    throw new Exception("Generated in f()");
}
public static void g() throws Exception{
    try {
        f();
    } catch (Exception e) {
         System.out.println("Exception catched in g(). Rethrow");
         throw new Exception("Generated in g()");
    }
}
public static void main(String[] args) {
    try{    g();
    }
    catch(Exception ex){
        System.out.println("Exception catched in main()"+
            "Just stay");
        System.out.println(ex.getMessage());
    }
}
\end{lstlisting}
Quando il metodo \texttt{f} viene stampato: \textit{Throw new Exception f()}\\
Subito dopo viene creata una nuova eccezione con messaggio ``Generated in f()"  che viene lanciata. L'eccezione risale lo stack trace e viene catchata dal \texttt{catch} di \texttt{g} 
Il \texttt{catch} di \texttt{g} stampa \textit{Exception catched in g(). Rethrow} e crea una nuova eccezione con messaggio ``Generated in g()" che vien lanciata utilizzando la keyword \texttt{throw}.
Il metodo main esegue il catch dell'eccezione, stampa \textit{Exception catched in main(). Just stay} e il messaggio contenuto all'interno dell'eccezione: \textit{Generated in g()}.



\subsection{Esercizio}
\Esercizio{modificare il codice dell'esercizio precedente utilizzand il costruttore \texttt{Exception(String message, Throwable cause)} all'interno di \texttt{g} al fine di specificare la causa dell'eccezione}


\begin{lstlisting}[language=Java]
public static void f() throws Exception{
    System.out.println("Throw new Exception f()");
    throw new Exception("Generated in f()");
}
public static void g() throws Exception{
    try {
        f();
    } catch (Exception e) {
     System.out.println("Exception catched in g(). Rethrow");
     throw new Exception("Generated in g()", e);
    }
}
public static void main(String[] args) {
    try{    g(); }
    catch(Exception ex){
        System.out.println("Exception catched in main()"+
            "Just stay");
        System.out.println(ex.getMessage());
    }
}
\end{lstlisting}

\subsection{Esercizio}
\Esercizio{discutere il seguente codice che cosa viene stampato? perch\`e?}

Ordine dei catch
\begin{lstlisting}[language=Java]
public class MyException extends Exception {
    private static final long serialVersionUID =
         3112454329854888478L;
    public MyException(String message){
        super("This is an instance of MyExeption"+message);
    }
}
\end{lstlisting}

Ordine dei catch
\begin{lstlisting}[language=Java]
public static void f() throws Exception{
    System.out.println("Throw new Exception f()");
    throw new MyException("Generated in f()");
}
public static void g() throws Exception{
    try{ f(); }
    catch(MyException ex){
        System.out.println("My exception catched in "+
            "g(). Rethrow");
    }
    catch (Exception e) {
        System.out.println("Exception catched in g()."+
            " Rethrow");
        throw new Exception("Generated in g()", e);
    }
}
\end{lstlisting}
Il ramo \texttt{catch(MyException ex)} consuma l'eccezione, il ramo \texttt{catch (Exception e)} non viene eseguito

\subsection{Esercizio}
\Esercizio{discutere il seguente codice che cosa viene stampato? perch\`e?}
Ordine dei catch
\begin{lstlisting}[language=Java]
public static void f() throws Exception{
    System.out.println("Throw new Exception f()");
    throw new MyException("Generated in f()");
}
public static void g() throws Exception{
    try{ f(); }
    catch (Exception e) {
        System.out.println("Exception catched in g()."+
            " Rethrow");
        throw new Exception("Generated in g()", e);
    }
    catch(MyException ex){
        System.out.println("My exception catched in "+
            "g(). Rethrow");
    }
    
}
\end{lstlisting}
Errore di compilazione

\subsection{Esercizio}
\Esercizio{scrivere una funzione che legge un byte alla volta da un \texttt{FileInputStream}}

\begin{lstlisting}[language=Java]
private void printFileContent(FileInputStream fileInputStream) {
 byte[] buffer = new byte[1024];
 int readBytes = 0;
 try {
   while ((readBytes = fileInputStream.read(buffer)) != 0) {
    for (int i = 0; i < readBytes; i++){
        System.out.print(buffer[i]);
    }
   }
 } catch (IOException e) {
   System.err.println("Error reading file content");
 } finally {
   try {
        fileInputStream.close();
   } catch (IOException e) {
       System.err.println("Error reading file");
   }
} 
}
\end{lstlisting}

\section{Esercizi per casa}
\begin{itemize}
\item Esercizio:
\begin{itemize}
\item modificare opportunamente l'esempio relativo al gioco degli scacchi utilizzando le eccezioni
\end{itemize} 
\end{itemize}

\clearpage

% ---- Bibliography ----




\addcontentsline{toc}{chapter}{Bibliography}
\bibliographystyle{alpha}
\bibliography{bib}
\nocite{*}


\end{document}

