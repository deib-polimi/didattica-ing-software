%%%%%%%%%%%%%%%%%%%%%%%%%%%%%%%%%%%%%%%%%
% Beamer Presentation
% LaTeX Template
% Version 1.0 (10/11/12)
%
% This template has been downloaded from:
% http://www.LaTeXTemplates.com
%
% License:
% CC BY-NC-SA 3.0 (http://creativecommons.org/licenses/by-nc-sa/3.0/)
%
%%%%%%%%%%%%%%%%%%%%%%%%%%%%%%%%%%%%%%%%%

%----------------------------------------------------------------------------------------
%	PACKAGES AND THEMES
%----------------------------------------------------------------------------------------

\documentclass{beamer}

\mode<presentation> {

% The Beamer class comes with a number of default slide themes
% which change the colors and layouts of slides. Below this is a list
% of all the themes, uncomment each in turn to see what they look like.

%\usetheme{default}
%\usetheme{AnnArbor}
%\usetheme{Antibes}
%\usetheme{Bergen}
%\usetheme{Berkeley}
%\usetheme{Berlin}
%\usetheme{Boadilla}
%\usetheme{CambridgeUS}
%\usetheme{Copenhagen}
%\usetheme{Darmstadt}
%\usetheme{Dresden}
%\usetheme{Frankfurt}
%\usetheme{Goettingen}
%\usetheme{Hannover}
%\usetheme{Ilmenau}
%\usetheme{JuanLesPins}
%\usetheme{Luebeck}
\usetheme{Madrid}
%\usetheme{Malmoe}
%\usetheme{Marburg}
%\usetheme{Montpellier}
%\usetheme{PaloAlto}
%\usetheme{Pittsburgh}
%\usetheme{Rochester}
%\usetheme{Singapore}
%\usetheme{Szeged}
%\usetheme{Warsaw}

% As well as themes, the Beamer class has a number of color themes
% for any slide theme. Uncomment each of these in turn to see how it
% changes the colors of your current slide theme.

%\usecolortheme{albatross}
%\usecolortheme{beaver}
%\usecolortheme{beetle}
%\usecolortheme{crane}
%\usecolortheme{dolphin}
%\usecolortheme{dove}
%\usecolortheme{fly}
%\usecolortheme{lily}
%\usecolortheme{orchid}
%\usecolortheme{rose}
%\usecolortheme{seagull}
%\usecolortheme{seahorse}
%\usecolortheme{whale}
%\usecolortheme{wolverine}

%\setbeamertemplate{footline} % To remove the footer line in all slides uncomment this line
%\setbeamertemplate{footline}[page number] % To replace the footer line in all slides with a simple slide count uncomment this line

%\setbeamertemplate{navigation symbols}{} % To remove the navigation symbols from the bottom of all slides uncomment this line
}
\usepackage[T1]{fontenc}

\usepackage{graphicx} % Allows including images
\usepackage{booktabs} % Allows the use of \toprule, \midrule and \bottomrule in tables
\usepackage{array,multirow,graphicx}
\usepackage{todonotes}
\newcommand{\todosk}[1]{\todo[inline]{#1}} 
\usepackage{media9} 
\usepackage{listings} % Required for insertion of code
\usepackage{courier} % Required for the courier font
\usepackage{listings}
\usepackage{color}
\usepackage{hyperref}
\usepackage{verbatim}
\usepackage[autostyle]{csquotes}


\definecolor{dkgreen}{rgb}{0,0.6,0}
\definecolor{gray}{rgb}{0.5,0.5,0.5}
\definecolor{mauve}{rgb}{0.58,0,0.82}

\lstset{frame=tb,
  language=Bash,
  aboveskip=3mm,
  belowskip=3mm,
  showstringspaces=false,
  columns=flexible,
  basicstyle={\small\ttfamily},
  numbers=none,
  numberstyle=\tiny\color{gray},
  keywordstyle=\color{blue},
  commentstyle=\color{dkgreen},
  stringstyle=\color{mauve},
  breaklines=true,
  breakatwhitespace=true
  tabsize=3
}


%----------------------------------------------------------------------------------------
%	TITLE PAGE
%----------------------------------------------------------------------------------------

\title[Git Guide]{Git Guide} % The short title appears at the bottom of every slide, the full title is only on the title page

\author{Sr\dj{}an Krsti\'c, Claudio Menghi} % Your name
\institute[] % Your institution as it will appear on the bottom of every slide, may be shorthand to save space
{
Politecnico di Milano \\ % Your institution for the title page
\medskip
\textit{srdan.krstic@polimi.it, claudio.menghi@polimi.it} % Your email address
}
\date{\today} % Date, can be changed to a custom date




\begin{document}

\begin{frame}
\titlepage % Print the title page as the first slide
\end{frame}


%----------------------------------------------------------------------------------------
%	PRESENTATION SLIDES
%----------------------------------------------------------------------------------------



\section{Overview}
\begin{frame}
\frametitle{Overview}


\begin{itemize}
\item Introduction
\item Using Git locally
  \begin{itemize}
  \item git init
  \item git status
  \item git add
  \item git commit
  \item git log
  \item git branch
  \item git checkout
  \item git merge
  \end{itemize}
\item Using Git remotely
  \begin{itemize}
  \item git clone
  \item git remote
  \item git pull
  \item git push
  \end{itemize}
\item Using Git with Eclipse
\end{itemize}
\end{frame}


% {
% %\usebackgroundtemplate{\includegraphics[width=\paperwidth]{cover1.jpg}}%
% \begin{frame}[plain]


% \end{frame}
% }

\begin{frame}

\LARGE	
\textbf{Introduction}


\end{frame}

\section{Introduction}
\begin{frame}
\frametitle{What is a version control system (VCS)?}


A version control system is a piece of software that helps the
developers to \textbf{collaborate} remotely and to \textbf{keep track} of the
history of their work. 

\end{frame}


\begin{frame}
\frametitle{Goals of VCS}

\begin{itemize}
\item enable simultaneous work
\item easily resolve conflicts
\item keep the history
\item also, to do all this fast and error-free
\end{itemize}
\end{frame}



\begin{frame}
\frametitle{First generation VCS}
Centralized versioning with full file locks

\begin{figure}

\includegraphics[scale=0.3]{figures/f4.png}
\includegraphics[scale=0.3]{figures/f3.png}
\end{figure}

\end{frame}

\begin{frame}
\frametitle{Second generation VCS}
Remote centralized versioning


\begin{figure}
\includegraphics[scale=0.3]{figures/f1.png}
\end{figure}


\end{frame}

\begin{frame}
\frametitle{Third generation VCS}
Remote distributed versioning


\begin{figure}
\includegraphics[scale=0.28]{figures/f2.png}
\end{figure}

\end{frame}


\begin{frame}

\LARGE	
\textbf{Using Git locally}


\end{frame}


\begin{frame}
\frametitle{Creating a Git repository}

\begin{itemize}
\item Creating an empty repository 
\item ``Cloning'' an existing repository (see Using Git remotely)
\end{itemize}

\end{frame}


\begin{frame}[fragile]
\frametitle{git init}
Creating a repository in folder "testrepo''
\small
\begin{lstlisting}
$ git init testrepo # creates a git repo 
$ cd testrepo       # enters testrepo folder
$ tree -a           # shows content
|-- .git
    |-- HEAD
    |-- config
    |-- description
    |-- hooks
    |-- info
    |   |-- exclude
    |-- objects
    |   |-- info
    |   |-- pack
    |-- refs
        |-- heads
        |-- tags
9 directories, 13 files
\end{lstlisting}

\end{frame}


\begin{frame}
\frametitle{Edit-Stage-Commit workflow}
~~~~~~~~Untracked files~~~~~~~~~~~~~~~Staged files~~~~~~~~~~~~Committed files
\begin{figure}
\includegraphics[scale=0.7]{figures/04.pdf}
\end{figure}

\end{frame}

\begin{frame}[fragile]
\frametitle{git add}

Staging files to be saved in the repository

\begin{lstlisting}
$ touch Main.java     #creates file Main.java
$ git add Main.java   #adds Main.java to stage
\end{lstlisting}

\end{frame}


\begin{frame}[fragile]
\frametitle{git status}

Checking the status of the working folder.

\begin{lstlisting}
$ git status #shows Main.java staged for commit
$ git touch Test.java #creates Test.java file
$ git status #shows also Test.java as untracked
$ git add Test.java
\end{lstlisting}
\end{frame}


\begin{frame}[fragile]
\frametitle{git commit}

Writing to the local repository

\begin{lstlisting}
$ git commit -m "First commit" #committing both files
$ git status #now there is nothing to commit
\end{lstlisting}
\end{frame}


\begin{frame}
\frametitle{First commit}
First commit
\begin{figure}
\includegraphics[scale=0.4]{figures/f8.png}
\end{figure}

\end{frame}


\begin{frame}
\frametitle{Commit representation}

\begin{figure}
\includegraphics[scale=0.3]{figures/f9.png}
\end{figure}

\end{frame}



\begin{frame}[fragile]
\frametitle{More commits}

Perform some more commits...

\begin{lstlisting}
# some modifications (edit + stage)
$ git commit -m "Second commit" 
$ git status 
\end{lstlisting}
\end{frame}


\begin{frame}[fragile]
\frametitle{git log}

\begin{lstlisting}
$ git log #shows all the commits
$ git log --oneline #more compact representation
$ git log --graph --oneline --decorate --all
\end{lstlisting}

The last command shows the most informative representation of the
commits.
\end{frame}

\begin{frame}[fragile]
\frametitle{Hands on - Committing (10 mins)}

Execute these commands:

\begin{lstlisting}
$ git init testrepo
$ cd testrepo
$ git status
$ # create some files
$ git status
$ git add <some files>
$ git status
$ git commit -m "<message>"
$ git status
$ git log
$ # create some more commits
$ git log
\end{lstlisting}
\end{frame}


\begin{frame}[fragile]
\frametitle{.gitignore}

Special file that forces \texttt{git status} to ignore files that you
do not want to version
\begin{lstlisting}
.classpath
.project
target/
.DS_Store
*~
\end{lstlisting}

\end{frame}



\begin{frame}
\frametitle{Changes as a directed acyclic graph (DAG)}

To create alternative and non-sequential versions, git uses the
concepts of branches.
\begin{figure}
\includegraphics[scale=0.3]{figures/f10.png}
\end{figure}

\end{frame}

\begin{frame}
\frametitle{Changes as a directed acyclic graph (DAG)}

...and modifications in different branches can be combined.
\begin{figure}
\includegraphics[scale=0.3]{figures/f11.png}
\end{figure}

\end{frame}


\begin{frame}
\frametitle{Changes as a directed acyclic graph (DAG)}

In general, at any moment git can maintain many different versions of
the repository.
\begin{figure}
\includegraphics[scale=0.25]{figures/f12.png}
\end{figure}

\end{frame}


\begin{frame}[fragile]
\frametitle{git branch}

Creating and modifying branches

\begin{lstlisting}
$ git branch feature #creates a branch called feature
$ git branch -m feature-GUI-Swing #renames the branch
$ git branch #lists existing branches
\end{lstlisting}
\end{frame}


\begin{frame}[fragile]
\frametitle{git checkout}

Switching between the different branches is done using \texttt{git
  checkout}.

\begin{lstlisting}
$ git checkout feature-GUI-Swing #switch branches
$ touch GUI.java 
$ # try to switch (1) between the branches
$ git add GUI.java
$ # try to switch (2) between the branches
$ git commit -m  "added gui" #on branch feature-GUI-Swing
$ # try to switch (3) between the branches
$ git log --graph --oneline --decorate --all
\end{lstlisting}

Notice that, after creating the GUI.java file,
switching (1) between branches would keep the file on both branches.
Also, when staged, GUI.java can be seen on both branches (2). Finally,
after committing the file, it can be only seen 
on feature-GUI-Swing branch (3). Therefore, you can decide on which
branch to commit, just before actual commit. However, there are
pitfalls: if you create a file that already exists on another branch,
you cannot switch to that branch before committing or
stashing/removing the file first.

\end{frame}


\begin{frame}
\frametitle{Merging branches (fast-forward merge)}

\begin{figure}
\includegraphics[scale=0.8]{figures/f13.png}
\end{figure}

\end{frame}

\begin{frame}
\frametitle{Merging branches (fast-forward merge)}

\begin{figure}
\includegraphics[scale=0.8]{figures/f14.png}
\end{figure}

\end{frame}


\begin{frame}
\frametitle{Merging branches (3-way merge)}

\begin{figure}
\includegraphics[scale=0.8]{figures/f15.png}\hspace{1.5cm}
\end{figure}

\end{frame}

\begin{frame}
\frametitle{Merging branches (3-way merge)}

\begin{figure}
\includegraphics[scale=0.8]{figures/f16.png}
\end{figure}

\end{frame}

\begin{frame}[fragile]
\frametitle{git merge (fast-forward)}

Performing a fast-forward merge:

\begin{lstlisting}
$ git checkout master #switch to master branch
$ git merge feature-GUI-Swing #merging feature-GUI-Swing
$ git log --graph --oneline --decorate 
$ 
\end{lstlisting}

Fast-forward merge can be done when both branches refer to commits in
the same sequence. In a ff merge above the master branch is updated to
the same commit as the one of the feature-GUI-Swing branch.

\end{frame}

\begin{frame}[fragile]
\frametitle{git branch -d}

Deleting branches

\begin{lstlisting}
$ git branch -d feature-GUI-Swing
$ git log --graph --oneline --decorate
\end{lstlisting}
\end{frame}


\begin{frame}[fragile]
\frametitle{git merge (3-way)}

Performing a 3-way merge

\begin{lstlisting}
$ git checkout -b feature-random #create and switch to feature-random branch
$ # perform some commits on feature-random branch
$ git checkout master
$ # perform some commits on master branch
$ git merge feature-random -m "merged with feature-random"
$ 
\end{lstlisting}

When performing a 3-way merge there are to divergent commits whose
changes git needs to combine into a single commit. In this case git
creates a new commit, hence we need to specify the commit message.
\end{frame}



\begin{frame}[fragile]
\frametitle{Hands on - Branching and Merging (10 mins)}

Execute these commands:

\begin{lstlisting}
$ git branch <branch name>
$ git branch
$ git checkout <branch name>
$ git branch
$ # perform some commits on <branch name>
$ git checkout master
$ # perform some commits on master
$ git merge <branch name>
\end{lstlisting}
\end{frame}

\begin{frame}
\frametitle{Auto-merge and Conflicts}

\begin{itemize}

\item Merge is just another edit-stage-commit workflow.

\item Git tries to automatically merge all files by combining them (edit
phase) adding them (stage phase) and committing them (commit phase).

\item However, If the two branches changed the same \textbf{line}
(or \textbf{adjacent} lines) of the same \textbf{file}, Git won't be able to figure out which
line to use in the final commit. 

\item When such a situation occurs, Git stops right before the commit so that
you can resolve the conflicts manually.

\end{itemize}

\end{frame}

\begin{frame}[fragile]
\frametitle{Resolving conflicts}
If we create a Main class and add an integer field "a'' in the master
branch while integer field "b'' in feature-random branch upon merging
we will get a conflict. Main.java file would look like this:
\begin{lstlisting}
$ cat Main.java
public class Main{
<<<<<<<<<< HEAD
int a;
==========
int b;
>>>>>>>>>> feature-random
}
\end{lstlisting}
Content between signs "< < < <'' and "===='' belong to the master
branch, while between "===='' and "> > > >'' belong to the
feature-random branch.
\end{frame}


\begin{frame}[fragile]
\frametitle{Resolving conflicts}
\begin{lstlisting}
$ # edit the file
$ git add Main.java
$ git commit -m "conflict resolved"
\end{lstlisting}
After editing, staging and committing the file
the conflict is resolved.
\end{frame}


\begin{frame}[fragile]
\frametitle{Hands on - Conflicts (10 mins)}

Execute these commands:

\begin{lstlisting}
$ # edit the <conflicted file>
$ git add <conflicted file>
$ git commit -m "<message>"
$ git log --graph --oneline --decorate --all
\end{lstlisting}
\end{frame}


\begin{frame}
\LARGE	
\textbf{Using Git remotely}
\end{frame}

\begin{frame}
\frametitle{Remote repository}

\begin{figure}
\includegraphics[scale=0.6]{figures/06.pdf}
\end{figure}

\end{frame}


\begin{frame}[fragile]
\frametitle{git remote}

Each git repository maintains a list of remote repositories in the
form (name, url).
The git remote command lets you create, view, and delete items from
this list.

\begin{lstlisting}
$ git remote add origin https://username@bitbucket.org/username/repo.git
$ git remote
\end{lstlisting}
The above \texttt{git remote add} command adds to the list the pair (origin, https://username@bitbucket.org/username/repo.git).

\end{frame}




\begin{frame}[fragile]
\frametitle{git push}
\begin{figure}

  \includegraphics[scale=0.3]{figures/f6.png}
\end{figure}
\end{frame}


\begin{frame}[fragile]
\frametitle{git push}

Push command pushes the current branch to the remote repository.

\begin{lstlisting}
$ git push --set-upstream origin master 
$ git branch -a
\end{lstlisting}

Git also maintains a list of (local branch, remote branch) pairs
called upstreams.

The \texttt{---set-upstream} or \texttt{-u} is used to bind the two
branches such that the later push and pull commands can refer to them
to update the remote or local repository, respectively.

The default behavior of push command is to push the current local
branch to its upstream remote branch (if their names match).

\texttt{git push ---all} command performs the default push behavior
for all local branches.


\end{frame}

\begin{frame}[fragile]
\frametitle{git push}

Push more branches...

\begin{lstlisting}
$ git checkout -b test
$ # make some commits to test branch
$ git push -u origin test 
$ git branch -a
$ git branch -vv
$ 
\end{lstlisting}

\texttt{git branch -vv} command shows all local branches and remote
branches that they are tracking.

\end{frame}

\begin{frame}
\frametitle{Push behavior}

Push has many pre-configured behaviors that can be chosen.
Using the configuration entry \texttt{push.default}
one can define the default action \texttt{git push} should
take. Possible values are:

\begin{itemize}
\item \textbf{nothing} - do not push anything (error out) unless a refspec is
  explicitly given. This is primarily meant for people who want to
  avoid mistakes by always being explicit. 

\item \textbf{current} - push the current branch to update a branch with the
  same name on the receiving end.

\item \textbf{upstream} - push the current branch back to the upstream
  branch. 

\item \textbf{matching} - push all branches having the same name on both
  ends. 

\item \textbf{simple} - works like a combination of upstream and
  matching. It will refuse to push if the upstream branch's name is
  different from the local one or if there is no upstream branch.
  \textbf{This mode has become the default in Git 2.0.}


\end{itemize}

\end{frame}




\begin{frame}[fragile]
\frametitle{git clone}

Finally we show how a repository can be cloned

\begin{lstlisting}
$ # use a different computer/folder
$ git clone https://username@bitbucket.org/username/repo.git
$ git branch
$ git branch -a
$ git branch -vv
$ 
\end{lstlisting}

\texttt{git clone} command fetches only the master branch from the remote
repository and sets the local master branch to track it.

\end{frame}


\begin{frame}[fragile]
\frametitle{git checkout --track}

Fetch command retrieves a specified remote branch.

\begin{lstlisting}
$ # stay on the second computer/folder
$ git branch -a
$ git branch -vv
$ git checkout -track origin/test
$ git branch -a
$ git branch -vv
\end{lstlisting}

In order to see the fetched branch you will need a local branch to
merge with. Therefore after a fetch we create a local test branch and
merge with the remote test branch. Finally with the push command we
set the local branch to track the remote branch (without pushing any
changes).


\end{frame}


\begin{frame}[fragile]
\frametitle{git pull}
\begin{figure}

  \includegraphics[scale=0.3]{figures/f7.png}
\end{figure}
\end{frame}

\begin{frame}[fragile]
\frametitle{git pull}

Pull command will perform fetching of the remote branch tracked by the
current local branch (via upstream), merge them with their
respective local branches and commit.
For example, \texttt{git pull} on the current master branch will pull
the origin/master branch and merge its commits with the local master branch.
\begin{lstlisting}
$ # stay on the second computer/folder
$ git checkout -b some-different-branch
$ git push -u origin some-different-branch
$ git branch -vv
$ # switch to the first computer/folder
$ git pull
$ git branch -vv
$ git checkout -b some-different-branch
$ git push -u origin some-different-branch
$ git branch -vv
\end{lstlisting}

\texttt{git pull ---all} command fetches all remote branches

\end{frame}

\begin{frame}[fragile]
\frametitle{Hands on - Remote repositories (15 mins)}

Execute these commands:

\begin{lstlisting}
$ # create bitbucket test repo (one person per group)
$ # add your colleagues to the repo
$ git remote add origin <URL>
$ git checkout master
$ git push -u origin master
$ # keep branchin', committin', pushin' and pullin'
\end{lstlisting}

Meanwhile your colleagues should execute:

\begin{lstlisting}
$ git clone <URL>
$ git branch -a
$ git branch -vv
$ # keep branchin', committin', pushin' and pullin'
$ 
\end{lstlisting}


\end{frame}


\begin{frame}
\LARGE	
\textbf{Using Git with Eclipse}
\end{frame}


\begin{frame}
\frametitle{git init}

\begin{itemize}
\item Right click on the project > Team > Share project
\item Choose git 
\item Click on Create and choose a path where you want to save the repo
\item Click on Finish
\end{itemize}

\end{frame}

\begin{frame}
\frametitle{git add}

Right click on the file > Team > Add to Index

\end{frame}

\begin{frame}
\frametitle{git commit}


\begin{itemize}
\item After adding the files...

\item Right click on the project > Team > Commit

\end{itemize}
\end{frame}


\begin{frame}
\frametitle{git status}

To see the status of the files look in the project explorer.
\begin{itemize}
\item Files with ? are untracked
\item Files with + are added
\item Files with <yellow cylinder> are committed
\item Files with > are modified
\end{itemize}

\end{frame}

\begin{frame}
\frametitle{git log}

Right click on the project > Team > Show in History

\end{frame}


\begin{frame}
\frametitle{git branch}

Right click on the project > Team > Switch To > New Branch...

\end{frame}

\begin{frame}
\frametitle{git checkout}

Right click on the project > Team > Switch To > <branch name>

\end{frame}

\begin{frame}
\frametitle{git merge}


\begin{itemize}
\item After switching to the appropriate branch:

\item Right click on the project > Team > Merge 

\item Then choose the other branch

\end{itemize}

\end{frame}


\begin{frame}
\frametitle{git clone}


\begin{itemize}
\item Right click on the project explorer > Import > Project from Git >
Clone URI 

\item Then paste the appropriate URL.

\end{itemize}

\end{frame}


\begin{frame}
\frametitle{git push}

\begin{itemize}
\item Right click on the project > Team > Push branch "<branch name>"
\item Next
\item Finish
\end{itemize}

\end{frame}

\begin{frame}
\frametitle{git pull}

Right click on the project > Team > Pull

\end{frame}



\begin{frame}
\frametitle{Git references}

\begin{itemize}
\item See our guide at BEEP!
\item \url{https://git-scm.com/}
\item \url{http://gitref.org/}
\item \url{https://progit.org/}
\item \url{https://www.atlassian.com/git/}
\item Eric Sink: Version Control by Example (Chapters 1, 4, 5, 6 and
  8)
\item Tech Talk: Linus Torvalds on git
\url{https://www.youtube.com/watch?v=4XpnKHJAok8}

\end{itemize}

\end{frame}


\end{document} 

