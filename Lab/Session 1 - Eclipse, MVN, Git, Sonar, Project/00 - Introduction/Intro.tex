%%%%%%%%%%%%%%%%%%%%%%%%%%%%%%%%%%%%%%%%%
% Beamer Presentation
% LaTeX Template
% Version 1.0 (10/11/12)
%
% This template has been downloaded from:
% http://www.LaTeXTemplates.com
%
% License:
% CC BY-NC-SA 3.0 (http://creativecommons.org/licenses/by-nc-sa/3.0/)
%
%%%%%%%%%%%%%%%%%%%%%%%%%%%%%%%%%%%%%%%%%

%----------------------------------------------------------------------------------------
%	PACKAGES AND THEMES
%----------------------------------------------------------------------------------------

\documentclass{beamer}

\mode<presentation> {

% The Beamer class comes with a number of default slide themes
% which change the colors and layouts of slides. Below this is a list
% of all the themes, uncomment each in turn to see what they look like.

%\usetheme{default}
%\usetheme{AnnArbor}
%\usetheme{Antibes}
%\usetheme{Bergen}
%\usetheme{Berkeley}
%\usetheme{Berlin}
%\usetheme{Boadilla}
%\usetheme{CambridgeUS}
%\usetheme{Copenhagen}
%\usetheme{Darmstadt}
%\usetheme{Dresden}
%\usetheme{Frankfurt}
%\usetheme{Goettingen}
%\usetheme{Hannover}
%\usetheme{Ilmenau}
%\usetheme{JuanLesPins}
%\usetheme{Luebeck}
\usetheme{Madrid}
%\usetheme{Malmoe}
%\usetheme{Marburg}
%\usetheme{Montpellier}
%\usetheme{PaloAlto}
%\usetheme{Pittsburgh}
%\usetheme{Rochester}
%\usetheme{Singapore}
%\usetheme{Szeged}
%\usetheme{Warsaw}

% As well as themes, the Beamer class has a number of color themes
% for any slide theme. Uncomment each of these in turn to see how it
% changes the colors of your current slide theme.

%\usecolortheme{albatross}
%\usecolortheme{beaver}
%\usecolortheme{beetle}
%\usecolortheme{crane}
%\usecolortheme{dolphin}
%\usecolortheme{dove}
%\usecolortheme{fly}
%\usecolortheme{lily}
%\usecolortheme{orchid}
%\usecolortheme{rose}
%\usecolortheme{seagull}
%\usecolortheme{seahorse}
%\usecolortheme{whale}
%\usecolortheme{wolverine}

%\setbeamertemplate{footline} % To remove the footer line in all slides uncomment this line
%\setbeamertemplate{footline}[page number] % To replace the footer line in all slides with a simple slide count uncomment this line

%\setbeamertemplate{navigation symbols}{} % To remove the navigation symbols from the bottom of all slides uncomment this line
}
\usepackage[T1]{fontenc}

\usepackage{graphicx} % Allows including images
\usepackage{booktabs} % Allows the use of \toprule, \midrule and \bottomrule in tables

%----------------------------------------------------------------------------------------
%	TITLE PAGE
%----------------------------------------------------------------------------------------

\title[Introduzione]{Introduzione} % The short title appears at the bottom of every slide, the full title is only on the title page

\author{Sr\dj{}an Krsti\'c, Marco Scavuzzo} % Your name
\institute[] % Your institution as it will appear on the bottom of every slide, may be shorthand to save space
{
Politecnico di Milano \\ % Your institution for the title page
\medskip
\textit{srdan.krstic@polimi.it, marco.scavuzzo@.polimi.it} % Your email address
}
\date{\today} % Date, can be changed to a custom date

\begin{document}

\begin{frame}
\titlepage % Print the title page as the first slide
\end{frame}

\begin{frame}
\frametitle{Overview} % Table of contents slide, comment this block out to remove it
\tableofcontents % Throughout your presentation, if you choose to use \section{} and \subsection{} commands, these will automatically be printed on this slide as an overview of your presentation
\end{frame}

%----------------------------------------------------------------------------------------
%	PRESENTATION SLIDES
%----------------------------------------------------------------------------------------


\section{Obiettivo}
\begin{frame}
\frametitle{Obiettivo}
La prova finale consiste nella preparazione, in autonomia ma sotto la guida del docente e dei responsabili di laboratorio, di un elaborato software da svolgere in Java. Il tema dell'elaborato,  proposto dal docente, sar\`a  di carattere interdisciplinare e riassuntivo dell'intero triennio, ma con particolare riferimento alle metodologie dell'Ingegneria del Software. Le esercitazioni introdurranno metodi e strumenti per la progettazione del software, quali ad esempio:
\begin{itemize}
\item L'ambiente Eclipse.
\item Strumenti di testing (JUnit)
\item Strumenti UML per Eclipse (Architexa, ...)
\item Strumenti di collaborazione (Git)
\item Strumenti di gestione di build e dipendenze  dei progetti (Maven)
\item ecc \ldots
\end{itemize}
\end{frame}


\section{Modalit\' a di svolgimento del laboratorio}
\begin{frame}
\frametitle{Modalit\` a di svolgimento del laboratorio}

%Obiettivo dell'attivit\` a di laboratorio \`e lo sviluppo di un elaborato da realizzare in Java.\\

 Lo studente deve dimostrare, tramite l'elaborato e opportune prove in
 laboratorio, la propria conoscenza dei \textbf{linguaggi},
 \textbf{metodi} e \textbf{strumenti} introdotti nelle esercitazioni e
 nell'ambito degli altri insegnamenti del corso di studi, con
 particolare riferimento all'insegnamento di Ingegneria del Software,
 e la propria capacit\` a di utilizzarli per la progettazione del
 software.\\

 L'attivit\` a di laboratorio \`e assistita dai responsabili, ma
 l'elaborato deve essere sviluppato dallo studente in modo autonomo.
\end{frame}


\begin{frame}
\frametitle{Modalit\` a di svolgimento del laboratorio}
\begin{itemize}
\item \` e richiesto il lavoro in team
\item composti da al pi\' u 3 persone
  (gruppi da due studenti sono consigliati)
\item la presenza \`e obbligatoria (1-2 assenze sono tollerabili) 
\item non \` e possibile creare gruppi che contengono studenti di scaglioni diversi
\item non copiate il codice tra diversi team (gli strumenti a disposizione
  permettono di capire se i diversi gruppi hanno copiato)
\item invece discussioni della progettazione tra diversi team sono incoraggiati
\end{itemize}
\end{frame}

\section{Valutazione}
\begin{frame}
\frametitle{Valutazione}
La valutazione dallo studente viene effettuata sul complesso del lavoro svolto e deriva dalla frequente interazione informale docente/studente. La frequenza al laboratorio assume quindi una particolare importanza. La mancata partecipazione ad alcune sessioni di laboratorio, in particolare a quelle in cui ci sono le valutazioni, pu\` o pertanto pregiudicare l'esito finale. Contattare tempestivamente i responsabili di laboratorio se si prevede di non potere seguire alcune sessioni.\\

\textbf{Non \`e previsto il recupero del laboratorio. Pertanto in caso di valutazione insufficiente lo studente dovr\`a ripetere il corso nell'anno accademico successivo.}
\end{frame}


\begin{frame}
\frametitle{Valutazione}
La valutazione include
\begin{itemize}
\item interazione informale docente/studente
\item frequenza al laboratorio (parzialmente)
\item utilizzo dei tool messi a disposizione
\item progetto (ha il maggiore impatto sulla valutazione)
\end{itemize}
\end{frame}


\section{Calendario}
\begin{frame}
\frametitle{Calendario}
\begin{itemize}


% \item  \textbf{15 aprile} Java, Eclipse, Junit, Debugging
% \item  \textbf{29 aprile} SVN, Maver, Sonar, Jenkins
% \item \textbf{13 maggio} DESIGN (UML)
% \item \textbf{20 maggio}  IMPLEMENTAZIONE 
% \item \textbf{27 maggio} CLIENT SERVER (Socket + RMI) 
% \item \textbf{3 giugno} INTERFACCIA GRAFICA (Swing)
% \item \textbf{10 giugno} CONCLUSIONE / VALUTAZIONE 
% \item \textbf{17 giugno} VALUTAZIONE

\item  \textbf{21 April} Tool introduction (Eclipse, Git, Maven, Sonar)
\item  \textbf{28 April} Design of the problem domain (UML, Design patterns)
\item \textbf{5 May} Architectural design (UML, Architectural patterns)
\item \textbf{12 May} Implementation and Testing (JUnit, TDD)
\item \textbf{19 May} Communication paradigms and design (Socket, RMI) 
\item \textbf{26 May} General discussion
\item \textbf{9 June} GUI design (Java Swing)
\item \textbf{16 June} General discussion
\item \textbf{23 June} Project evaluation

\end{itemize}
Come si pu\`o vedere il calendario \`e molto denso, quindi si deve dedicare molto tempo per passare il corso.
\end{frame}


%----------------------------------------------------------------------------------------

\end{document} 