\documentclass{article}

\usepackage[T1]{fontenc}
\usepackage{fancyhdr} % Required for custom headers
\usepackage{lastpage} % Required to determine the last page for the footer
\usepackage{extramarks} % Required for headers and footers
\usepackage[usenames,dvipsnames]{color} % Required for custom colors
\usepackage{graphicx} % Required to insert images
\usepackage{listings} % Required for insertion of code
\usepackage{courier} % Required for the courier font
\usepackage{lipsum} % Used for inserting dummy 'Lorem ipsum' text into the template
\usepackage{hyperref}
\usepackage{color}
\usepackage[normalem]{ulem}
\usepackage{url}
\usepackage [acronym] {glossaries}
\loadglsentries {acronyms}


% Margins
\topmargin=-0.45in
\evensidemargin=0in
\oddsidemargin=0in
\textwidth=6.5in
\textheight=9.0in
\headsep=0.25in

\linespread{1.1} % Line spacing


% Set up the header and footer
\pagestyle{fancy}
\lhead{\hmwkAuthorName} % Top left header
\chead{\hmwkClass\ (\hmwkClassInstructor): \hmwkTitle} % Top center head
\rhead{\firstxmark} % Top right header
\lfoot{\lastxmark} % Bottom left footer
\cfoot{} % Bottom center footer
\rfoot{Page\ \thepage\ of\ \protect\pageref{LastPage}} % Bottom right footer
\renewcommand\headrulewidth{0.4pt} % Size of the header rule
\renewcommand\footrulewidth{0.4pt} % Size of the footer rule

\setlength\parindent{0pt} % Removes all indentation from paragraphs

\usepackage{listings}
\usepackage{color}
\usepackage[applemac]{inputenc}
\usepackage{booktabs} % Allows the use of \toprule, \midrule and \bottomrule in tables
\usepackage{regexpatch}
\usepackage{todonotes}
\newcommand{\todosk}[1]{\todo[inline]{#1}} 


\definecolor{dkgreen}{rgb}{0,0.6,0}
\definecolor{gray}{rgb}{0.5,0.5,0.5}
\definecolor{mauve}{rgb}{0.58,0,0.82}
\usepackage{array,multirow,graphicx}

\lstset{frame=tb,
  language=Java,
  aboveskip=3mm,
  belowskip=3mm,
  showstringspaces=false,
  columns=flexible,
  basicstyle={\small\ttfamily},
  numbers=none,
  numberstyle=\tiny\color{gray},
  keywordstyle=\color{blue},
  commentstyle=\color{dkgreen},
  stringstyle=\color{mauve},
  breaklines=true,
  breakatwhitespace=true
  tabsize=3
}

%----------------------------------------------------------------------------------------
%	DOCUMENT STRUCTURE COMMANDS
%	Skip this unless you know what you're doing
%----------------------------------------------------------------------------------------

% Header and footer for when a page split occurs within a problem environment
\newcommand{\enterProblemHeader}[1]{
\nobreak\extramarks{#1}{#1 continued on next page\ldots}\nobreak
\nobreak\extramarks{#1 (continued)}{#1 continued on next page\ldots}\nobreak
}

% Header and footer for when a page split occurs between problem environments
\newcommand{\exitProblemHeader}[1]{
\nobreak\extramarks{#1 (continued)}{#1 continued on next page\ldots}\nobreak
\nobreak\extramarks{#1}{}\nobreak
}




%----------------------------------------------------------------------------------------
%	NAME AND CLASS SECTION
%----------------------------------------------------------------------------------------

\newcommand{\hmwkTitle}{Requisiti di Progetto} % Assignment title
\newcommand{\hmwkDueDate}{Aprile 12, 2016} % Due date
\newcommand{\hmwkClass}{Ingegneria del Software 1} % Course/class
\newcommand{\hmwkClassInstructor}{Sr\dj{}an Krsti\'c and Claudio Menghi} % Teacher/lecturer
%\newcommand{\hmwkClassInstructor}{} % Teacher/lecturer
\newcommand{\hmwkAuthorName}{} % Your name

%----------------------------------------------------------------------------------------
%	TITLE PAGE
%----------------------------------------------------------------------------------------

\title{
\vspace{2in}
\textmd{\textbf{\hmwkClass:\ \hmwkTitle}}\\
%\normalsize\vspace{0.1in}\small{Due\ on\ \hmwkDueDate}\\
\vspace{0.1in}\large{\textit{\hmwkClassInstructor}}
\vspace{3in}
}

\author{\textbf{\hmwkAuthorName}}
\date{} % Insert date here if you want it to appear below your name

%----------------------------------------------------------------------------------------

\begin{document}

\maketitle

%----------------------------------------------------------------------------------------
%	TABLE OF CONTENTS
%----------------------------------------------------------------------------------------

%\setcounter{tocdepth}{1} % Uncomment this line if you don't want subsections listed in the ToC

\newpage
\tableofcontents
\newpage



%----------------------------------------------------------------------------------------
\section{Introduzione}
Il progetto consiste nello sviluppo di una versione software del gioco
da tavolo \acrfull{cof}.

Il progetto finale dovr\` a includere:
\begin{itemize}
\item diagramma UML iniziale dell'applicazione (al alto livello)
\item diagrammi UML finali che mostrino come \`e stato progettato il software;
\item implementazione funzionante del gioco conforme alle regole del gioco e alle specifiche presenti in questo documento
\item codice sorgente dell'implementazione
\item codice sorgente dei test di unit\`a
\end{itemize}

\textbf{La data di consegna: 3, Giugno 2016, AoE}\\
\textbf{La data di valutazione: TBD.}
%\todosk{we need to define the dates for submission and evaluation...}

\section{Requisiti di Progetto}

%\todosk{Translate this:}
I requisiti del progetto consistono in due gruppi di requisiti:

  \begin{itemize}
  \item requisiti game-specific
  \item requisiti game-agnostic 
  \end{itemize}

\subsection{Game-specific requirements}
\subsubsection{Regole base}
Le regole del gioco sono descritte nel file \textbf{GameRules.pdf}
(Inglese) e \textbf{RegoleGioco.pdf} (Italiano),
caricati su BeeP. Il file definisce \textbf{regole base} di gioco.
In oltre, vengono aggiunte le seguenti \textbf{regole addizionali}. 

\subsubsection{Regole Addizionali}
\label{RegoleAddizionali}
Nel seguito ci riferiamo con il termine \textbf{regole complete} all'unione
delle regole base e regole addizionali.

\begin{itemize}
\item \textbf{Gioco configurabile}: 
il gioco deve essere progettato al fine di poter creare:
\begin{enumerate}
\item mappa configurabile
\begin{enumerate}
\item connessioni arbitrarie tra le citt\'a;
\item bonus arbitrari sulle carte permit e sulle citt\'a e sulla
  ``percorso della nobilt\'a'';
\item numero di bonus arbitrari sulle carte permit e sulle citt\'a e sulla
  ``percorso della nobilt\'a'';
\end{enumerate}
\item numero arbitrario di giocatori.
\item una lista di mappe (punti a-c per ogni mappa) deve essere
  caricata da un opportuno file di configurazione. Punto 2 deve essere
  passato come parametro alla creazione del gioco;
\end{enumerate}
\item \textbf{Market}: dopo che tutti i giocatori hanno giocato il
  loro turno (dopo ogni giro) viene lanciata la fase di Market.
 Partendo dal primo giocatore  fino all'ultimo, ogni giocatore pu\' o
 scegliere le carte (di permesso e politiche) e gli assistenti che intende vendere e il
 prezzo richiesto per ognuno di essi.
 Quando TUTTI i giocatori hanno scelto quali elementi vendere,
seguendo un ordine casuale i giocatori possono comprare gli elementi messi in vendita dagli altri giocatori.
\end{itemize}

\subsection{Game-agnostic requirements}

In questa sezione vengono presentati i requisiti tecnici
dell'applicazione.  Deve essere un sistema distributivo composto da
un lato gioco che pu\'o gestire molti giochi simultanei e multipli
lati giocatore che possono partecipare nel un solo gioco. Si raccomanda
l'utilizzo del pattern \textbf{MVC} (Model-View-Controller) per
progettare l'intero sistema.

\subsubsection{I Componenti del Gioco}

Il gioco consiste di ``Lato Gioco'' e ``Lato giocatore''. \\

\textbf{Lato Giocatore}\\

\begin{itemize}
\item Il lato giocatore deve poter essere istanziato pi\`u volte. 
\item Il lato giocatore deve essere sviluppato obbligatoriamente utilizzando JavaSE. 
\item L'interfaccia grafica deve essere mediante Swing o altre
  tecnologie (tuttavia in questo caso non sarete supportati dai responsabili), 
\item Il lato giocatore deve supportare RMI e Socket, in relazione al numero di studenti del gruppo, come specificato in tabella~\ref{TabellaDiValutazione}. 
\item Nel caso in cui sia richiesta sia l'implementazione RMI che
  quelle per mezzo di socket, l'applicazione, all'avvio, deve permettere
  all'utente di selezionare il metodo utilizzato per la
  comunicazione.
\item Nel caso in cui sia richiesta sia l'implementazione CLI che
  quelle per mezzo di GUI, l'applicazione, all'avvio, deve permettere
  all'utente di selezionare il metodo utilizzato per la
  visualizzazione.
\end{itemize}

\textbf{Lato Gioco}\\

Questo componente deve gestire le partite e deve poter essere
istanziato una sola volta. Deve permettere di:
\begin{itemize}
\item creare una nuova partita, inizializzarla, giocarla e concluderla secondo le regole del gioco.
\item Deve essere in grado di gestire pi\`u partite
  contemporaneamente.
\item Deve essere implementato secondo la logica JavaSE.
\item Nel caso in cui sia l'implementazione via socket che quelle via
  RMI sia richiesta, quando un giocatore si connette al server, il server deve comunicare utilizzando la connessione selezionata.
\end{itemize}

\subsubsection{Avvio della partita}
\label{avvio}
L'assunzione base \`e che ogni giocatore che voglia partecipare a
una partita conosca l'indirizzo (numerico o simbolico) del lato gioco. Quando un giocatore si connette, 
\begin{itemize}
\item se c\`e una partita in fase di avvio, il giocatore viene automaticamente aggiunto alla partita
\item  \textbf{Regole base:\footnote{se non sono richeste le regole
      addizionali}} la partita inizia non appena si raggiungono i 4
  giocatori. Quando 2 giocatori si connettono a una partita viene
  inizializzato un timeout di 20 secondi. Non appena il timeout scatta
  la partita inizia anche se non sono raggiunti i 4 giocatori.
\item  \textbf{Regole complete:\footnote{se sono richeste le regole
      addizionali}} quando 2 giocatori si connettono a una partita viene
  inizializzato un timeout di 20 secondi. Non appena il timeout scatta
  la partita inizia. Il numero di giocatori pu\'o essere arbitrario,
  anche maggiore di 4.
\item se non ci sono partite in fase di avvio, viene creata una nuova partita.
\end{itemize}
Si precisa che una nuova partita viene creata solamente quando un utente si connette e non ci sono partite in attesa, altrimenti l'utente entra automaticamente a far parte della partita in fase di avvio.

\subsubsection{Corso della partita}
\label{corso}
Il lato gioco consente ai vari giocatori di svolgere i propri turni
secondo le regole di gioco. E' necessario gestire il caso in cui i
lati giocatore si disconnettano.
\begin{itemize}
\item ogni giocatore ha un periodo di tempo fissato passato come parametro al momento della creazione del server per  eseguire le mosse 
\item se un giocatore va offline il lato gioco attende per il periodo di
  cui sopra,
  dopo il quale sospende il giocatore (nota il giocatore non esegue
  mosse ma viene comunque considerato nel conteggio dei punti etc.) 
\item tutti i giocatori vengono notificati della mancanza di un giocatore
\item il gioco continua, saltando i giri del giocatore sospeso
\item il lato giocatore pu� riconnettersi e continuare il gioco se si
  sceglie di implementare la funzionalit� ``Ripristino sessione giocatore''.
\end{itemize}

\subsubsection{Funzionalita Avanzate}
\label{FunzionalitaAvanzate}
Di seguito sono proposte alcune funzionalit\`a avanzate che concorrono alla valutazione.  Attenzione: il loro contributo non \`e  necessariamente additivo. Design e codice verranno comunque valutati in quanto tali e contribuiranno al giudizio globale.

\begin{itemize}
\item \textbf{Gestione degli utenti}. Realizzare un sistema di
  gestione degli utenti che supporti il login dei giocatori, conservi
  per ciascuno le statistiche di gioco (numero di vittorie, tempo di
  gioco, numero di sconfitte, etc...) e produca una classifica
  ordinata, prima per numero di vittorie e quindi per minimo tempo di
  gioco cumulativo. Inoltre, per ogni parita si desidera
  memorizzare, in un frame laterale, la sequenza di mosse effettuate dai
  vari giocatori in ordine di occorrenza.
\item  \textbf{Lato Gioco persistente}. Implementare la possibilit\`a di
  salvare lo stato del lato gioco su disco e di ricaricarlo all'avvio
  successivo nel modo che si ritiene pi\`u idoneo. 
\item  \textbf{Ripristino sessione giocatore}. Ai giocatori \`e permesso abbandonare
  temporaneamente la partita per via della perdita di connettivit\`a.
  Se il giocatore si riconnette prima della fine della partita pu\`o
  ricominciare a giocare, come se nulla fosse successo (ovviamente
  perde i turni durante i quali non ha giocato).
\item \textbf{Generazione automatica delle configurazioni di
    gioco}. Il lato gioco deve essere in grado di generare automaticamente
  una configurazione del gioco partendo da alcuni parametri dati.
\item \textbf{Lobby} questo requisito rimpiazza quelli specificati in
  sezione~\ref{avvio}. Una volta avviato il server i giocatori entrano una
  lobby (area gestione giochi) dove possono: 
\begin{enumerate}
\item vedere la lista dei partite create (la mappa, il numero massimo
  di giocatori, il numero di giocatori attuale, ...)
\item creare partite (definire mappa, timeout per le mosse...)
\item partecipare alle partite
\end{enumerate}
\item \textbf{Chat} i giocatori devono poter comunicare tra di loro
  mediante chat
\item \textbf{AI} deve essere possibile giocare contro il computer (un
  agente intelligente).
Questo \'e una funzionalit\'a molto complessa. 
Per informazioni addizionali contattare i responsabili.
\end{itemize}



\section{Valutazione}

\begin{table}[b]
  \centering
\newcolumntype{B}{>{\raggedright\arraybackslash}m{1cm}}

\begin{tabular}{ B c c c  }
\toprule
\setlength{\columnsep}{0.01cm}

Punteggio & \multicolumn{3}{c}{Numero studenti} \\

Max & 1 & 2 & 3 \\

\midrule 
18 &
Regole base&
Regole complete & 
Regole complete \\

22 &
CLI + GUI &
RMI + CLI & 
Socket + CLI \\

24 & 
RMI + CLI &
Socket + CLI  &
RMI + Socket + CLI \\

26 &
Socket + CLI &
RMI + Socket + CLI &
RMI + Socket + GUI\\

28 &
RMI + Socket + CLI &
RMI + Socket + GUI &
All Tech\\

30L &
RMI + Socket + GUI &
All Tech &   
1 Funzionalit\`a Avanzata\\

\bottomrule
\end{tabular}
\caption{Tabella di valutazione}
\label{TabellaDiValutazione}
\end{table}


I gruppi di studenti devono implementare  le specifiche 
descritte in tabella~\ref{TabellaDiValutazione} in relazione al
punteggio desiderato. 
Nota: in tabella sono rappresentati i
\textbf{massimi} punteggi ottenibili in relazione alle feature
implementate. Se un gruppo decide di implementare solo le regole del
gioco (la prima linea) l'applicazione \`e centralizzata e utilizza un
CLI. Altrimenti, l'applicazione deve comprendere dei diversi
componenti distribuiti che comunicano in rete.

La seguente \`e la spiegazione delle abbreviazioni nella
tabella.

\begin{itemize}
\item \textbf{Command Line Interface (CLI)}: \`e implementato come un interfaccia testuale e i vari giocatori che si alternano nei turni utilizzeranno la tastiera.
\item \textbf{Graphical User Interface (GUI)}: consiste in un interfaccia grafica 
\item \textbf{RMI:} la comunicazione avviene mediante ``Remote method invocation''
\item \textbf{Socket}: la comunicazione avviene mediante messaggi scambiati
  attraverso socket. Lo studente deve autonomamente definire e
  implementare un protocollo di comunicazione tra i componenti distribuiti.
\end{itemize}

Per esempio, per un gruppo di 3 studenti e un punteggio
desiderato di al massimo 24 punti \`e necessario implementare le
regole complete (base e addizionali) del gioco,
comunicazione con RMI, Socket e il Command Line Interface (CLI).
All Tech significa che il gruppo deve implementare due modi di
comunicazione (RMI e Socket) e due tipi di interfaccia (CLI e GUI).
Abbiamo ottimizzato il carico di lavoro per il gruppi di 3
persone e vi consigliamo quindi di costituire gruppi di questa dimensione.\\
\textbf{Gli studenti di telecomunicazioni, possono fare i gruppi di un massimo
di 4 persone ed i requisiti applicati corrisponderanno il gruppo con
una persona in meno.}\\
\textbf{Nota: Non � possibile fare i gruppi di 1 persona, la
  tabella~\ref{TabellaDiValutazione} contiene le informazioni solo
  come riferimento agli studenti di telecomunicazioni.}\\

Saranno oggetto di valutazione
\begin{itemize}
\item la qualit\` a della \textbf{progettazione} con particolare riferimento a
  un uso appropriato di interfacce, ereditariet\`a, composizione tra
  classi, uso dei design pattern;
\item la qualit\` a della progettazione dell'\textbf{architettura
  dell'applicazione}; divisione della responsabilit\'a; progettazione
  della comunicazione;
\item la stabilit\` a dell'implementazione e la \textbf{conformit\` a alle
  specifiche} date; 
\item la \textbf{qualit\` a} e la \textbf{leggibilit\` a} del codice scritto, con
  particolare riferimento a nomi di variabili/metodi/classi/package,
  all'inserimento di commenti e documentazione JavaDoc
  (preferibilmente in inglese), la mancanza di codice ripetuto e
  metodi di eccessiva lunghezza;
\item la \textbf{qualit\` a} e la \textbf{copertura} dei casi di test: il nome e i
  commenti di ogni test dovranno chiaramente specificare le funzionalit\`
  a testate e i componenti coinvolti;
 \item il corretto utilizzo degli strumenti (Eclipse, Git, Maven, ...);
\item il livello di autonomia e impegno nello svolgimento del progetto.
\end{itemize}

Le funzionalit\`a avanzate possono essere implementate da tutti i
gruppi e comportano dei punteggi aggiuntivi. Ovviamente per
implementare queste funzionalit\`a \`e necessario che TUTTO il resto
del progetto sia implementato in maniera COMPLETA e ADEGUATA
(copertura con test, ben commentata etc).

\end{document}
