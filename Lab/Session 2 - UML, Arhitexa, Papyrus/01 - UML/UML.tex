\documentclass{article}

\usepackage{fancyhdr} % Required for custom headers
\usepackage{lastpage} % Required to determine the last page for the footer
\usepackage{extramarks} % Required for headers and footers
\usepackage[usenames,dvipsnames]{color} % Required for custom colors
\usepackage{graphicx} % Required to insert images
\usepackage{listings} % Required for insertion of code
\usepackage{courier} % Required for the courier font
\usepackage{lipsum} % Used for inserting dummy 'Lorem ipsum' text into the template
\usepackage{hyperref}
% Margins
\topmargin=-0.45in
\evensidemargin=0in
\oddsidemargin=0in
\textwidth=6.5in
\textheight=9.0in
\headsep=0.25in

\linespread{1.1} % Line spacing

% Set up the header and footer
\pagestyle{fancy}
\lhead{\hmwkAuthorName} % Top left header
\chead{\hmwkClass\ (\hmwkClassInstructor\ \hmwkClassTime): \hmwkTitle} % Top center head
\rhead{\firstxmark} % Top right header
\lfoot{\lastxmark} % Bottom left footer
\cfoot{} % Bottom center footer
\rfoot{Page\ \thepage\ of\ \protect\pageref{LastPage}} % Bottom right footer
\renewcommand\headrulewidth{0.4pt} % Size of the header rule
\renewcommand\footrulewidth{0.4pt} % Size of the footer rule

\setlength\parindent{0pt} % Removes all indentation from paragraphs

\usepackage{listings}
\usepackage{color}

\definecolor{dkgreen}{rgb}{0,0.6,0}
\definecolor{gray}{rgb}{0.5,0.5,0.5}
\definecolor{mauve}{rgb}{0.58,0,0.82}

\lstset{frame=tb,
  language=Java,
  aboveskip=3mm,
  belowskip=3mm,
  showstringspaces=false,
  columns=flexible,
  basicstyle={\small\ttfamily},
  numbers=none,
  numberstyle=\tiny\color{gray},
  keywordstyle=\color{blue},
  commentstyle=\color{dkgreen},
  stringstyle=\color{mauve},
  breaklines=true,
  breakatwhitespace=true
  tabsize=3
}

%------------------------------------------------------------------------------------------
%	DOCUMENT STRUCTURE COMMANDS
%	Skip this unless you know what you're doing
%-------------------------------------------------------------------------------------------

% Header and footer for when a page split occurs within a problem environment
\newcommand{\enterProblemHeader}[1]{
\nobreak\extramarks{#1}{#1 continued on next page\ldots}\nobreak
\nobreak\extramarks{#1 (continued)}{#1 continued on next page\ldots}\nobreak
}

% Header and footer for when a page split occurs between problem environments
\newcommand{\exitProblemHeader}[1]{
\nobreak\extramarks{#1 (continued)}{#1 continued on next page\ldots}\nobreak
\nobreak\extramarks{#1}{}\nobreak
}



%-------------------------------------------------------------------------------------------
%	NAME AND CLASS SECTION
%-------------------------------------------------------------------------------------------

\newcommand{\hmwkTitle}{UML and Sonar} % Assignment title
\newcommand{\hmwkDueDate}{Martedi,\ Aprile 29,\ 2014} % Due date
\newcommand{\hmwkClass}{Ingegneria del Software 1} % Course/class
\newcommand{\hmwkClassTime}{} % Class/lecture time
\newcommand{\hmwkClassInstructor}{} % Teacher/lecturer
\newcommand{\hmwkAuthorName}{} % Your name

%-------------------------------------------------------------------------------------------
%	TITLE PAGE
%-------------------------------------------------------------------------------------------

\title{
\vspace{2in}
\textmd{\textbf{\hmwkClass:\ \hmwkTitle}}\\
\normalsize\vspace{0.1in}\small{Due\ on\ \hmwkDueDate}\\
\vspace{0.1in}\large{\textit{\hmwkClassInstructor\ \hmwkClassTime}}
\vspace{3in}
}

\author{\textbf{\hmwkAuthorName}}
\date{} % Insert date here if you want it to appear below your name

%-------------------------------------------------------------------------------------------

\begin{document}

\maketitle

%-------------------------------------------------------------------------------------------
%	TABLE OF CONTENTS
%-------------------------------------------------------------------------------------------

%\setcounter{tocdepth}{1} % Uncomment this line if you don't want subsections listed in the ToC

\newpage
\tableofcontents
\newpage



%-------------------------------------------------------------------------------------------
\section{Introduction}
L'Unified Modeling Language (UML) \`e un linguaggio di modellazione di uso generale e fornisce uno strumento standard per progettare i nostri sistemi. 

I diagrammi UML forniscono due diverse informazioni sui nostri modelli:

\begin{enumerate}
\item Informazioni Statiche (o strutturali): forniscono una descrizione \emph{statica} del sistema utilizzando classi, attributi, operazioni e relazioni. I class diagram sono i diagrammi statici comunemente utilizzati.
\item Informazioni Dinamiche  (o comportamentali): descrivono il comportamento dinamico dell'applicazione mostrando le relazioni di collaborazione tra i vari oggetti e come lo stato interno degli oggetti cambia nel tempo. I sequence diagrams, activity diagrams e gli state machine diagrams sono diagrammi UML dinamici.
\end{enumerate}


\subsection{Class diagram}
I class Diagram descrivono la struttura del sistema, per mezzo di classi, attributi, operazioni (o metodi) e relazioni tra gli oggetti.\\

\subsubsection{Classe}
\begin{itemize}
\item  Le \textit{classi} sono illustrate per mezzo di rettangoli divisi verticalmente. Nella varie partizioni sono indicati:,
\begin{itemize}
\item il \emph{nome} della classe
\item gli \emph{attributi} della classe: sono le informazioni memorizzate nell'oggetto
\item i \emph{metodi}: sono operazioni che un oggetto pu\`o effettuare
\end{itemize}
\item Gli attributi e i metodi sono chiamati \emph{membri} della classe 
\item Ogni attributo e ogni metodo \`e associato a una visibilit\`a ovvero
\begin{itemize}
\item  $+$ Public,
\item  $-$ Private,
\item $\#$ Protected
\item  \~{} (Friendly).
\end{itemize}
\item possiamo specificare due tipi di cope per i membri della classe: instance and static.
\end{itemize}


\subsubsection{Associazione}
Rappresenta le relazioni statiche tra le classi. 
\begin{itemize}
\item L'associazione binaria (con 2 capi) \`e normalmente rappresentata per mezzo di una linea.
\item una associazione \`e solitamente labellata con un nome
\item i capi di una associazione sono solitamente decorati con indicatori di appartenenza, molteplicit\`a etc
\end{itemize}

\subsubsection{Aggregazione}
L'aggregazione \`e un particolare tipo di associazione utilizzata per descrivere una relazione di tipo "has a". In altre parole l' aggregazione \`e un tipo particolare di associazione che rappresenta la relazione tra una parte e il suo contenitore. L'aggregazione \`e utilizzata quando una classe \`e un contenitore di altre oggetti. Tuttavia contenitore e contenute non sono in una relazione ``forte" ovvero se il contenitore \`e distrutto le altri oggetti (contenuti) continuano ad esistere.

\subsubsection{Composizione}
\`E un particolare tipo di aggregazione che viene utilizzato quando \`e presente una ``forte" relazione tra contenitore e contenuto. Se il contenitore viene distrutto ogni istanza degli oggetti contenuti viene distrutta di conseguenza.

\subsubsection{Generalizzazione}
\`E una relazione di tipo ``\`e un". Indica che una delle due classi (la sottoclasse) \`e considerata come una specializzazione dell'altra (la super classe). Questo significa che ogni istanza della sottoclasse \`e anche un istanza della super classe e che tutti i metodi pubblici e protetti della super classe sono ereditati dalla sotto classe.

\subsubsection{Realizzazione}
\`E una relazione tra due modelli di elementi. Di solito una classe e un interfaccia, dove un elemento del modello (la classe) implementa il comportamento che altro elemento (l'interfaccia) specifica. Diverse classi possono ereditare il comportamento da una singola interfaccia. 

\subsubsection{Dipendenza}
Indica che una delle classi dipende da un'altra classe perch\`e la utilizza in un certo istante di tempo. Una classe dipende da un altra se la utilizza come parametro o variabile locale in un metodo.

\subsection{Sequence diagram}
\ldots

\section{Example problem}

Si crei un opportuno class diagram per rappresentare una versione semplificate del gioco degli scacchi. Sapendo che:

\begin{enumerate}
\item Si ha una griglia 8x8 denominata Scacchiera 

\item Ogni elemento della griglia si chiama Casella 

\item In ogni casella ci pu\`o essere al pi\`u un Pezzo 

\item I pezzi possono essere: Torre, Cavallo, Alfiere, Regina, Re, ognuno con differenti 
capacit\`a di movimento (si ignori il Pedone per il momento). 

\item La scacchiera viene creata con dei pezzi dentro le caselle posizionati 
opportunamente. 

\item Una casella pu\`o essere vuota o avere un pezzo al suo interno. 

\item Un pezzo pu\`o appartenere al giocatore Bianco o al giocatore Nero e pu\`o spostarsi 
da una casella ad un’altra secondo determinate regole dipendenti dal pezzo. 

\item Un pezzo pu\`o muoversi solo verso una casella vuota od una casella occupata da 
un pezzo avversario. In questo secondo caso il pezzo avversario viene rimosso. I movimenti consentono di spostarsi di al pi\`u una casella in orizzontale, verticale o obliquo.
\end{enumerate}

Si descriva il sequence diagram che consente di effettuare un movimento.
Si implementi un primo prototipo dell'applicazione.

\subsection{UML diagrams}

\includegraphics[scale=0.6]{uml-fig.pdf}\\
\vspace{1cm}
\includegraphics[scale=0.55]{seq-fig.pdf}
\newpage

\subsection{Mapping to Java}

\subsubsection{Coordinate class}
\begin{lstlisting}
 package com.polimi.scacchi;

 public class Coordinate {
 
 	/**
 	 * the x coordiante
 	 */
 	private int x;
 	/**
 	 * the y coordiante
 	 */
 	private int y;
 	 	
 	/**
 	 * Constructor for the Coordinate class
 	 * @param x - x coordinate
 	 * @param y - y coordinate
 	 */
 	public Coordinate(int x, int y){
 		this.x=x;
 		this.y=y;
 	}
 	/**
 	 * @return the x
 	 */
 	public int getX() {
 		return x;
 	}
 	/**
 	 * @param x the x to set
 	 */
 	public void setX(int x) {
 		this.x = x;
 	}
 	/**
 	 * @return the y
 	 */
 	public int getY() {
 		return y;
 	}
 	/**
 	 * @param y the y to set
 	 */
 	public void setY(int y) {
 		this.y = y;
 	}	
 }
 
\end{lstlisting}

\newpage

\subsubsection{Casella class}
\begin{lstlisting}
package com.polimi.scacchi;

public class Casella extends Coordinate {	
	
	/**
	 * Currently placed piece, null if empty
	 */
	private Pezzo pezzo;
	
	/**
	 * Constructor for the Casella class
	 * @param x - x coordinate
	 * @param y - y coordinate
	 */
	public Casella(int x, int y){
		super(x,y);
	}
	
	/**
	 * @return the pezzo
	 */
	public Pezzo getPezzo() {
		if(pezzo!=null)	
			return pezzo.clone();
		return null;
	}

	/**
	 * @param pezzo the pezzo to set and update the pezzo reference to casella
	 */
	public void setPezzo(Pezzo pezzo) {
		if(pezzo==null){
			throw new IllegalArgumentException("The pezzo to be added in the casella cannot be null");
		}
		this.pezzo = pezzo.clone();
	}	
	
	/**
	 * removes the pezzo from the casella (if any)
	 */
	public void unsetPezzo() {
		this.pezzo = null;
	}	
	/**
	 * Checks if the field is empty
	 * @return True if the field is empty, False otherwise
	 */
	public boolean eVuoto(){
		return this.pezzo==null;
	}


	/**
	 * @return the string that describes the Casella object
	 */
	@Override
	public String toString() {
		if(this.eVuoto())
		{
			return " ";
		}
		else
		{
			return this.pezzo.toString();
		}
	}
}
\end{lstlisting}

\subsubsection{Pezzo class}
\begin{lstlisting}
package com.polimi.scacchi;

public abstract class Pezzo {	
	
	/**
	 * the color of the piece
	 */
	protected Colore colore;	
	/**
	 * The current field where the piece is placed, null if nowhere
	 */
	protected Casella casella;
	
	public Pezzo(Colore colore,Casella casella){
		if(colore==null){
			throw new IllegalArgumentException("Il colore del pezzo non deve essere nullo");
		}
		if(casella==null){
			throw new IllegalArgumentException("La casella iniziale del pezzo non deve essere nulla");
		}
		this.colore =colore;
		this.casella=casella;
	}	
	/**
	 * Abstract method  that checks if it is allowed to move a certain piece
	 * @param casellaFinale - the target field where we want to move the piece
	 * @return True if it is allowed to move the piece according to the rules 
	 * of the game, false otherwise
	 */
	public abstract boolean mossaValida(Casella casellaFinale);
	/**
	 * @return the casella
	 */
	public Casella getCasella() {
		return casella;
	}
	/**
	 * @return the colore
	 */
	public Colore getColore() {
		return colore;
	}	
	@Override
	public abstract Pezzo clone();
}
\end{lstlisting}

\subsubsection{Re class}
\begin{lstlisting}
package com.polimi.scacchi;

public class Re extends Pezzo {

	public Re(Colore colore, Casella casella) {
	
		super(colore,casella);
	}

	/* (non-Javadoc)
	 * @see com.polimi.scacchiera.Scacchiera.Pezzo#mossaValida(com.polimi.scacchiera.Scacchiera.Casella)
	 */
	@Override
	public boolean mossaValida(Casella casellaFinale) {
		if(casellaFinale==null){
			throw new IllegalArgumentException("La casella finale della mossa non deve essere nulla");
		}
		
		//cannot go to the field already occupied by a piece with the same color
		if(!casellaFinale.eVuoto()) {
			if(casellaFinale.getPezzo().getColore()==this.getColore()) {
				return false;
			}
		}
		return Math.abs(this.getCasella().getX()-casellaFinale.getX())<=1 &&
				Math.abs(this.getCasella().getY()-casellaFinale.getY())<=1;
	}
	@Override
	public Re clone(){
		return new Re(this.colore, this.casella);
	}

	/**
	 * Prints the piece
	 */
	@Override
	public String toString() {
		return "K";		
	}
}
\end{lstlisting}

\subsubsection{Scacchiera class}
\begin{lstlisting}
package com.polimi.scacchi;

public class Scacchiera {
	/**
	 * 8x8 Matrix of caselle
	 */
	private Casella[][] caselle;

	private static final int SIZE = 8;

	/**
	 * Constructor for Scacchiera, initializes all the fields.
	 */
	public Scacchiera() {
		//initialize fields
		caselle = new Casella[SIZE][SIZE];
		for(int i=0;i<SIZE;i++) {
			for(int j=0;j<SIZE;j++) {
				caselle[i][j]= new Casella(i,j);
			}
		}
				
		//set pieces
		caselle[0][4].setPezzo(new Re(Colore.BLACK,caselle[0][4]));
		//etc...
		
	}
	
	/** 
	 * @see java.lang.Object#toString()
	 */
	@Override
	public String toString(){
		String ret="";
        
    	ret+="---------------------------------\n";

        for(int i=0;i<SIZE;i++)
		{
			ret+="| ";
			for(int j=0;j<SIZE;j++)
			{
				ret+=caselle[i][j].toString();
				ret+=" | ";
				
			}
			ret+="\n";
			ret+="---------------------------------\n";
		}
		
		ret+="\n ";
		return ret;
	}
	
	/**
	 * Moves a piece from one field to another
	 * @param ci - coordinates of the inital field
	 * @param cf - coordinates of the final field
	 */
	public void muovi(Coordinate ci, Coordinate cf){
		if(ci==null){
			throw new IllegalArgumentException("Le coordinate iniziali del pezzo non possono essere nulle");
		}
		if(cf==null){
			throw new IllegalArgumentException("Le coordinate finali del pezzo non possono essere nulle");
		}
		if(ci.getX() >= 0  && cf.getX() >= 0 && 
		   ci.getX() < SIZE && cf.getX() < SIZE &&
		   ci.getY() >= 0  && cf.getY() >= 0 && 
		   ci.getY() < SIZE && cf.getY() < SIZE)
		{
			Casella initalCasella = caselle[ci.getX()][ci.getY()];
			Casella finalCasella = caselle[cf.getX()][cf.getY()];
						
			if(!initalCasella.eVuoto()) {
				Pezzo pezzo = initalCasella.getPezzo();
				if(pezzo.mossaValida(finalCasella)) {
					initalCasella.unsetPezzo();;
					finalCasella.setPezzo(pezzo);
				}
			}
		}
	}	
	
}
\end{lstlisting}

\subsubsection{Main class}
\begin{lstlisting}
package com.polimi.scacchi;

public class App 
{
    public static void main( String[] args )
    {
    	
        Scacchiera s = new Scacchiera();
        System.out.println(s.toString());
        s.muovi(new Coordinate(0, 4), new Coordinate(1, 5));
        System.out.println(s.toString());
        s.muovi(new Coordinate(1, 5), new Coordinate(5, 5));
        System.out.println(s.toString());
    }
\end{lstlisting}

 \subsubsection{Output}
 \begin{lstlisting}
-------------------------
|   |   |   |   | K|    |   |   | 
-------------------------
|   |   |   |   |   |   |   |   | 
-------------------------
|   |   |   |   |   |   |   |   | 
-------------------------
|   |   |   |   |   |   |   |   | 
-------------------------
|   |   |   |   |   |   |   |   | 
-------------------------
|   |   |   |   |   |   |   |   | 
-------------------------
|   |   |   |   |   |   |   |   | 
-------------------------
|   |   |   |   |   |   |   |   | 
-------------------------
 
 
-------------------------
|   |   |   |   |   |   |   |   | 
-------------------------
|   |   |   |   |   | K|    |   | 
-------------------------
|   |   |   |   |   |   |   |   | 
-------------------------
|   |   |   |   |   |   |   |   | 
-------------------------
|   |   |   |   |   |   |   |   | 
-------------------------
|   |   |   |   |   |   |   |   | 
-------------------------
|   |   |   |   |   |   |   |   | 
-------------------------
|   |   |   |   |   |   |   |   | 
-------------------------
 
 
-------------------------
|   |   |   |   |   |   |   |   | 
-------------------------
|   |   |   |   |   | K|    |   | 
-------------------------
|   |   |   |   |   |   |   |   | 
-------------------------
|   |   |   |   |   |   |   |   | 
-------------------------
|   |   |   |   |   |   |   |   | 
-------------------------
|   |   |   |   |   |   |   |   | 
-------------------------
|   |   |   |   |   |   |   |   | 
-------------------------
|   |   |   |   |   |   |   |   | 
-------------------------
\end{lstlisting}



\end{document}

