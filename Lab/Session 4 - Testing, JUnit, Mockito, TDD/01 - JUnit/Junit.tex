\documentclass{article}

\usepackage{fancyhdr} % Required for custom headers
\usepackage{lastpage} % Required to determine the last page for the footer
\usepackage{extramarks} % Required for headers and footers
\usepackage[usenames,dvipsnames]{color} % Required for custom colors
\usepackage{graphicx} % Required to insert images
\usepackage{listings} % Required for insertion of code
\usepackage{courier} % Required for the courier font
\usepackage{lipsum} % Used for inserting dummy 'Lorem ipsum' text into the template
\usepackage{hyperref}
% Margins
\topmargin=-0.45in
\evensidemargin=0in
\oddsidemargin=0in
\textwidth=6.5in
\textheight=9.0in
\headsep=0.25in

\linespread{1.1} % Line spacing

% Set up the header and footer
\pagestyle{fancy}
\lhead{\hmwkAuthorName} % Top left header
\chead{\hmwkClass\ (\hmwkClassInstructor\ \hmwkClassTime): \hmwkTitle} % Top center head
\rhead{\firstxmark} % Top right header
\lfoot{\lastxmark} % Bottom left footer
\cfoot{} % Bottom center footer
\rfoot{Page\ \thepage\ of\ \protect\pageref{LastPage}} % Bottom right footer
\renewcommand\headrulewidth{0.4pt} % Size of the header rule
\renewcommand\footrulewidth{0.4pt} % Size of the footer rule

\setlength\parindent{0pt} % Removes all indentation from paragraphs

\usepackage{listings}
\usepackage{color}

\definecolor{dkgreen}{rgb}{0,0.6,0}
\definecolor{gray}{rgb}{0.5,0.5,0.5}
\definecolor{mauve}{rgb}{0.58,0,0.82}

\lstset{frame=tb,
  language=Java,
  aboveskip=3mm,
  belowskip=3mm,
  showstringspaces=false,
  columns=flexible,
  basicstyle={\small\ttfamily},
  numbers=none,
  numberstyle=\tiny\color{gray},
  keywordstyle=\color{blue},
  commentstyle=\color{dkgreen},
  stringstyle=\color{mauve},
  breaklines=true,
  breakatwhitespace=true
  tabsize=3
}

%----------------------------------------------------------------------------------------
%	DOCUMENT STRUCTURE COMMANDS
%	Skip this unless you know what you're doing
%----------------------------------------------------------------------------------------

% Header and footer for when a page split occurs within a problem environment
\newcommand{\enterProblemHeader}[1]{
\nobreak\extramarks{#1}{#1 continued on next page\ldots}\nobreak
\nobreak\extramarks{#1 (continued)}{#1 continued on next page\ldots}\nobreak
}

% Header and footer for when a page split occurs between problem environments
\newcommand{\exitProblemHeader}[1]{
\nobreak\extramarks{#1 (continued)}{#1 continued on next page\ldots}\nobreak
\nobreak\extramarks{#1}{}\nobreak
}




%----------------------------------------------------------------------------------------
%	NAME AND CLASS SECTION
%----------------------------------------------------------------------------------------

\newcommand{\hmwkTitle}{Junit} % Assignment title
\newcommand{\hmwkDueDate}{Martedi,\ Aprile 15,\ 2014} % Due date
\newcommand{\hmwkClass}{Ingegneria del Software 1} % Course/class
\newcommand{\hmwkClassTime}{} % Class/lecture time
\newcommand{\hmwkClassInstructor}{} % Teacher/lecturer
\newcommand{\hmwkAuthorName}{} % Your name

%----------------------------------------------------------------------------------------
%	TITLE PAGE
%----------------------------------------------------------------------------------------

\title{
\vspace{2in}
\textmd{\textbf{\hmwkClass:\ \hmwkTitle}}\\
\normalsize\vspace{0.1in}\small{Due\ on\ \hmwkDueDate}\\
\vspace{0.1in}\large{\textit{\hmwkClassInstructor\ \hmwkClassTime}}
\vspace{3in}
}

\author{\textbf{\hmwkAuthorName}}
\date{} % Insert date here if you want it to appear below your name

%----------------------------------------------------------------------------------------

\begin{document}

\maketitle

%----------------------------------------------------------------------------------------
%	TABLE OF CONTENTS
%----------------------------------------------------------------------------------------

%\setcounter{tocdepth}{1} % Uncomment this line if you don't want subsections listed in the ToC

\newpage
\tableofcontents
\newpage



%----------------------------------------------------------------------------------------
\section{Preliminaries}
\subsection{Test di unit\`a}
Consentono di testare una singola entit\`a (classe o metodo). 
\subsection{JUnit}
JUnit \`e un framework per il test di unita per il linguaggio Java e consente di seguire un paradigma di programmazione test-driven.

\subsection{Unit test case}
\begin{itemize}
\item porzione di codice che assicura che un altra parte di codice (in genere un metodo) funziona come aspettato.
\item un test formale ben scritto \`e caratterizzato da un input noto e un output atteso che \`e stabilito prima dell'esecuzione del test.
\item in genere per ogni requisito (funzionalit\`a implementate) devono esserci \textbf{almeno} due test. Uno positivo e l'altro negativo
\item junit consente di utilizzare le annotazioni per identificare i metodi di test.
\item le asserzioni servono a confrontare i risultati ottenuti con quelli attesi
\end{itemize}

\subsection{Caratteristiche di JUnit}
JUnit fornisce le seguenti features:
\begin{itemize}
\item \emph{Fixture}: \`e possibile settare uno stato predefinito degli oggetti prima di eseguire un test. L'obbiettivo \`e di assicurare che l'ambiente nel quale i test sono eseguiti \`e noto e fissato in modo che i test siano ripetibili (@Before (setting), @After (pulizia)). 
\item \emph{Test runner}: \`e utilizzato per eseguire i test (trasparente all'utente)
\end{itemize}




\section{Problem}
Vogliamo creare un metodo che converte dollari in franchi svizzeri (CHF).

\textbf{Requisito base}.
Dobbiamo essere capaci di moltiplicare un numero per un altro. 
Esempio se il cambio Dollari-CHF 2 :1, 5 Dollari vengono convertiti in 10 franchi.

\begin{itemize}
\item Visto che siamo test driven partiamo dai test e non dagli oggetti
\item File $>$ new $>$ Junit Test Case (Test)
\item scegliamo il nome del test TestMultiplication. Nota il nome del test deve iniziare con la parola \textbf{Test} e deve essere chiara la funzionalit\`a testata
\end{itemize}
\section{Step 1}
\subsection{Step 1.1 Creating the test}

\begin{lstlisting}
package org.ingsoft.junit;

import static org.junit.Assert.*;

import org.ingsoft.junit.tdd.Dollar;
import org.junit.Test;

public class TestMultiplication {
	@Test
	public void test() {
		Dollar five=new Dollar(5);
		five.times(2);
		assertEquals(10, five.amount);
	}
}
\end{lstlisting}

Problemi: public fields, side effects, integer for monetary amounts.

Vogliamo compilare il prima possibile. Cosa dobbiamo fare?
\begin{itemize}
\item Creare la classe dollaro
\item Creare un costruttore
\item Creare il metodo times
\item Creare l'attributo pubblico  amount
\end{itemize}

\subsection {Step 1.2 Making the test running}
\begin{lstlisting}
package org.ingsoft.junit.tdd;
public class Dollar {
	
	public int amount=0;
	
	public Dollar(int value){
		
	}
	public void times(int value){
		
	}
}
\end{lstlisting}
Compila perfetto. Problema il test non \`e passato.

\begin{lstlisting}
package org.ingsoft.junit.tdd;
public class Dollar {
	
	public int amount=10;
	
	public Dollar(int value){
		
	}
	public void times(int value){
		
	}
}
\end{lstlisting}
Compila perfetto il test \`e passato.

\subsection {Step 1.3 Refactor}
C'\`e una dipendenza tra il codice e il test. Ovvero abbiamo del codice duplicato (il numero 10). Come possiamo eliminarlo?
\begin{itemize}
\item Passo  1: nel metodo  times scriviamo this.amount = 5*2 
\item Passo 2: da dove possiamo ottenere il 5?  
\begin{itemize}
\item Mettiamo nel costruttore this.amount=value;
\item Mettiamo nella metodo times  this.amount = this.amount*2 
\end{itemize}
\item Passo 3: da dove possiamo ottenere il 2?
\begin{itemize}
\item mettiamo nel metodo times  this.amount = this.amount* value
\end{itemize}
\item Passo 4: rimuoviamo il codice replicato
\begin{itemize}
\item  replace amount = amount* value with amount *= value
\end{itemize}
\end{itemize}

\section{Step 2}
Problema dopo aver eseguito un operazione l'oggetto dollaro cambia.
\subsection{Step 2.1 Creating the test}
\begin{lstlisting}
package org.ingsoft.junit;

import static org.junit.Assert.*;

import org.ingsoft.junit.tdd.Dollar;
import org.junit.Test;

public class TestMultiplication {
	@Test
	public void test() {
		Dollar five=new Dollar(5);
		five.times(2);
		assertEquals(10, five.amount);
		five.times(3);
		assertEquals(15, five.amount);
	}
}
\end{lstlisting}


\subsection{Step 2.2 Making the test running}
Cambiare l'interfaccia della classe dollaro per fagli ritornare un nuovo oggetto.
\begin{lstlisting}
@Test
	public void test() {
		Dollar five=new Dollar(5);
		Dollar product=five.times(2);
		assertEquals(10, product.amount);
		product=five.times(3);
		assertEquals(15, product.amount);
	}
\end{lstlisting}

\begin{lstlisting}
package org.ingsoft.junit.tdd;

public class Dollar {
	
	public int amount;
	
	public Dollar(int value){
		this.amount=value;
	}
	public Dollar times(int value){
		this.amount *= value;
		return null;
	}
}
\end{lstlisting}


\begin{lstlisting}
public Dollar times(int value){
		return new Dollar(this.amount *= value);
	}
\end{lstlisting}



\subsection{Step 2.3 Refactor}
Posso mettere l'attributo amount come privato.


\section{Best Practices}
\begin{itemize}
\item non utilizzare il costruttore del test case per settare il test case.
\item non assumere di conoscere l'ordine nel quale i test case vengono eseguiti (anche all'interno del singolo junit).
\item non scrivere test cases che hanno side effects 
\item scrivere i test che leggono dati da locazioni del file sistem utilizzando path relativi
\item memorizzare i dati che sono necessari per i test assieme ai test stessi
\item assicurati che i nomi dei test sono time-indipendent
\item scegliere i nomi dei  test nel modo corretto
\begin{itemize}
\item il nome del test deve iniziare con la parola Test (esempio \emph{TestClassUnderTest})
\item il nome dei metodi nel test case devono descrivere cosa viene testato (esempio \emph{testLoggingEmptyMessage()})
\end{itemize}
\item utilizza i metodi assert e fail di junit nel modo corretto per mantenere il codice pulito. Considera per esempio la leggibilit\' a di assertEquals ("The number of credentials should be 3", 3, creds); rispetto a assert (creds == 3); 
\item commentare i test con la javadoc
\item mantieni i test piccoli e veloci
\end{itemize}

\section{Testing private methods}
Per testare dei metodi privati ci sono 2 strade possibili:
\begin{itemize}
\item testarlo attraverso un altro metodo pubblico che lo utilizza
%\item utilizzare la \emph{reflection} (nel caso sia proprio necessario).
\end{itemize}
%in pratica conviene testare solo la parte di codice visibile da un ipotetico “client” completamente esterno al codice, e quindi trascurare i metodi protetti

\section{Che cosa testare?}
Non \`e una regola assoluta, ma il principio generale \`e: \emph{quanto pi\`u una parte di codice \`e visibile dall'esterno, tanto pi\`u deve essere testata}




\end{document}



