% !TeX spellcheck = it_IT
\documentclass{article}
\usepackage[utf8]{inputenc}
\inputencoding{utf8}
\usepackage[T1]{fontenc}
\usepackage{fancyhdr} % Required for custom headers
\usepackage{lastpage} % Required to determine the last page for the footer
\usepackage{extramarks} % Required for headers and footers
\usepackage[usenames,dvipsnames]{color} % Required for custom colors
\usepackage{graphicx} % Required to insert images
\usepackage{listings} % Required for insertion of code
\usepackage{courier} % Required for the courier font
\usepackage{lipsum} % Used for inserting dummy 'Lorem ipsum' text into the template
\usepackage{hyperref}
\usepackage{multirow}
\usepackage{tabularx}
\usepackage{longtable}
\usepackage{listings}
\usepackage{subfigure}
\usepackage{afterpage}
\usepackage{amsmath,amssymb}            
\usepackage{fancyhdr}
\usepackage{graphicx}
\usepackage{amsthm}
\usepackage[scriptsize]{caption} 
\hyphenation{a-gen-tiz-za-zio-ne}
% Margins
\topmargin=-0.45in
\evensidemargin=0in
\oddsidemargin=0in
\textwidth=6.5in
\textheight=9.0in
\headsep=0.25in

\linespread{1.1} % Line spacing

\lstset{
  numbers=left,
  stepnumber=5,    
  firstnumber=1,
  numberfirstline=true
}

% Set up the header and footer
\pagestyle{fancy}
\lhead{\hmwkAuthorName} % Top left header
\chead{\hmwkClass\ (\hmwkClassInstructor\ \hmwkClassTime): \hmwkTitle} % Top center head
\rhead{\firstxmark} % Top right header
\lfoot{\lastxmark} % Bottom left footer
\cfoot{} % Bottom center footer
\rfoot{Page\ \thepage\ of\ \protect\pageref{LastPage}} % Bottom right footer
\renewcommand\headrulewidth{0.4pt} % Size of the header rule
\renewcommand\footrulewidth{0.4pt} % Size of the footer rule

\setlength\parindent{0pt} % Removes all indentation from paragraphs

\usepackage{listings}
\usepackage{color}
\usepackage{graphicx}
\usepackage{mathrsfs}
\usepackage{framed}
\usepackage{amssymb}
\usepackage[figuresright]{rotating}
\usepackage{comment}
\usepackage{listings}
\usepackage{xcolor}
\usepackage{amsmath}
\usepackage{algpseudocode}
\usepackage{algorithm}
\usepackage{flushend}
\usepackage{times}
\usepackage{url}
\usepackage{comment}
\usepackage{paralist}
\usepackage[multiple]{footmisc}
\usepackage{amsthm}
\usepackage{multicol}


\definecolor{dkgreen}{rgb}{0,0.6,0}
\definecolor{gray}{rgb}{0.5,0.5,0.5}
\definecolor{mauve}{rgb}{0.58,0,0.82}

\lstset{frame=tb,
  language=Java,
  aboveskip=3mm,
  belowskip=3mm,
  showstringspaces=false,
  columns=flexible,
  basicstyle={\small\ttfamily},
  numbers=none,
  numberstyle=\tiny\color{gray},
  keywordstyle=\color{blue},
  commentstyle=\color{dkgreen},
  stringstyle=\color{mauve},
  breaklines=true,
  breakatwhitespace=true
  tabsize=3,
  literate={à}{{\'a}}1
  {è}{{\'e}}1
  {ì}{{\'u}}1
  {ò}{{\'o}}1
  {ù}{{\'u}}1
}

%----------------------------------------------------------------------------------------
%	DOCUMENT STRUCTURE COMMANDS
%	Skip this unless you know what you're doing
%----------------------------------------------------------------------------------------

% Header and footer for when a page split occurs within a problem environment
\newcommand{\enterProblemHeader}[1]{
\nobreak\extramarks{#1}{#1 continued on next page\ldots}\nobreak
\nobreak\extramarks{#1 (continued)}{#1 continued on next page\ldots}\nobreak
}

% Header and footer for when a page split occurs between problem environments
\newcommand{\exitProblemHeader}[1]{
\nobreak\extramarks{#1 (continued)}{#1 continued on next page\ldots}\nobreak
\nobreak\extramarks{#1}{}\nobreak
}




%----------------------------------------------------------------------------------------
%	NAME AND CLASS SECTION
%----------------------------------------------------------------------------------------

\newcommand{\hmwkTitle}{Ereditariet\`a e Interfacce} % Assignment title
\newcommand{\hmwkDueDate}{Martedi,\ Marzo 24,\ 2015} % Due date
\newcommand{\hmwkClass}{Ingegneria del Software 1} % Course/class
\newcommand{\hmwkClassTime}{} % Class/lecture time
\newcommand{\hmwkClassInstructor}{Claudio Menghi, Alessandro Rizzi} % Teacher/lecturer
\newcommand{\hmwkAuthorName}{} % Your name

%----------------------------------------------------------------------------------------
%	TITLE PAGE
%----------------------------------------------------------------------------------------

\title{
\vspace{2in}
\textmd{\textbf{\hmwkClass:\ \hmwkTitle}}\\
\normalsize\vspace{0.1in}\small{Due\ on\ \hmwkDueDate}\\
\vspace{0.1in}\large{\textit{\hmwkClassInstructor\ \hmwkClassTime}}
\vspace{3in}
}

\author{\textbf{\hmwkAuthorName}}
\date{} % Insert date here if you want it to appear below your name

%----------------------------------------------------------------------------------------

\begin{document}

\maketitle

%----------------------------------------------------------------------------------------
%	TABLE OF CONTENTS
%----------------------------------------------------------------------------------------

%\setcounter{tocdepth}{1} % Uncomment this line if you don't want subsections listed in the ToC

\newpage
\tableofcontents
\newpage

\theoremstyle{definition} 

\newtheorem{mydef}{Definition}
\newtheorem{lemma}{Lemma}

\newtheorem{theorem}{Theorem}[section]

\section{Ripasso degli elementi del linguaggio Java}
\subsection{Ereditarietà}

Nella programmazione ad oggetti l'\emph{ereditarietà} è un meccanismo tramite il quale,
è possibile definire una classe a partire da un'altra e riutilizzare gli attributi e i metodi
già definiti. I costruttori \emph{non} sono ereditati, anche se \`e possibile chiamare i costruttori del padre nel figlio.

È detta classe figlio la classe che eredita, classe padre la classe ereditata.
La classe figlio ha accesso a tutti gli attributi e i metodi della classe padre a seconda della loro visibilit\`a. Tutti gli attributi e i metodi \texttt{public} e \texttt{protected} sono visibili quelli di \texttt{default} solo se padre e figlio risiedono nello stesso package.

Con l'eccezione di  \texttt{Object}, che non ha sopra classi, ogni classe ha una unica super-classe (ereditariet\'a singola). Se una classe non specifica una sopraclasse estende implicitamente la classe \texttt{Object}.

L'idea dell'ereditariet\`a \`e molto semplice: quando vuoi creare una nuova classe $A$ ed esiste gi\`a una classe $B$ che contiene del codice che desideri puoi derivare $A$ da $B$. In questo caso puoi riusare i campi e i metodi della classe esistente senza doverli riscrivere (e debuggarli).

Che cosa \`e possibile fare con l'ereditariet\`a
\begin{itemize}
\item i campi ereditati possono essere usati direttamente, esattamente come ogni altro campo;
\item \`e possibile dichiarare attributi nella sottoclasse che non sono presenti nella classe padre;
\item \`e possibile utilizzare i metodi ereditati
\item \`e possibile scrivere un metodo nella sottoclasse con la stessa signature di quello della superclasse, mascherandolo
\item \`e possibile srivere un metodo statico nella sottoclasse con la stessa signature di quello della super-classe, mascherandolo
\item \`e possibile dichiarare nuovi metodi nella sottoclasse che non sono presenti nella super-classe
\item \`e possibile scrivere nella sotto-classe un costruttore che invoca un costruttore della superclasse usando la keyword \texttt{super}
\end{itemize}


\subsection{Overriding dei metodi}
La classe figlio può \emph{ridefinire} (\emph{override}) metodi visibili nella classe padre
\textbf{N.B.: Si parla di overriding solo se vi è visibilit\`a del metodo del padre nella classe figlio}.
La signature del metodo ridefinito deve corrispondere alla signature del metodo del padre.

Nella ridefinizione è possibile cambiare la visibilità dei metodi e il tipo di ritorno.
La nuova visibilità deve essere maggiore (secondo la scala \texttt{private} < \texttt{default} < \texttt{protected} < \texttt{public}), mentre il tipo di ritorno deve essere un sottotipo del metodo
originale ovvero \emph{covariante}\footnote{Vedremo la definizione di covarianza pi\`u avanti nel corso}.


\subsection{Polimorfismo}

Si definisce \emph{polimorfismo} il fatto che un oggetto pu\`o assumere diverse ``forme" ovvero, \`e possibile assegnare a una variabile di tipo $T$ un oggetto di tipo $S$ dove $S$ \`e una sottoclasse di $T$.


Per questo motivo in Java si distingue tra tipo statico e tipo dinamico.
\begin{itemize}
	\item \textbf{tipizzazione statica} Equivale al tipo presente nella dichiarazione della variabile.
	È unico per tutta la vita della variabile e disponibile a compile-time (basta guardarne la dichiarazione)
	\item \textbf{tipizzazione dinamica}: Equivale al tipo dell'istanza effettiva associata alla variabile (o coincide col tipo statico o è un sottotipo di quest'ultimo).
	Può cambiare durante l'esecuzione del programma.
\end{itemize}

In Java è possibile assegnare a una variabile di classe $T$ un'istanza di classe $S$
dove $S$ eredita $T$. Consideriamo le seguenti istruzioni

\begin{lstlisting}[language=Java,escapechar=|]
Automobile myCar=new AutomobileElettrica();
\end{lstlisting}
Il tipo statico di \texttt{myCar} \`e \texttt{Automobile}, il suo tipo dinamico \`e \texttt{AutomobileElettrica}.

%Viceversa è possibile assegnare un istanza di $T$ a una variabile di classe $S$ effettuando un casting:

%\begin{lstlisting}
%T t;
%S s = (S)t;
%\end{lstlisting}

%Nota bene: in quest'ultimo case l'operazione di assegnamento potrebbe non andare a buon fine, se il tipo di \texttt{t} non è compatibile con \texttt{S} (ovvero \texttt{t} non è istanza ne di S ne di un sottotipo di S).



\subsection{Binding}

Si definisce con \emph{binding} l'associazione della chiamata di un determinato \emph{metodo} all'effettivo codice associato al metodo.\\

A compile time, il compilatore non genera il codice per eseguire il metodo, ma il codice per cercare l'\emph{implementazione} corretta in base al tipo \emph{dinamico} dell'oggetto su cui il metodo \`e invocato.

Per effettuare il binding Java utilizza il seguente algoritmo.
Ipotizziamo di chiamare un metodo \texttt{foo} sull'oggetto \texttt{o}


\begin{lstlisting}
O o = ...
o.foo(...);
\end{lstlisting}

\begin{enumerate}
	\item Per prima cosa si trova il metodo all'interno della classe \texttt{O} tipo statico di \texttt{o} che abbia la segnatura corrispondente ai tipi statici degli argomenti della chiamata
	(\textbf{binding statico}).
	\item A questo punto si guarda al tipo dinamico di \texttt{o}: se 
	\texttt{o} è un'istanza di un sottotipo di \texttt{O}, \texttt{S}, si deve controllare
	se \texttt{S} ridefinisca o meno \texttt{foo}. Se è ridefinito utilizzo l'implementazione del
	sottotipo, altrimenti quella di \texttt{O} (\textbf{binding dinamico}).
\end{enumerate}

Nei metodi ridefiniti la visibilità e il tipo di ritorno possono essere cambiati







\section{Esercizi}

\subsection{Esercizio 1: Scacchiera}
\begin{framed}
\textbf{Esercizio 1}: Si implmenti il gioco degli scacchi
\begin{itemize}
\item Si ha una griglia 8x8 denominata Scacchiera
\item Ogni elemento della griglia si chiama Casella
\item  In ogni casella ci pu\`o essere al pi\`u un Pezzo
\item  I pezzi possono essere: Torre, Cavallo, Alfiere, Regina, Re, ognuno con differenti capacit\`a di movimento (si ignori il Pedone per il momento).
\item La scacchiera viene creata con dei pezzi dentro le caselle posizionati opportunamente.
\item  Una casella pu\`o essere vuota o avere un pezzo al suo interno.
\item Un pezzo pu\`o appartenere al giocatore Bianco o al giocatore Nero e pu\`o spostarsi da una casella ad unaltra secondo determinate regole dipendenti dal pezzo.
\item Un pezzo pu\`o muoversi solo verso una casella vuota od una casella occupata da un pezzo avversario. In questo secondo caso il pezzo avversario viene rimosso. I movimenti consentono di spostarsi di al pi\`u una casella in orizzontale, verticale o obliquo.
\end{itemize}
\end{framed}

Il primo passo da eseguire \`e analizzare il problema ed identificare le entit\`a (ovvero delle classi candidate). In questo caso, abbiamo \texttt{Scacchiera}, \texttt{Casella}, \texttt{Pezzo},  \texttt{Color} \ldots

\subsection{Esercizio 1.1 Colori}
Iniziamo ad analizzare un sottoproblema: modelliamo il colore delle pedine
\begin{framed}
\textbf{Esercizio 1.1} Le pedine possono avere $2$ colori: \emph{nero} e \emph{bianco}
\end{framed}
\begin{itemize}
\item come modelliamo i colori?
\begin{itemize}
\item i colori sono modellabili con un set \emph{predefinito} di costanti: \texttt{BLACK} and \texttt{WHITE}
\end{itemize}
\end{itemize}

\begin{lstlisting}[language=Java,escapechar=|]
package scacchiera;

// Le pedine possono essere di colore bianco o nero: un set definito a priori E NON modificabile di valori
public enum Color {
	WHITE, BLACK
}
\end{lstlisting}

\subsection{Esercizio 1.2 Casella}
Un nuovo sottoproblema: modelliamo la casella
\begin{framed}
\textbf{Esercizio 1.2} Analizziamo le caratteristiche di una casella:
\begin{itemize}
\item ogni casella ha una coordinata
\item ogni casella pu\`o avere un pezzo al suo interno
\item deve essere possibile aggiungere un pezzo all'interno della casella
\item deve essere possibile controllare se all'interno della casella \`e presente un pezzo
\end{itemize}
\end{framed}

Leggendo la specifica identifichiamo un nuovo elemento: \texttt{Coordinata}: 

\begin{itemize}
\item \`e una classe mutabile?
\begin{itemize}
\item sembra ragionevole implementare una \texttt{Coordinata} per mezzo di una classe immutabile
\end{itemize}
\item Quali \texttt{funzionalit\`a} offre una coordinata? 
\begin{itemize}
\item \texttt{int getX()} ritorna la coordinata X della Casella
\item \texttt{int getY()} ritorna la coordinata Y della Casella (assumiamo che la casella 0,0 sia in alto a sinistra e la casella 8,8 sia in basso a destra)
\end{itemize}
\item Come memorizziamo lo stato di una coordinata?
\begin{itemize}
\item memorizziamo lo stato per mezzo di due interi: x,y entrambi final
\end{itemize}
\item Come deve essere il costruttore di un oggetto di tipo \texttt{Coordinata}?
\begin{itemize}
\item \texttt{Coordinata(int x, int y)}
\end{itemize}
\end{itemize}

\begin{lstlisting}[language=Java,escapechar=|]
package scacchi.scacchiera;

public class Coordinata {

	private final int x;
	private final int y;
	
	public Coordinata(int x, int y){
		this.x=x;
		this.y=y;
	}

	public int getX() {
		return x;
	}

	public int getY() {
		return y;
	}
}

\end{lstlisting}

Analizziamo ora una \texttt{Casella}. La classe casella fornisce le stesse funzionalit\`a di una coordinata \texttt{getX} e \texttt{getY}: estende coordinata. 

\begin{itemize}
\item \`e una classe mutabile?
\begin{itemize}
\item No, deve essere possibile aggiungere/rimuovere un pezzo da una casella
\end{itemize}
\item Quali \texttt{funzionalit\`a} offre una coordinata? 
\begin{itemize}
\item \texttt{boolean isEmpty()} ritorna true se nella casella non \`e presente un pezzo
\item \texttt{Pezzo getPezzo()} ritorna il pezzo contenuto sulla casella, null se non sono presenti pezzi sulla casella.
\item \texttt{void setPezzo(Pezzo pezzo)} pone il \texttt{Pezzo} pezzo nella casella, se un pezzo era gi\`a presente \`e rimosso dal gioco.
\item \texttt{Pezzo unsetPezzo()} rimuove il pezzo dalla casella e ritorna il pezzo rimosso
\end{itemize}
\item Come memorizziamo lo stato di una \texttt{Casella}?
\begin{itemize}
\item \texttt{Pezzo} contiene il pezzo contenuto nella casella, \texttt{null} se nessun pezzo \`e contenuto
\end{itemize}
\item Come deve essere il costruttore di un oggetto di tipo \texttt{Casella}?
\begin{itemize}
\item \texttt{Casella(int x, int y)} dove x e y sono le coordinate della casella
\end{itemize}
\end{itemize}

\begin{lstlisting}[language=Java,escapechar=|]
package scacchi.scacchiera;

// casella estende coordinata
public class Casella extends Coordinata {

	private Pezzo pezzo;
	
	public Casella(int x, int y) {
		// chiama il costruttore di coordinata
		super(x, y);
		setPezzo(null);
		
	}

	public Pezzo getPezzo() {
		return pezzo;
	}

	public void setPezzo(Pezzo pezzo) {
		this.pezzo = pezzo;
	}
	
	public boolean isEmpty(){
		if(this.pezzo==null){
			return true;
		}
		return false;
	}
	
	public Pezzo unsetPezzo() {
		Pezzo pieceToBeReturned=this.pezzo;
		this.pezzo = null;
		return pieceToBeReturned;
	}
	@Override
	public String toString(){
		if(this.getPezzo()!=null){
			return this.getPezzo().toString();
		}
		else{
			return "         ";
		}
		
	}
}
\end{lstlisting}

\subsection{Esercizio 1.3 Pezzi}
Iniziamo ad analizzare un sottoproblema: modelliamo i pezzi
\begin{framed}
\textbf{Esercizio 1.3} I pezzi possono essere: Torre, Cavallo, Alfiere, Regina, Re, ognuno con differenti capacit\`a di movimento (si assuma che il Pedone possa effettuare solo movimenti di una casella).
\end{framed}
\begin{itemize}
\item Quale relazione esiste fra gli elementi da modellare?\\
\textit{Nel nostro caso specifico un Torre \textbf{\`e un} pezzo, un Cavallo \textbf{\`e un} pezzo,  una Re  \textbf{\`e un} pezzo\ldots. La relazione \textbf{\`e un} \`e esattamente il tipo di relazione che \`e possibile modellare con la gerarchia.}
\item Qual'\`e la super-classe e qual\`e la sottoclasse?\\
\textit{La super-classe \`e l'oggetto che contiene gli attributi che sono in comune. Nel nostro caso Pezzo \`e la super classe, Torre, Cavallo, Alfiere\ldots sono sotto-classi}
\item Come traduciamo questa relazione in Java? Usiamo una classe, una classe astratta o un interfaccia?
\textit{Se deve essere possibile istanziare un elemento con tipo uguale alla super-classe utilizziamo una classe ``normale". In caso constrario, un interfaccia fornisce uno scheletro dell'oggetto ma nessuna implementazione. Una classe astratta al contrario pu\`o contenere delle parti implementate}
\textit{Nel caso specifico non deve essere possibile istanziare un oggetto di tipo pezzo ma solo di uno dei suoi sottotipi. Quindi optiamo o per una classe astratta o per un interfaccia. Nel caso specifico utilizzeremo una classe astratta e ne vedremo dopo le motivazioni}
\end{itemize}

Come sempre \`e necessario pensare in termini di funzionalit\`a e di stato. 
\begin{itemize}
\item Il  pezzo deve fornire le seguenti funzionalit\`a:
\begin{itemize}
\item \texttt{Color getColore()} ritorna il colore del pezzo 
\item \texttt{Casella getCasella()} ritorna la casella corrente del pezzo 
\item \texttt{void setCasella(Casella casella)} cambia la casella del pezzo
\item \texttt{boolean mossaValida(Casella casella)} ritorna \texttt{true} se la casella \`e una destinazione valida per il pezzo (la mossa \`e consentita dalle regole di movimento del pezzo). Che cosa succede se la casella \`e null?
che cosa succede se la casella non 
\end{itemize}
\item come memorizziamo lo stato di pezzo? \`e necessario tenere traccia di:
\begin{itemize}
\item \texttt{Color color} che non deve cambiare dopo l'inizializzazione, quindi \`e final
\item \texttt{Casella casella} contiene la casella corrente del pezzo
\end{itemize}
\item come deve essere il costruttore di \texttt{Casella}
\begin{itemize}
\item \texttt{public Pezzo(Casella casella, Color color)} quando un \texttt{Pezzo} viene creata gli viene associato un colore e una casella iniziale
\end{itemize}
\end{itemize}

\begin{lstlisting}[language=Java,escapechar=|]
package scacchiera;

public abstract class Pezzo {
	
	private final Color color;
	private Casella casella;
	
	public Pezzo(Casella casella, Color color){
		this.setCasella(casella);
		this.color=color;
	}
	
	public Color getColor() {
		return color;
	}

	public Casella getCasella() {
		return casella;
	}

	public void setCasella(Casella casella) {
		this.casella = casella;
	}
	public abstract boolean mossaValida(Scacchiera scacchiera, Casella destinazione);
}

\end{lstlisting}
\subsubsection{Re}
La classe \texttt{Re} estende la classe Pezzo implementando la logica di movimento della Pedina \texttt{Re}. In particolare, saranno valide per il \texttt{Re} alcune mosse non valide per il cavallo etc.
\begin{lstlisting}[language=Java,escapechar=|]
package scacchi.scacchiera.pezzi;

import scacchi.scacchiera.Casella;
import scacchi.scacchiera.Color;
import scacchi.scacchiera.Pezzo;
import scacchi.scacchiera.Scacchiera;

public class Re extends Pezzo {

	public Re(Casella casella, Color color) {
		super(casella, color);
	}

	@Override
	public boolean mossaValida(Scacchiera scacchiera, Casella casellaFinale) {
		// se la casella finale non \`e nulla
		if (casellaFinale != null) {
			// se la casella non \`e vuota e \`e presente un Pezzo dello stesso
			// colore
			if (!casellaFinale.isEmpty()
					&& casellaFinale.getPezzo().getColor() == this.getColor()) {
				return false;
			}
			// ritorna true se la distanza tra la cella final e iniziale \`e di
			// una cella in orizzontale e in verticale
			return Math.abs(this.getCasella().getX() - casellaFinale.getX()) <= 1
					&& Math.abs(this.getCasella().getY() - casellaFinale.getY()) <= 1;

		}
		return false;
	}

	@Override
	public String toString() {
		return "R ["+this.getColor()+"]";
	}
}
\end{lstlisting}

\subsubsection{Alfiere}

\begin{lstlisting}[language=Java,escapechar=|]
package scacchi.scacchiera.pezzi;

import scacchi.scacchiera.Casella;
import scacchi.scacchiera.Color;
import scacchi.scacchiera.Coordinata;
import scacchi.scacchiera.Pezzo;
import scacchi.scacchiera.Scacchiera;

public class Alfiere extends Pezzo {

	public Alfiere(Casella casella, Color color) {
		super(casella, color);
	}

	@Override
	public boolean mossaValida(Scacchiera scacchiera, Casella destinazione) {
		if (destinazione != null) {
			// se la casella destinazione non \`e vuota e \`e presente un Pezzo dello stesso
			// colore ritorna falso
			if (!destinazione.isEmpty()
					&& destinazione.getPezzo().getColor() == this.getColor()) {
				return false;
			}
			// calcola la distanza della cella rispetto all'asse delle x e delle y
			int deltaX=destinazione.getX()-this.getCasella().getX();
			int deltaY=destinazione.getY()-this.getCasella().getY();
			// se il valore assoluto delle distanze \`e diverso ritorna false
			if(Math.abs(deltaX)!=Math.abs(deltaY)){
				return false;
			}
			// controlla tutte le caselle sulla diagonale, se in una di queste c'e' un pezzo ritorna false
			for(int i=1;i<Math.abs(deltaX);i++){
				if(!scacchiera.getCasella(new Coordinata(this.getCasella().getX()+Integer.signum(deltaX)*i, this.getCasella().getY()+Integer.signum(deltaY)*i)).isEmpty()){
					return false;
				}
			}
			// ritorna true
			return true;
		}
		return false;
	}

	@Override
	public String toString() {
		return "A ["+this.getColor()+"]";
	}
}
\end{lstlisting}

\subsubsection{Torre}

\begin{lstlisting}[language=Java,escapechar=|]
package scacchi.scacchiera.pezzi;

import scacchi.scacchiera.Casella;
import scacchi.scacchiera.Color;
import scacchi.scacchiera.Coordinata;
import scacchi.scacchiera.Pezzo;
import scacchi.scacchiera.Scacchiera;

public class Torre extends Pezzo {

	public Torre(Casella casella, Color color) {
		super(casella, color);
	}

	@Override
	public boolean mossaValida(Scacchiera scacchiera, Casella destinazione) {
		if (destinazione != null) {
			// se la casella destinazione non \`e vuota e \`e presente un Pezzo dello stesso
			// colore ritorna falso
			if (!destinazione.isEmpty()
					&& destinazione.getPezzo().getColor() == this.getColor()) {
				return false;
			}
			// se la riga o la colonna non e' costante ritorno false
			int deltaX=destinazione.getX()-this.getCasella().getX();
			int deltaY=destinazione.getY()-this.getCasella().getY();
			if(!((deltaX==0)||(deltaY==0))){
				return false;
			}
			if(deltaY==0){
				// controlla tutte le caselle sulla diagonale, se in una di queste c'e' un pezzo ritorna false
				for(int i=1;i<Math.abs(deltaX);i++){
					if(!scacchiera.getCasella(new Coordinata(this.getCasella().getX() +i, this.getCasella().getY())).isEmpty()){
						return false;
					}
				}
			}
			if(deltaX==0){
				for(int i=1;i<Math.abs(deltaX);i++){
					if(!scacchiera.getCasella(new Coordinata(this.getCasella().getX(), this.getCasella().getY()+i)).isEmpty()){
						return false;
					}
				}
			}
			return true;
		}
		return false;
	}

	@Override
	public String toString() {
		return "T ["+this.getColor()+"]";
	}
}
\end{lstlisting}
\subsubsection{Cavallo}
\begin{lstlisting}[language=Java,escapechar=|]
package scacchi.scacchiera.pezzi;

import scacchi.scacchiera.Casella;
import scacchi.scacchiera.Color;
import scacchi.scacchiera.Pezzo;
import scacchi.scacchiera.Scacchiera;

public class Cavallo extends Pezzo {

	public Cavallo(Casella casella, Color color) {
		super(casella, color);
	}

	@Override
	public boolean mossaValida(Scacchiera scacchiera, Casella destinazione) {
		if (destinazione != null) {
			// se la casella destinazione non \`e vuota e \`e presente un Pezzo
			// dello stesso
			// colore ritorna falso
			if (!destinazione.isEmpty()
					&& destinazione.getPezzo().getColor() == this.getColor()) {
				return false;
			}
			// calcola la distanza della cella rispetto all'asse delle x e delle
			// y
			int deltaX = destinazione.getX() - this.getCasella().getX();
			int deltaY = destinazione.getY() - this.getCasella().getY();
			if((Math.abs(deltaX)==2&&Math.abs(deltaY)==1)||(Math.abs(deltaY)==2&&Math.abs(deltaX)==1)){
				return true;
			}
		}
		return false;
	}

	@Override
	public String toString() {
		return "C ["+this.getColor()+"]";
	}
}
\end{lstlisting}

\subsubsection{Donna}
\begin{lstlisting}[language=Java,escapechar=|]
package scacchi.scacchiera.pezzi;

import scacchi.scacchiera.Casella;
import scacchi.scacchiera.Color;
import scacchi.scacchiera.Coordinata;
import scacchi.scacchiera.Pezzo;
import scacchi.scacchiera.Scacchiera;

public class Donna extends Pezzo {

	public Donna(Casella casella, Color color) {
		super(casella, color);
	}

	@Override
	public boolean mossaValida(Scacchiera scacchiera, Casella destinazione) {
		if (destinazione != null) {
			// se la casella destinazione non \`e vuota e \`e presente un Pezzo
			// dello stesso
			// colore ritorna falso
			if (!destinazione.isEmpty()
					&& destinazione.getPezzo().getColor() == this.getColor()) {
				return false;
			}
			// calcola la distanza della cella rispetto all'asse delle x e delle
			// y
			int deltaX = destinazione.getX() - this.getCasella().getX();
			int deltaY = destinazione.getY() - this.getCasella().getY();
			// se il valore assoluto delle distanze \`e diverso ritorna false
			if (Math.abs(deltaX) == Math.abs(deltaY)) {
				// controlla tutte le caselle sulla diagonale, se in una di
				// queste c'e' un pezzo ritorna false
				for (int i = 1; i < Math.abs(deltaX); i++) {
					if (!scacchiera.getCasella(
							new Coordinata(this.getCasella().getX()
									+ Integer.signum(deltaX) * i, this
									.getCasella().getY()
									+ Integer.signum(deltaY) * i)).isEmpty()) {
						return false;
					}
				}
				return true;
			}
			if ((deltaX == 0) || (deltaY == 0)) {

				if (deltaY == 0) {
					// controlla tutte le caselle sulla diagonale, se in una di
					// queste c'e' un pezzo ritorna false
					for (int i = 1; i < Math.abs(deltaX); i++) {
						if (!scacchiera.getCasella(
								new Coordinata(this.getCasella().getX() + i,
										this.getCasella().getY())).isEmpty()) {
							return false;
						}
					}
				}
				if (deltaX == 0) {
					for (int i = 1; i < Math.abs(deltaX); i++) {
						if (!scacchiera.getCasella(
								new Coordinata(this.getCasella().getX(), this
										.getCasella().getY() + i)).isEmpty()) {
							return false;
						}
					}
				}
				return true;
			}

			// ritorna true
			return false;
		}
		return false;
	}

	@Override
	public String toString() {
		return "D ["+this.getColor()+"]";
	}
}
\end{lstlisting}

\subsubsection{Pedone}
\begin{lstlisting}[language=Java,escapechar=|]
package scacchi.scacchiera.pezzi;

import scacchi.scacchiera.Casella;
import scacchi.scacchiera.Color;
import scacchi.scacchiera.Coordinata;
import scacchi.scacchiera.Pezzo;
import scacchi.scacchiera.Scacchiera;

public class Pedone extends Pezzo {

	public Pedone(Casella casella, Color color) {
		super(casella, color);
	}

	@Override
	public boolean mossaValida(Scacchiera scacchiera, Casella destinazione) {
		if (destinazione != null) {
			// se la casella destinazione non \`e vuota e \`e presente un Pezzo
			// dello stesso
			// colore ritorna falso
			if (!destinazione.isEmpty()
					&& destinazione.getPezzo().getColor() == this.getColor()) {
				return false;
			}
			// calcola la distanza della cella rispetto all'asse delle x e delle
			// y
			int deltaX = destinazione.getX() - this.getCasella().getX();
			int deltaY = destinazione.getY() - this.getCasella().getY();
			if(this.getColor().equals(Color.BLACK)&&deltaY>0){
				return false;
			}
			if(this.getColor().equals(Color.WHITE)&&deltaY<0){
				return false;
			}
			if(Math.abs(deltaY)>1||Math.abs(deltaX)>1){
				return false;
			}
			if(Math.abs(deltaY)==1&&Math.abs(deltaX)==1&&scacchiera.getCasella(new Coordinata(this.getCasella().getX()+deltaX, this.getCasella().getY()+deltaY)).isEmpty()){
				return false;
			}
			return true;
		}
		return false;
	}

	@Override
	public String toString() {
		return "P ["+this.getColor()+"]";
	}
}
\end{lstlisting}
\subsection{Esercizio 1.4 Scacchiera}
\begin{framed}
\textbf{Esercizio 1.4} Una scacchiera contiene dei pezzi, quando viene creata i pezzi sono posti sulla scacchiera in maniera corretta.
\end{framed}
\begin{lstlisting}[language=Java,escapechar=|]
package scacchi.scacchiera;

import scacchi.scacchiera.pezzi.Alfiere;
import scacchi.scacchiera.pezzi.Cavallo;
import scacchi.scacchiera.pezzi.Donna;
import scacchi.scacchiera.pezzi.Pedone;
import scacchi.scacchiera.pezzi.Re;
import scacchi.scacchiera.pezzi.Torre;

public class Scacchiera {

	/**
	 * 8x8 Matrix of caselle
	 */
	private final Casella[][] caselle;
	private static final int SIZE = 8;

	/**
	 * Constructor for Scacchiera, initializes all the fields.
	 */
	public Scacchiera() {
		// initialize fields

		caselle = new Casella[SIZE][SIZE];
		for (int i = 0; i < SIZE; i++) {
			for (int j = 0; j < SIZE; j++) {
				caselle[i][j] = new Casella(i, j);
			}
		}

		this.intializeReRow(0, Color.WHITE);
		this.intializePedoniRow(1, Color.WHITE);
		this.intializePedoniRow(6, Color.BLACK);
		this.intializeReRow(7, Color.BLACK);

	}

	private void intializePedoniRow(int row, Color color) {
		for (int i = 0; i < SIZE; i++) {
			Pezzo pedone = new Pedone(caselle[row][i], color);
			caselle[row][i].setPezzo(pedone);
		}
	}

	private void intializeReRow(int row, Color color) {
		Pezzo torreBianca1 = new Torre(caselle[row][0], color);
		caselle[row][0].setPezzo(torreBianca1);
		Pezzo cavalloBianca1 = new Cavallo(caselle[row][1], color);
		caselle[row][1].setPezzo(cavalloBianca1);
		Pezzo alfiereBianca1 = new Alfiere(caselle[row][2], color);
		caselle[row][2].setPezzo(alfiereBianca1);
		Pezzo reBianco = new Re(caselle[row][3], color);
		caselle[row][3].setPezzo(reBianco);
		Pezzo reginaBianco = new Donna(caselle[row][4], color);
		caselle[row][4].setPezzo(reginaBianco);
		Pezzo alfiereBianca2 = new Alfiere(caselle[row][5], color);
		caselle[row][5].setPezzo(alfiereBianca2);
		Pezzo cavalloBianco2 = new Cavallo(caselle[row][6], color);
		caselle[row][6].setPezzo(cavalloBianco2);
		Pezzo torreBianca2 = new Torre(caselle[row][7], color);
		caselle[row][7].setPezzo(torreBianca2);
	}

	public Casella getCasella(Coordinata coordinata) {
		return this.caselle[coordinata.getX()][coordinata.getY()];
	}

	/**
	 * @see java.lang.Object#toString()
	 */
	@Override
	public String toString() {
		String ret = "";
		ret += "---------------------------------------------------------------------------------------------------\n";
		for (int i = 0; i < SIZE; i++) {
			ret += "| ";
			for (int j = 0; j < SIZE; j++) {
				ret += caselle[i][j].toString();
				ret += " | ";
			}
			ret += "\n";
			ret += "---------------------------------------------------------------------------------------------------\n";
		}
		ret += "\n ";
		return ret;
	}
}
\end{lstlisting}

\subsection{Esercizio 1.5 Player}
\begin{framed}
\textbf{Esercizio 1.5} Nel gioco degli scacchi ci sono due giocatori a ciascuno dei quali \`e associato un colore
\end{framed}
\begin{lstlisting}[language=Java,escapechar=|]
package scacchi;

import scacchi.scacchiera.Color;

public class Player {

	private final Color color;
	
	public Player(Color color){
		this.color=color;
	}

	@Override
	public String toString() {
		return "Player [color=" + color + "]";
	}

	public Color getColor() {
		return color;
	}
}
\end{lstlisting}

\subsection{Esercizio 1.6 Gioco}
\begin{framed}
\textbf{Esercizio 1.6} Il gioco consiste in una serie di mosse iterativamente effettuate dai due giocatori
\end{framed}
\begin{lstlisting}[language=Java,escapechar=|]
package scacchi;

import java.io.IOException;
import java.util.Scanner;

import scacchi.scacchiera.Casella;
import scacchi.scacchiera.Color;
import scacchi.scacchiera.Coordinata;
import scacchi.scacchiera.Pezzo;
import scacchi.scacchiera.Scacchiera;

public class Gioco {

	private static final int N_PLAYERS = 2;
	private final Player[] players;
	private final Scacchiera scacchiera;

	public Gioco() {
		players = new Player[N_PLAYERS];
		players[0] = new Player(Color.WHITE);
		players[1] = new Player(Color.BLACK);
		scacchiera = new Scacchiera();
	}

	public void gioca() throws IOException {
		int i = 0;
		while (true) {
			System.out.println("Tocca a " + players[i % 2]);
			Player currPlayer = players[i % 2];
			boolean valid = false;
			Scanner in = new Scanner(System.in);
			do {
				System.out.println(this.scacchiera);
				System.out
						.println("Specifica la riga dove si trova la pedina che vuoi muovere");
				int riga = in.nextInt();
				System.out
						.println("Specifica la colonna dove si trova la pedina che vuoi muovere");
				int colonna = in.nextInt();

				Casella casellaSelezionata = scacchiera
						.getCasella(new Coordinata(riga, colonna));
				if (casellaSelezionata.getPezzo() != null
						&& casellaSelezionata.getPezzo().getColor()
								.equals(currPlayer.getColor())) {
					Pezzo pezzo = casellaSelezionata.getPezzo();

					System.out
							.println("Specifica la riga dove vuoi muovere il pezzo");
					int rigaDestinazione = in.nextInt();
					System.out
							.println("Specifica la colonna dove vuoi muovere il pezzo");
					int colonnaDestinazione = in.nextInt();
					Casella casellaDestinazione=this.scacchiera.getCasella(new Coordinata(rigaDestinazione, colonnaDestinazione));
					if(pezzo.mossaValida(this.scacchiera, casellaDestinazione)){
						casellaSelezionata.unsetPezzo();
						casellaDestinazione.setPezzo(pezzo);
						pezzo.setCasella(casellaDestinazione);
						valid=true;
						
					}
					else{
						System.out
						.println("Non e' possibile muovere la pedina nella colonna specificata");
				
					}

				} else {
					System.out.println("Nella casella [" + riga + "]["
							+ colonna
							+ "] non e' contenuta una pedina del tuo colore");
				}

			}while(!valid);
        	i++;
		}
	}
}
\end{lstlisting}
\subsection{Esercizio 1.7: Main}
\begin{lstlisting}[language=Java,escapechar=|]
package scacchi;

import java.io.IOException;

public class Main {

	public static void main(String[] args) throws IOException {
		Gioco gioco=new Gioco();
		gioco.gioca();
	}
}

\end{lstlisting}
\subsection{Esercizio 2: Persone e Studenti}
\begin{framed}
\textbf{Esercizio 2}: Implementare le classi per rappresentare delle persone e degli studenti.
\begin{itemize}
\item  Vogliamo tenere traccia del nome e della data di nascita di una persona.
\item Gli studenti sono delle persone cui \`e associata una lista di esami sostenuti.
\item Vogliamo controllare se lo studente ha abbastanza crediti per laurearsi.
\item Vogliamo fornire uno strumento per il calcolo della media pesata
\end{itemize}
\end{framed}

\subsubsection{Raffinamento}
Il primo passo da effettuare consiste nell'identificare le entità in gioco.
In questo caso, supponiamo vengano considerate le entità di persona, studente e voto.
Per ognuna di queste si dovrà implementare la relativa classe che ne modelli il comportamento.
Avremo quindi le classi \texttt{Person}, \texttt{Student} e \texttt{Grade}.

\subsubsection{Persona}
\begin{itemize}
\item quali funzionalit\`a deve fornire una \texttt{Persona}?
\begin{itemize}
\item \texttt{String getName()}: ritorna il nome della persona
\item \texttt{Date getBirthday()}: ritorna il compleanno della pesona
\end{itemize}
\item come memorizzo lo stato degli oggetti di tipo persona?
\begin{itemize}
\item \texttt{String name}: contiene il nome della persona
\item \texttt{Date birthday}: contiene la data di compleanno di una persona
\end{itemize}
\item come \`e strutturato il costruttore di un oggetto persona?
\begin{itemize}
\item \texttt{Person(String name, Date birthday)}: crea una persona con un determinato nome e una data data di compleanno
\end{itemize}
\item Ha senso che un elemento di tipo persona sia \texttt{immutabile?}
\end{itemize}

\begin{lstlisting}[language=Java,escapechar=|]
import java.util.Date;

public class Person {
	private final String name;
	private final Date birthday;
	public Person (String name, Date birthday){
		this.name = name;
		// creo una copia dell'oggetto date per evitare che venga modificato dall'esterno
		this.birthday = new Date(birthday.getTime());		
	}
	
	public String getName(){
		return name;
	}
	// anche qui creo una copia per evitare modifiche esterne
	public Date getBirthday(){
		return new Date(birthday.getTime());
	}
	@Override
	public String toString(){
		return "My name is " + name;
	}
}
\end{lstlisting}


\subsubsection{Grade}
\begin{itemize}
\item quali funzionalit\`a deve fornire un \texttt{Grade}?
\begin{itemize}
\item \texttt{ String getSubject()}: ritorna la materia corrispondente al grade
\item \texttt{int getPoints()}: ritorna il punteggio ottenuto
\item \texttt{int getCredits()}: ritorna i crediti dell'esame
\end{itemize}
\item come memorizzo lo stato degli oggetti di tipo persona?
\begin{itemize}
\item \texttt{String subject}: contiene il nome dell'esame
\item \texttt{int points}: contiene il punteggio ottenuto
\item \texttt{int credits}: contiene i crediti relativi all'esame
\end{itemize}
\item come \`e strutturato il costruttore di un oggetto Grade?
\begin{itemize}
\item \texttt{Grade(String subject, int points, int credits)}: crea un Grade che si riferisce a una determinata materia, con un determinato punteggio e un dato numero di crediti
\end{itemize}
\item Ha senso che un elemento di tipo Grade sia \texttt{immutabile?}
\end{itemize}

\begin{lstlisting}[language=Java,escapechar=|]
import java.util.Date;

public class Grade {
	private final String subject; 	// La materia dell'esame
	private final int points;		// La valutazione ricevuta
	private final int credits;		// Il valore in crediti dell'esame

	public Grade(String subject, int points, int credits) {
	    // che cosa succede se il punteggio e' minore di 18 e se e' maggiore di 30?
		this.subject = subject;
		this.points = points;
		this.credits = credits;
	}

	public String getSubject() {
		return subject;
	}

	public int getPoints() {
		return points;
	}

	public int getCredits() {
		return credits;
	}
}
\end{lstlisting}

\subsubsection{Studente}
\begin{itemize}
\item quali funzionalit\`a deve fornire uno \texttt{Studente}?
\begin{itemize}
\item \texttt{void addGrade(Grade grade)}: aggiunge una valutazione alla carriera dello studente
\item \texttt{boolean canGraduate()}: controlla se lo studente pu\`o graduarsi
\item \texttt{double getWeightedGradeAverage()}: ritorna la media pesata dello studente
\end{itemize}
\item come memorizzo lo stato degli oggetti di tipo Studenti? Nota \`e ragionevole che lo studente estenda Persona
\begin{itemize}
\item \texttt{List<Grade>}: contiene la lista degli esami effettuati dallo studente
\end{itemize}
\item come \`e strutturato il costruttore di un oggetto Studente?
\begin{itemize}
\item \texttt{Student (String name, Date birthday)}: crea uno studente con un dato nome e data di nascita
\end{itemize}
\item Ha senso che un elemento di tipo Studente sia \texttt{immutabile?}: No, gli esami associati allo studente devono cambiare (anche se uno sviluppatore potrebbe comunque decidere di implementarlo mediante una classe immutabile.
\end{itemize}
\begin{lstlisting}[language=Java,escapechar=|]
import java.util.ArrayList;
import java.util.Date;
import java.util.List;

public class Student extends Person{
	private final List<Grade> grades; 
	
	public Student (String name, Date birthday){
		super(name,birthday);		
		grades = new ArrayList<Grade>();
	}

	// Aggiunge una valutazione alla carriera dello studente.
	public void addGrade(Grade grade) {
		grades.add(grade);
	}

	// Controlla se lo studente ha abbastanza crediti per potersi laureare.
	public boolean canGraduate() {
		return totalCredits() >= 180;
	}

	// Calcola la media pesata.
	public double getWeightedGradeAverage() {
		double sumOfWeightedPoints = 0;
		for (Grade grade : grades) {
			sumOfWeightedPoints += grade.getCredits() * grade.getPoints();
		}
		return sumOfWeightedPoints / totalCredits();
	}

	// Calcola il numero di crediti sostenuti dallo studente.
	private int totalCredits() {
		int totalCredits = 0;
		for (Grade grade : grades) {
			totalCredits += grade.getCredits();
		}
		return totalCredits;
	}

	@Override
	public String toString() {
		return super.toString() + " and I am a student";
	}
}
\end{lstlisting}

\subsubsection{Main Esempio}
\begin{lstlisting}[language=Java,escapechar=|]
import java.util.Date;

/**
 * classe di esempio per mostrare 
 * la differenza tra tipi statici e dinamici
 * @author valerio
 *
 */
public class Esempio {
	public static void main(String[] args){
		// Il costruttore di Date e' deprecato, lo usiamo
		// solo per comodita' in questo esempio.
		Person bill = new Person("Bill", new Date(1955,10,28));
		Person pippo = new Student("Pippo", new Date(1930,05,02));		
		System.out.println(bill);
		System.out.println(pippo);
	}
}
\end{lstlisting}
\begin{itemize}
\item Qual \`e il po statico di ciascuna variabile?: \textit{il tipo statico di bill e di pippo \`e Person}
\item Qual \`e il tipo dinamico (dopo l'inizializzazione)? \textit{Il tipo dinamico di bill \`e persona, mentre di pippo \`e studente}
\item  Cosa stampa? Perch\`e? \textit{Stampa ''My name is Bill" e "My name is Pippo and I am a student"} 
\item  Cosa succederebbe se scrivessi l’istruzione pippo.canGraduate() \textit{Errore di compilazione, il tipo statico di Pippo \`e persona, quindi il compilatore cerca nel tipo \texttt{Person} la signature del metodo \texttt{canGraduate} che non \`e presente.}
\end{itemize}

\subsection{Esercizio 3: Forme Geometriche}
\begin{framed}
\textbf{Esercizio 3}: Implementare una gerarchia di classe che rappresenti delle forme geometriche, permettendo di calcolare
\begin{itemize}
\item  La loro area
\item Perimetro
\item E di rappresentarle sullo schermo.
\end{itemize}
\end{framed}

\subsubsection{TwoDimensionalCanvas}
Perch\`e usiamo un interfaccia per modellare TwoDimensionalCanvas?

\emph{Una classe (anche astratta) ci dice che un oggetto \`e qualcosa, una interfaccia rappresenta un comportamento che la classe ha (indipendentemente da come \`e fatto l’oggetto). Non conosciamo nulla su come \`e fatto TwoDimensionalCanvas sappiamo solamente che alcuni oggetti avranno le funzionalit\`a specificate in TwoDimensionalCanvas}

\begin{lstlisting}[language=Java,escapechar=|]
public interface TwoDimensionalCanvas {
	void drawPoint(Point point);

	void drawLine(Point firstEndPoint, Point secondEndPoint);

	void drawCircle(Point center, double radius);
}
\end{lstlisting}


\subsubsection{Point}
\begin{lstlisting}[language=Java,escapechar=|]
public class Point {
	private final double x;
	private final double y;

	public Point(double x, double y) {
		this.x = x;
		this.y = y;
	}

	public double getX() {
		return x;
	}

	public double getY() {
		return y;
	}

	public Point rotate(Point center, double degrees) {
		double rotatedX = center.x + (x - center.x) * Math.cos(degrees) - (y - center.y) * Math.sin(degrees);
		double rotatedY = center.y + (x - center.x) * Math.sin(degrees) + (y - center.y) * Math.cos(degrees);
		return new Point(rotatedX, rotatedY);
	}
	
	public String toString(){
		return "x = " + x + " y = " + y;
	}
}
\end{lstlisting}


\subsubsection{Shape}
Prima di tutto definiamo la classe astratta \texttt{Shape}. 
Perch\`e usiamo una classe astratta?

\emph{Una classe (anche astratta) ci dice che un oggetto \`e qualcosa, una interfaccia rappresenta un comportamento che la classe ha (indipendentemente da come \`e fatto l’oggetto)}

\begin{itemize}
\item quali funzionalit\`a deve fornire uno \texttt{Shape}?
\begin{itemize}
\item \texttt{double getArea()}: ritorna l'area della figura
\item \texttt{ double getPerimeter()}: ritorna il perimetro della figura
\item \texttt{List<Point> getSequenceOfPointsToDraw();}: ritorna i punti da disegnare
\item \texttt{public void draw(TwoDimensionalCanvas canvas)}: stampa la figura su un particolare canvas
\end{itemize}
\end{itemize}
\begin{lstlisting}[language=Java,escapechar=|]
import java.util.List;

public abstract class Shape {
	public abstract double getArea();
	public abstract double getPerimeter();
	public abstract List<Point> getSequenceOfPointsToDraw();
	
	public void draw(TwoDimensionalCanvas canvas){
		List<Point> points = getSequenceOfPointsToDraw();
		for (int i = 0; i < points.size() - 1; i++ ){
			canvas.drawLine(points.get(i), points.get(i + 1));
		}
	}
}
\end{lstlisting}

\subsubsection{Circle}
\begin{lstlisting}[language=Java,escapechar=|]
import java.util.ArrayList;
import java.util.List;

public class Circle extends Shape{
	private final Point center;
	private final double radius;

	public Circle(Point center, double radius) {
		his.center = center;
		this.radius = radius;
	}

	@Override
	public double getArea() {
		return Math.PI * radius * radius;
	}

	@Override
	public double getPerimeter() {
		return 2 * Math.PI * radius;
	}

	@Override
	public List<Point> getSequenceOfPointsToDraw() {
		// Primo approccio: 
		// Approssimiamo il cerchio con una figura regolare con 256 lati
		List<Point> points = new ArrayList<Point>();
		int numberOfSegments = 256;
		for (int i = 0; i < numberOfSegments; i++) {
			double newPointX = center.getX() + radius * Math.cos(2 * Math.PI * i / numberOfSegments);
			double newPointY = center.getY() + radius * Math.sin(2 * Math.PI * i / numberOfSegments);
			points.add(new Point(newPointX, newPointY));
		}
		return points;
	}

	@Override
	public void draw(TwoDimensionalCanvas canvas) {
		/* Secondo approccio: overriding di draw
		 * Seguendo questo approccio, possiamo utilizzare le primitive per
		 * disegnare messe a disposizione dal piano. L'introduzione di una nuova
		 * forma richiede un minor numero di modifiche al codice.
		 */
		canvas.drawCircle(center, radius);
	}
}
\end{lstlisting}


\subsubsection{Square}
\begin{lstlisting}[language=Java,escapechar=|]

import java.util.ArrayList;
import java.util.List;

public class Square extends Shape{

	private final Point initialPoint;
	// It represents the orientation, in radiant degrees of the square
	private final double orientation;
	private final double sideLength;

	public Square(Point initialPoint, double orientation, double sideLength) {
	
		this.initialPoint = initialPoint;
		this.orientation = orientation;
		this.sideLength = sideLength;
	}

	@Override
	public double getArea() {
		return sideLength * sideLength;
	}

	@Override
	public double getPerimeter() {
		return 4 * sideLength;
	}

	@Override
	public List<Point> getSequenceOfPointsToDraw() {
		List<Point> points = new ArrayList<Point>();
		points.add(initialPoint);
		Point nonOrientedSecondPoint = new Point(initialPoint.getX(), initialPoint.getY() + sideLength);
		points.add(nonOrientedSecondPoint.rotate(initialPoint, orientation));
		Point nonOrientedThirdPoint = new Point(initialPoint.getX() + sideLength, initialPoint.getY() + sideLength);
		points.add(nonOrientedThirdPoint.rotate(initialPoint, orientation));
		Point nonOrientedFourthPoint = new Point(initialPoint.getX() + sideLength, initialPoint.getY());
		points.add(nonOrientedFourthPoint.rotate(initialPoint, orientation));
		return points;
	}
}
\end{lstlisting}

\subsubsection{TextualTwoDimensionalCanvas}
\begin{lstlisting}[language=Java,escapechar=|]
public class TextualTwoDimensionalCanvas implements TwoDimensionalCanvas{

	@Override
	public void drawPoint(Point point) {
		System.out.println("drawing point: " + point);
		
	}

	@Override
	public void drawLine(Point firstEndPoint, Point secondEndPoint) {
		System.out.println("drawing Line from: " + 
	firstEndPoint + " to: " + secondEndPoint);
		
	}

	@Override
	public void drawCircle(Point center, double radius) {
		System.out.println("drawing circle with center: " + 
	center + " and radius: " + radius);
		
	}

}
\end{lstlisting}
\subsubsection{ShapeClient}
\begin{lstlisting}[language=Java,escapechar=|]
public class ShapeClient {
	public static void main(String[] args){
		Shape s1 = new Square(new Point(0,0), 0 ,10);
		Shape s2 = new Circle(new Point(0,0),5);
		TwoDimensionalCanvas canvas = new TextualTwoDimensionalCanvas();
		s1.draw(canvas);
		s2.draw(canvas);
	}
}
\end{lstlisting}



\subsection{Esercizio 4: Animali (Interfacce)}


\subsection{Esercizio 5: Forme Geometriche (Resizable, Rotatable)}
\begin{framed}
\textbf{Esercizio 5}: Deve essere possibile ruotare e ridimensionare gli oggetti shape. A seconda della loro natura (e.g., un cerchio non pu\`o essere ruotato ma pu\`o essere ridimensionato)
\end{framed}

\subsubsection{Rotatable}
\begin{lstlisting}[language=Java,escapechar=|]
public interface Rotatable {
	public Shape rotate(double angle);
}
\end{lstlisting}

\subsubsection{Resizable}
\begin{lstlisting}[language=Java,escapechar=|]
public interface Resizable {
	public Shape resize(double scale);
}
\end{lstlisting}

\subsubsection{Square}

\begin{lstlisting}[language=Java,escapechar=|]
public class Square extends Shape implements Resizable, Rotatable{
     //...
      @Override
       public Square rotate(double angle) {
             return new Square(initialPoint, orientation+angle, sideLength);
       }
       @Override
        public Square resize(double scale) {
             return new Square(initialPoint, orientation, sideLength*scale);
       }
}
\end{lstlisting}

\subsubsection{Circle}



\begin{lstlisting}[language=Java,escapechar=|]
public class Circle extends Shape implements Resizable { 
    //..
    @Override
    public Circle resize(double scale) {
    return new Circle(center, radius*scale);
    }
}
\end{lstlisting}

\subsection{Esercizio 6: Tipo statico e tipo dinamico}
Dire cosa stampa il seguente programma 
\begin{lstlisting}[language=Java,escapechar=|]
public static void main(String[] args) {
		ClassA a1, a2;
		ClassB b1;
		ClassC c1;

		a1 = new ClassB();
		b1 = new ClassB();
		c1 = new ClassC();
		a2 = new ClassC();

		/*
		 * b1 e' di classe ClassB, viene chiesto di invocare il metodo stampa su
		 * un parametro b1 di tipo statico ClassB. A compile scelgo la signature
		 * "compatibile" e piu' "vicina", considerando i tipi statici delle
		 * variabili. In particolare in questo caso la signature scelta e'
		 * public void stampa(ClassB p).
		 * 
		 * A run time b1 e' di classe classB e e il parametro p, che equivale a
		 * B1 e' anch'esso di classe ClassB per cui viene chiamato il metodo
		 * public void stampa(ClassB p) di class B (stampa BBB)
		 */
		b1.stampa(b1);
		/*
		 * a1 e' di classe ClassA, viene chiesto di invocare il metodo stampa su
		 * un parametro b1 di tipo statico ClassB. A compile scelgo la signature
		 * "compatibile" e piu' "vicina", considerando i tipi statici delle
		 * variabili. In particolare in questo caso la signature scelta e'
		 * public void stampa(ClassA p) della classe ClassA.
		 * 
		 * A run-time a1 e' di classe ClassB. Per questo partendo da classB e
		 * risalendo nella gerarchia di raffinamento cerco un metodo che matchi
		 * la signature public void stampa(ClassA p). In questo caso il metodo
		 * e' gia' presente in ClassB. Quindi viene invocato il metodo public
		 * void stampa(ClassA p) della classe ClassB (Stampa AAA/BBB)
		 */
		a1.stampa(b1);

		/*
		 * b1 e' di classe ClassB, viene chiesto di invocare il metodo stampa su
		 * un parametro c1 di tipo statico ClassC. A compile scelgo la signature
		 * "compatibile" e piu' "vicina", considerando i tipi statici delle
		 * variabili (ClassB). In particolare in questo caso la signature scelta
		 * all'interno di ClassB e' public void stampa(ClassA p) della classe
		 * ClassB.
		 * 
		 * A run-time b1 e' di classe ClassB, c1 e' di classe ClassC. Per questo
		 * partendo da classB e risalendo nella gerarchia di raffinamento cerco
		 * un metodo che matchi la signature public void stampa(ClassA p). In
		 * questo caso il metodo e' gia' presente in ClassB. Quindi viene
		 * invocato il metodo public void stampa(ClassA p) della classe ClassB
		 * (Stampa AAA/BBB)
		 */
		b1.stampa(c1);

		/*
		 * c1 e' di classe statica ClassC, viene chiesto di invocare il metodo stampa su
		 * un parametro c1 di tipo statico ClassC. A compile scelgo la signature
		 * "compatibile" e piu' "vicina", considerando i tipi statici delle
		 * variabili (ClassC). In particolare in questo caso la signature scelta
		 * all'interno di ClassC e' public void stampa(ClassC p) della classe
		 * ClassC.
		 * 
		 * A run-time c1 e' di classe ClassC, c1 e' di classe ClassC. Per questo
		 * partendo da ClassC e risalendo nella gerarchia di raffinamento cerco
		 * un metodo che matchi la signature public void stampa(ClassC p). In
		 * questo caso il metodo e' gia' presente in ClassC. Quindi viene
		 * invocato il metodo public void stampa(ClassC p) della classe ClassC
		 * (Stampa CCC)
		 */
		c1.stampa(c1);
		
		/*
		 * c1 e' di classe statica ClassC, viene chiesto di invocare il metodo stampa su
		 * un parametro a1 di tipo statico ClassA. A compile scelgo la signature
		 * "compatibile" e piu' "vicina", considerando i tipi statici delle
		 * variabili. In particolare in questo caso la signature scelta
		 * all'interno di ClassC e' public void stampa(ClassA p) della classe
		 * ClassC.
		 * 
		 * A run-time c1 e' di classe ClassC, a1 e' di classe ClassB. Per questo
		 * partendo da ClassC e risalendo nella gerarchia di raffinamento cerco
		 * un metodo che matchi la signature public void stampa(ClassA p). In
		 * questo caso il metodo e' gia' presente in ClassC. Quindi viene
		 * invocato il metodo public void stampa(ClassA p) della classe ClassC
		 * (Stampa AAA/CCC)
		 */
		c1.stampa(a1);

		/*
		 * a2 e' di classe statica ClassA, viene chiesto di invocare il metodo stampa su
		 * un parametro c1 di tipo statico ClassC. A compile scelgo la signature
		 * "compatibile" e piu' "vicina", considerando i tipi statici delle
		 * variabili. In particolare in questo caso la signature scelta
		 * all'interno di ClassA e' public void stampa(ClassA p) della classe
		 * ClassA.
		 * 
		 * A run-time a2 e' di classe ClassC, c1 e' di classe ClassC. Per questo
		 * partendo da ClassC e risalendo nella gerarchia di raffinamento cerco
		 * un metodo che matchi la signature public void stampa(ClassA p). In
		 * questo caso il metodo e' gia' presente in ClassC. Quindi viene
		 * invocato il metodo public void stampa(ClassA p) della classe ClassC
		 * (Stampa AAA/CCC)
		 */
		a2.stampa(c1);
	}
\end{lstlisting}
dove 
\begin{lstlisting}[language=Java,escapechar=|]
public class ClassA {

	public void stampa(ClassA p){
		System.out.println("AAA");
	}
}
\end{lstlisting}

\begin{lstlisting}[language=Java,escapechar=|]
public class ClassB extends ClassA {
	public void stampa(ClassB p){
		System.out.println("BBB");
	}
	public void stampa(ClassA p){
		System.out.println("AAA/BBB");
	}
}
\end{lstlisting}

\begin{lstlisting}[language=Java,escapechar=|]
public class ClassC extends ClassA {
	public void stampa(ClassC p){
		System.out.println("CCC");
	}
	public void stampa(ClassA p){
		System.out.println("AAA/CCC");
	}
}
\end{lstlisting}

\subsubsection{Esercizio 6a}
\begin{framed}
\textbf{Esercizio 6a}: Cosa succede se \texttt{ClassC} eredita da \texttt{ClassB} e non da \texttt{ClassA}
\end{framed}
\begin{lstlisting}[language=Java,escapechar=|]
package tipostaticoedinamico;

public class Main {

	public static void main(String[] args) {
		ClassA a1, a2;
		ClassB b1;
		ClassC c1;

		a1 = new ClassB();
		b1 = new ClassB();
		c1 = new ClassC();
		a2 = new ClassC();

		/*
		 * b1 e' di classe ClassB, viene chiesto di invocare il metodo stampa su
		 * un parametro b1 di tipo statico ClassB. A compile scelgo la signature
		 * "compatibile" e piu' "vicina", considerando i tipi statici delle
		 * variabili. In particolare in questo caso la signature scelta e'
		 * public void stampa(ClassB p).
		 * 
		 * A run time b1 e' di classe classB e e il parametro p, che equivale a
		 * B1 e' anch'esso di classe ClassB per cui viene chiamato il metodo
		 * public void stampa(ClassB p) di class B (stampa BBB)
		 */
		b1.stampa(b1);
		/*
		 * a1 e' di classe ClassA, viene chiesto di invocare il metodo stampa su
		 * un parametro b1 di tipo statico ClassB. A compile scelgo la signature
		 * "compatibile" e piu' "vicina", considerando i tipi statici delle
		 * variabili. In particolare in questo caso la signature scelta e'
		 * public void stampa(ClassA p) della classe ClassA.
		 * 
		 * A run-time a1 e' di classe ClassB. Per questo partendo da classB e
		 * risalendo nella gerarchia di raffinamento cerco un metodo che matchi
		 * la signature public void stampa(ClassA p). In questo caso il metodo
		 * e' gia' presente in ClassB. Quindi viene invocato il metodo public
		 * void stampa(ClassA p) della classe ClassB (Stampa AAA/BBB)
		 */
		a1.stampa(b1);

		/*
		 * b1 e' di classe ClassB, viene chiesto di invocare il metodo stampa su
		 * un parametro c1 di tipo statico ClassC. A compile scelgo la signature
		 * "compatibile" e piu' "vicina", considerando i tipi statici delle
		 * variabili (ClassB). In particolare in questo caso la signature scelta
		 * all'interno di ClassB e' public void stampa(ClassB p) della classe
		 * ClassB.
		 * 
		 * A run-time b1 e' di classe ClassB, c1 e' di classe ClassC. Per questo
		 * partendo da classB e risalendo nella gerarchia di raffinamento cerco
		 * un metodo che matchi la signature public void stampa(ClassB p). In
		 * questo caso il metodo e' gia' presente in ClassB. Quindi viene
		 * invocato il metodo public void stampa(ClassB p) della classe ClassB
		 * (Stampa BBB)
		 */
		b1.stampa(c1);

		/*
		 * c1 e' di classe statica ClassC, viene chiesto di invocare il metodo stampa su
		 * un parametro c1 di tipo statico ClassC. A compile scelgo la signature
		 * "compatibile" e piu' "vicina", considerando i tipi statici delle
		 * variabili (ClassC). In particolare in questo caso la signature scelta
		 * all'interno di ClassC e' public void stampa(ClassC p) della classe
		 * ClassC.
		 * 
		 * A run-time c1 e' di classe ClassC, c1 e' di classe ClassC. Per questo
		 * partendo da ClassC e risalendo nella gerarchia di raffinamento cerco
		 * un metodo che matchi la signature public void stampa(ClassC p). In
		 * questo caso il metodo e' gia' presente in ClassC. Quindi viene
		 * invocato il metodo public void stampa(ClassC p) della classe ClassC
		 * (Stampa CCC)
		 */
		c1.stampa(c1);
		
		/*
		 * c1 e' di classe statica ClassC, viene chiesto di invocare il metodo stampa su
		 * un parametro a1 di tipo statico ClassA. A compile scelgo la signature
		 * "compatibile" e piu' "vicina", considerando i tipi statici delle
		 * variabili. In particolare in questo caso la signature scelta
		 * all'interno di ClassC e' public void stampa(ClassA p) della classe
		 * ClassC.
		 * 
		 * A run-time c1 e' di classe ClassC, a1 e' di classe ClassB. Per questo
		 * partendo da ClassC e risalendo nella gerarchia di raffinamento cerco
		 * un metodo che matchi la signature public void stampa(ClassA p). In
		 * questo caso il metodo e' gia' presente in ClassC. Quindi viene
		 * invocato il metodo public void stampa(ClassA p) della classe ClassC
		 * (Stampa AAA/CCC)
		 */
		c1.stampa(a1);

		/*
		 * a2 e' di classe statica ClassA, viene chiesto di invocare il metodo stampa su
		 * un parametro c1 di tipo statico ClassC. A compile scelgo la signature
		 * "compatibile" e piu' "vicina", considerando i tipi statici delle
		 * variabili. In particolare in questo caso la signature scelta
		 * all'interno di ClassA e' public void stampa(ClassA p) della classe
		 * ClassA.
		 * 
		 * A run-time a2 e' di classe ClassC, c1 e' di classe ClassC. Per questo
		 * partendo da ClassC e risalendo nella gerarchia di raffinamento cerco
		 * un metodo che matchi la signature public void stampa(ClassA p). In
		 * questo caso il metodo e' gia' presente in ClassC. Quindi viene
		 * invocato il metodo public void stampa(ClassA p) della classe ClassC
		 * (Stampa AAA/CCC)
		 */
		a2.stampa(c1);
	}
}
\end{lstlisting}

\subsection{Esercizio 7: Tipo statico e tipo dinamico}
Dire cosa stampa il seguente programma 
\begin{lstlisting}[language=Java,escapechar=|]
public class Main {

	public static void main(String[] args) {
		Figura f1, f2;
		Quadrato q=new Rettangolo();
		Rettangolo r=new Rettangolo();
		f1=new Quadrato();
		f2=new Rettangolo();
		
		/*
		 * f1 e' di classe statica Figura, viene chiesto di invocare il metodo stampa su
		 * un parametro f2 di tipo statico Figura. A compile scelgo la signature
		 * "compatibile" e piu' "vicina", considerando i tipi statici delle
		 * variabili. In particolare in questo caso la signature scelta
		 * all'interno di Figura e' public void stampa(Figura p) della classe
		 * Figura.
		 * 
		 * A run-time f1 e' di classe Quadrato, f2 e' di classe Rettangolo. Per questo
		 * partendo da Quadrato e risalendo nella gerarchia di raffinamento cerco
		 * un metodo che matchi la signature public void stampa(Figura p). In
		 * questo caso il metodo e' presente solo in Figura. Quindi viene
		 * invocato il metodo public void stampa(Figura p) della classe Figura
		 * (Stampa Figura)
		 */
		f1.stampa(f2);
		
		/*
		 * q e' di classe statica Quadrato, viene chiesto di invocare il metodo stampa su
		 * un parametro r di tipo statico Rettangolo. A compile scelgo la signature
		 * "compatibile" e piu' "vicina", considerando i tipi statici delle
		 * variabili. In particolare in questo caso la signature scelta
		 * all'interno di Quadrato e' public void stampa(Rettangolo p) della classe
		 * Quadrato.
		 * 
		 * A run-time q e' di classe Rettangolo, r e' di classe Rettangolo. Per questo
		 * partendo da Rettangolo e risalendo nella gerarchia di raffinamento cerco
		 * un metodo che matchi la signature public void stampa(Rettangolo p). In
		 * questo caso il metodo e' presente in Rettangolo. Quindi viene
		 * invocato il metodo public void stampa(Rettangolo p) della classe Rettangolo
		 * (Stampa Rettangolo Particolare)
		 */
		q.stampa(r);
		
		/*
		 * f1 e' di classe statica Figura, viene chiesto di invocare il metodo stampa su
		 * un parametro q di tipo statico Quadrato. A compile scelgo la signature
		 * "compatibile" e piu' "vicina", considerando i tipi statici delle
		 * variabili. In particolare in questo caso la signature scelta
		 * all'interno di Figura e' public void stampa(Figura p) della classe
		 * Figura.
		 * 
		 * A run-time f1 e' di classe Quadrato, q e' di classe Rettangolo. Per questo
		 * partendo da Quadrato e risalendo nella gerarchia di raffinamento cerco
		 * un metodo che matchi la signature public void stampa(Figura p). In
		 * questo caso il metodo e' presente in Figura. Quindi viene
		 * invocato il metodo public void stampa(Figura p) della classe Figura
		 * (Stampa Figura)
		 */
		f1.stampa(q);
		
		/*
		 * q e' di classe statica Quadrato, viene chiesto di invocare il metodo stampa su
		 * un parametro f1 di tipo statico Figura. A compile scelgo la signature
		 * "compatibile" e piu' "vicina", considerando i tipi statici delle
		 * variabili. In particolare in questo caso la signature scelta
		 * all'interno di Quadrato e' public void stampa(Figura q) della classe
		 * Figura.
		 * 
		 * A run-time q e' di classe Rettangolo, f1 e' di classe Quadrato. Per questo
		 * partendo da Rettangolo e risalendo nella gerarchia di raffinamento cerco
		 * un metodo che matchi la signature public void stampa(Figura p). In
		 * questo caso il metodo e' presente in Rettangolo. Quindi viene
		 * invocato il metodo public void stampa(Figura p) della classe Rettangolo
		 * (Stampa Figura particolare)
		 */
		q.stampa(f1);
		
		/*
		 * q e' di classe statica Quadrato, viene chiesto di invocare il metodo stampa su
		 * un parametro q di tipo statico Quadrato. A compile scelgo la signature
		 * "compatibile" e piu' "vicina", considerando i tipi statici delle
		 * variabili. In particolare in questo caso la signature scelta
		 * all'interno di Quadrato e' public void stampa(Quadrato q) della classe
		 * Quadrato.
		 * 
		 * A run-time q e' di classe Rettangolo. Per questo
		 * partendo da Rettangolo e risalendo nella gerarchia di raffinamento cerco
		 * un metodo che matchi la signature public void stampa(Quadrato p). In
		 * questo caso il metodo e' presente in Rettangolo. Quindi viene
		 * invocato il metodo public void stampa(Quadrato p) della classe Rettangolo
		 * (Stampa Rettangolo particolare)
		 */
		q.stampa(q);

	}
}
\end{lstlisting}

Dove
\begin{lstlisting}[language=Java,escapechar=|]
public class Figura {
	public void stampa(Figura f){
		System.out.println("Figura");
	}
}
\end{lstlisting}
\begin{lstlisting}[language=Java,escapechar=|]
public class Quadrato extends Figura {
	public void stampa(Quadrato q){
		System.out.println("Quadrato");
	}
}
\end{lstlisting}
\begin{lstlisting}[language=Java,escapechar=|]
public class Rettangolo extends Quadrato {
	public void stampa(Rettangolo q) {
		System.out.println("Rettangolo");
	}

	public void stampa(Quadrato q) {
		System.out.println("Rettangolo particolare");
	}

	public void stampa(Figura q) {
		System.out.println("Figura particolare");
	}
}
\end{lstlisting}

\section{Esercizi per casa}




\clearpage

% ---- Bibliography ----




\addcontentsline{toc}{chapter}{Bibliography}
\bibliographystyle{alpha}
\bibliography{bib}
\nocite{*}


\end{document}

