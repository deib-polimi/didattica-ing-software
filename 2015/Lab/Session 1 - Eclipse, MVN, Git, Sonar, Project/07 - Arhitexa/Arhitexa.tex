\documentclass{article}

\usepackage[T1]{fontenc}
\usepackage{fancyhdr} % Required for custom headers
\usepackage{lastpage} % Required to determine the last page for the footer
\usepackage{extramarks} % Required for headers and footers
\usepackage[usenames,dvipsnames]{color} % Required for custom colors
\usepackage{graphicx} % Required to insert images
\usepackage{listings} % Required for insertion of code
\usepackage{courier} % Required for the courier font
\usepackage{lipsum} % Used for inserting dummy 'Lorem ipsum' text into the template
\usepackage{hyperref}
\usepackage{color}
\usepackage[normalem]{ulem}
\usepackage{url}

% Margins
\topmargin=-0.45in
\evensidemargin=0in
\oddsidemargin=0in
\textwidth=6.5in
\textheight=9.0in
\headsep=0.25in

\linespread{1.1} % Line spacing

% Set up the header and footer
\pagestyle{fancy}
\lhead{\hmwkAuthorName} % Top left header
\chead{\hmwkClass\ (\hmwkClassInstructor): \hmwkTitle} % Top center head
\rhead{\firstxmark} % Top right header
\lfoot{\lastxmark} % Bottom left footer
\cfoot{} % Bottom center footer
\rfoot{Page\ \thepage\ of\ \protect\pageref{LastPage}} % Bottom right footer
\renewcommand\headrulewidth{0.4pt} % Size of the header rule
\renewcommand\footrulewidth{0.4pt} % Size of the footer rule

\setlength\parindent{0pt} % Removes all indentation from paragraphs

\usepackage{listings}
\usepackage{color}

\definecolor{dkgreen}{rgb}{0,0.6,0}
\definecolor{gray}{rgb}{0.5,0.5,0.5}
\definecolor{mauve}{rgb}{0.58,0,0.82}

\lstset{frame=tb,
  language=Java,
  aboveskip=3mm,
  belowskip=3mm,
  showstringspaces=false,
  columns=flexible,
  basicstyle={\small\ttfamily},
  numbers=none,
  numberstyle=\tiny\color{gray},
  keywordstyle=\color{blue},
  commentstyle=\color{dkgreen},
  stringstyle=\color{mauve},
  breaklines=true,
  breakatwhitespace=true
  tabsize=3
}

\DeclareUrlCommand\ULurl{%
  \renewcommand\UrlFont{\ttfamily\color{blue}}%
  \renewcommand\UrlLeft{\uline\bgroup}%
  \renewcommand\UrlRight{\egroup}}



%----------------------------------------------------------------------------------------
%	DOCUMENT STRUCTURE COMMANDS
%	Skip this unless you know what you're doing
%----------------------------------------------------------------------------------------

% Header and footer for when a page split occurs within a problem environment
\newcommand{\enterProblemHeader}[1]{
\nobreak\extramarks{#1}{#1 continued on next page\ldots}\nobreak
\nobreak\extramarks{#1 (continued)}{#1 continued on next page\ldots}\nobreak
}

% Header and footer for when a page split occurs between problem environments
\newcommand{\exitProblemHeader}[1]{
\nobreak\extramarks{#1 (continued)}{#1 continued on next page\ldots}\nobreak
\nobreak\extramarks{#1}{}\nobreak
}




%----------------------------------------------------------------------------------------
%	NAME AND CLASS SECTION
%----------------------------------------------------------------------------------------

\newcommand{\hmwkTitle}{Arhitexa} % Assignment title
\newcommand{\hmwkDueDate}{Aprile 21, 2015} % Due date
\newcommand{\hmwkClass}{Ingegneria del Software 1} % Course/class
\newcommand{\hmwkClassInstructor}{Sr\dj{}an Krsti\'c and Marco Scavuzzo} % Teacher/lecturer
%\newcommand{\hmwkClassTime}{} % Class/lecture time
\newcommand{\hmwkAuthorName}{} % Your name

%----------------------------------------------------------------------------------------
%	TITLE PAGE
%----------------------------------------------------------------------------------------

\title{
\vspace{2in}
\textmd{\textbf{\hmwkClass:\ \hmwkTitle}}\\
\normalsize\vspace{0.1in}\small{Da completare entro \hmwkDueDate}\\
\vspace{0.1in}\large{\textit{\hmwkClassInstructor}}
\vspace{3in}
}

\author{\textbf{\hmwkAuthorName}}
\date{} % Insert date here if you want it to appear below your name

%----------------------------------------------------------------------------------------

\begin{document}

\maketitle

%----------------------------------------------------------------------------------------
%	TABLE OF CONTENTS
%----------------------------------------------------------------------------------------

%\setcounter{tocdepth}{1} % Uncomment this line if you don't want subsections listed in the ToC

\newpage
\tableofcontents
\newpage



%----------------------------------------------------------------------------------------
\section{Introduction}

The tool suite builds useful diagrams to help developers Quickly
Understand Code and Document In Seconds. Unlike typical UML/modeling
tools, Architexa is designed to be code-centric - around your
needs. Developers can easily:


\begin{itemize}

\item Make sense of your code with quick visualizations of key components.
\item Take advantage of powerful interactive exploration of the diagrams.
\item Document code architecture and share discussions with team members.

\end{itemize}

\section{Installation}

\begin{itemize}

\item In order for the Arhitexa plugin to work with Eclipse Luna you
  first need to install the Eclipse compatibility plugin. Go to Help
  $>$ Install New Software... and paste the following link:

\url{http://download.eclipse.org/eclipse/updates/4.4/}

\item then in the ``Eclipse Tests, Examples, and Extras'' category check ``Eclipse 2.0 Style Plugin Support''
\item install it
\item after restarting Eclipse, install Arhitexa plugin. Go again to Help
  $>$ Install New Software... and paste the following link:

\url{http://update.architexa.com/4.2/client}

\item select all and install
\item after restarting Eclipse you should be able to use Arhitexa
\item if during any of the installations you encounter an error ``No
  repository found containing...'', you can try to uncheck ``Contact
  all update sites during install to find required software'' option
  when installing the plugins.

\end{itemize}

\section{Usage}
We will use Arhitexa extensively in the lab sessions to discuss your
projects and we expect from you to come with an Arhitexa diagram of
(a subset of) your implementation that supports your questions.

To get started, watch a short overview video to see the
capabilities Arhitexa offers:


\url{http://www.architexa.com/support/videos/intro/video}



\end{document}
