\documentclass{article}
\usepackage[T1]{fontenc}

\usepackage{fancyhdr} % Required for custom headers
\usepackage{lastpage} % Required to determine the last page for the footer
\usepackage{extramarks} % Required for headers and footers
\usepackage[usenames,dvipsnames]{color} % Required for custom colors
\usepackage{graphicx} % Required to insert images
\usepackage{listings} % Required for insertion of code
\usepackage{courier} % Required for the courier font
\usepackage{lipsum} % Used for inserting dummy 'Lorem ipsum' text into the template
\usepackage{hyperref}
% Margins
\topmargin=-0.45in
\evensidemargin=0in
\oddsidemargin=0in
\textwidth=6.5in
\textheight=9.0in
\headsep=0.25in

\linespread{1.1} % Line spacing

% Set up the header and footer
\pagestyle{fancy}
\lhead{\hmwkAuthorName} % Top left header
\chead{\hmwkClass\ (\hmwkClassInstructor): \hmwkTitle} % Top center head
\rhead{\firstxmark} % Top right header
\lfoot{\lastxmark} % Bottom left footer
\cfoot{} % Bottom center footer
\rfoot{Page\ \thepage\ of\ \protect\pageref{LastPage}} % Bottom right footer
\renewcommand\headrulewidth{0.4pt} % Size of the header rule
\renewcommand\footrulewidth{0.4pt} % Size of the footer rule

\setlength\parindent{0pt} % Removes all indentation from paragraphs

\usepackage{listings}
\usepackage{color}

\definecolor{dkgreen}{rgb}{0,0.6,0}
\definecolor{gray}{rgb}{0.5,0.5,0.5}
\definecolor{mauve}{rgb}{0.58,0,0.82}

\lstset{frame=tb,
  language=Java,
  aboveskip=3mm,
  belowskip=3mm,
  showstringspaces=false,
  columns=flexible,
  basicstyle={\small\ttfamily},
  numbers=none,
  numberstyle=\tiny\color{gray},
  keywordstyle=\color{blue},
  commentstyle=\color{dkgreen},
  stringstyle=\color{mauve},
  breaklines=true,
  breakatwhitespace=true
  tabsize=3
}

%----------------------------------------------------------------------------------------
%	DOCUMENT STRUCTURE COMMANDS
%	Skip this unless you know what you're doing
%----------------------------------------------------------------------------------------

% Header and footer for when a page split occurs within a problem environment
\newcommand{\enterProblemHeader}[1]{
\nobreak\extramarks{#1}{#1 continued on next page\ldots}\nobreak
\nobreak\extramarks{#1 (continued)}{#1 continued on next page\ldots}\nobreak
}

% Header and footer for when a page split occurs between problem environments
\newcommand{\exitProblemHeader}[1]{
\nobreak\extramarks{#1 (continued)}{#1 continued on next page\ldots}\nobreak
\nobreak\extramarks{#1}{}\nobreak
}




%----------------------------------------------------------------------------------------
%	NAME AND CLASS SECTION
%----------------------------------------------------------------------------------------


\newcommand{\hmwkTitle}{Debugging in Eclipse} % Assignment title
\newcommand{\hmwkDueDate}{Aprile 21, 2015} % Due date
\newcommand{\hmwkClass}{Ingegneria del Software 1} % Course/class
\newcommand{\hmwkClassTime}{} % Class/lecture time
\newcommand{\hmwkClassInstructor}{Sr\dj{}an Krsti\'c and Marco Scavuzzo} % Teacher/lecturer
\newcommand{\hmwkAuthorName}{} % Your name

%----------------------------------------------------------------------------------------
%	TITLE PAGE
%----------------------------------------------------------------------------------------

\title{
\vspace{2in}
\textmd{\textbf{\hmwkClass:\ \hmwkTitle}}\\
\normalsize\vspace{0.1in}\small{Da completare entro \hmwkDueDate}\\
\vspace{0.1in}\large{\textit{\hmwkClassInstructor}}
\vspace{3in}
}

\author{\textbf{\hmwkAuthorName}}
\date{} % Insert date here if you want it to appear below your name

%----------------------------------------------------------------------------------------

\begin{document}

\maketitle

%----------------------------------------------------------------------------------------
%	TABLE OF CONTENTS
%----------------------------------------------------------------------------------------

%\setcounter{tocdepth}{1} % Uncomment this line if you don't want subsections listed in the ToC

\newpage
\tableofcontents
\newpage



%----------------------------------------------------------------------------------------
\section{Preliminaries}
Il \emph{debugging} (o semplicemente debug), in informatica, indica l'attivit\' a   che consiste nell'individuazione della porzione di software affetta da errore (bug) rilevata nei software a seguito dell'utilizzo del programma.

Un \emph{breakpoint} nel codice sorgente specifica dove l'esecuzione di un programma si deve fermare. Un volta raggiunto il breakpoint \' e possibile analizzare le variabili, cambiare il loro valore etc.

Un \emph{watchpoint} \' e un break point su un campo. Il debugger si ferma quando il campo \' e letto o modificato

\subsection{Fraction Class}
\begin{lstlisting}
package org.ingsoft.debugging;

/**
 * A class representing a fraction of integer values. The class provides functionality for simplifying the fraction and performing the basic fraction computations.
 * @author Claudio
 */
public class Fraction {
	
	
	private int numerator;
	private int denominator;
	
	/**
	 * Constructs a fraction with the specified numerator and denominator
	 * @param numerator the numerator of the fraction
	 * @param denominator the denominator of the fraction
	 */
	public Fraction(int numerator, int denominator){
		this.numerator=numerator;
		this.denominator=denominator;
	}
	
	/**
	 * Constructs a fraction with the specified numerator and a denominator of 1
	 * @param numerator the numerator of the fraction
	 */
	public Fraction(int numerator){
		this(numerator,1);
	}
	
	/**
	 * @return the numerator
	 */
	public int getNumerator() {
		return numerator;
	}
	/**
	 * @param numerator the numerator to set
	 */
	public void setNumerator(int numerator) {
		this.numerator = numerator;
	}
	/**
	 * @return the denominator
	 */
	public int getDenominator() {
		return denominator;
	}
	/**
	 * @param denominator the denominator to set
	 */
	public void setDenominator(int denominator) {
		this.denominator = denominator;
	}

	/**
	 * Add this fraction to the specified fraction and returns it as a new fraction (not simplified). 
	 * It does not modify this fraction.
	 * @param f the fraction to be added to this fraction
	 * @return a new fraction equivalent to this fraction plus the parameter
	 */
	public Fraction add(Fraction f)
	{
		int num= (this.numerator * f.denominator) + (this.denominator * f.numerator);
		int den= this.denominator * f.denominator;
		Fraction sum=null;
		sum.setNumerator(num);
		sum.setDenominator(den);
		//Fraction sum=new Fraction(num, den);		
		
		return sum;
	}
	
	@Override
	public String toString(){
		return this.numerator+"/"+this.denominator;
	}
}
\end{lstlisting}

\subsection{Main}
\begin{lstlisting}
package org.ingsoft.debugging;

public class FractionMain {

	public static void main(String[] args) {
		Fraction f=new Fraction(3,4);
		Fraction g=new Fraction(5);
		Fraction[] myfractions=new Fraction[5];
		
		//add the fractions and store the result
		Fraction sum=f.add(g);
		
		myfractions[0]=f;
		myfractions[1]=g;
		myfractions[4]=sum;
		
		// Print the result
		System.out.println(myfractions[4].toString());
	}
}
\end{lstlisting}


\section{Interfaccia di Debug (Debug perspective)}
\begin{itemize}
\item \emph{call stack} mostra le parti del codice che sono in esecuzione e come sono legate le une alle altre
\item \emph{Breakpoints}: mostra l'insieme dei break point del vostro programma. Consente di rimovere, attivare e disattivare un break point. \`E possibile anche disattivare tutti i breakpoint cliccando su ``skip all breakpoints'' (pallino azzurro barrato)
\item \emph{Variables} mostra attributi e variabili (cliccando sulla freccetta in alto a destra \`e possibile selezionare  tra le altre
\begin{itemize}
\item campi da mostrare (Java)
\item il tipo di ogni variabile (Layout)
\end{itemize}
\end{itemize}


\section{Utilizzare il debugger}
Prima di tutto assicuriamoci che il debugger consideri solo il "nostro" codice ed eliminiamo il  debug di altro codice (e.g., librerie java).
Cliccare su preferences $>$ java debug $>$ step filtering $>$ use step filters $>$ check everything $>$ finish

\subsection{Aggiungere rimuovere e disabilitare un break point su un istruzione}
\begin{itemize}
\item Per \textit{aggiungere} un break point \`e sufficiente fare doppio click sul bordo della riga dove vogliamo aggiungerlo. Il break point indica la riga sulla quale si desidera che la computazione venga fermata.
\item  Per \textit{disabilitare} un break point cliccare con il tasto destro sul break point e premere ``disable break point''.
\item  Per \textit{eliminare} un break point cliccare due volte su di esso.
\end{itemize}

\subsection{Breakpoint properties}
Dopo aver aggiunto un breakpoint puoi selezionare le propriet\`a del break point (per esempio \`e possibile specificare che un breakpoint deve divenire attivo dopo  essere stato raggiunto 12 volte o quando una particolare condizione \`e verificata)
\begin{itemize}
\item \textit{right click} $>$ \textit{breakpoint properties}
\end{itemize}


\subsection{Aggiungere un break point su un metodo}
\`E possibile aggiungere un break point su un metodo facendo doppio click sul margine sinistro del metodo.
\`E possibile configurare se il debugger debba fermarsi quando all'uscita o all'entrata del metodo, agendo sulle propriet\`a.

\subsection{Aggiungere un watchpoint}
\begin{itemize}
\item posso settare un watch point cliccando due volte sul margine destro
\item  Posso configurare le propriet\`a per far in modo che l'esecuzione si fermi durante la lettura (Field access)  o la modifica (Field Modification) di un campo, o entrambi.
\end{itemize}

\subsection{Lanciare il debugger}
\begin{itemize}
\item Per lanciare il debugger \`e sufficiente cliccare sul bug (insetto) presente a fianco del pulsante run o, alternativamente, andare su \textit{run} $>$ \textit{debug}
\item Una volta lanciato il debug, eclipse automaticamente ci porta nella debug perspective.
\item la linea evidenziata in verde corrisponde alla linea di codice che sta per essere eseguita (nota che tale linea non \`e ancora stata eseguita).
\end{itemize}

\subsection{Comandi di debugging}

\begin{itemize}
\item \textit{step into} consente di valutare le linee di codice che eseguono una data operazione. Per esempio se la linea di codice selezionata chiama un particolare metodo, cliccando su step into si va a valutare il codice contenuto in quel metodo
\item \textit{step over} esegue le line di codice ma non mostra il comportamento interno di quelle linee. (e.g., se viene chiamato un metodo salta il debugging delle linee di codice di quel metodo).
\item \textit{step return} permette di ritornare da un metodo nel quale e' stata eseguita una step into.
\item \textit{use step filters} se cliccato abilita l'utilizzo dei step filters (si consiglia di tenerlo abilitato per evitare di debuggare codice java nativo)
\item \textit{resume} viene utilizzato, quando sono inseriti pi\' u break points, per andare al break point successivo
\end{itemize}

\subsection{Cambiare il valore delle variabili}
Per cambiare il valore a una variabile durante il debug \`e sufficiente cliccare all'interno del campo value (nota viene utilizzato il metodo toString() per mostrare il contenuto della variabile)


\subsection{Espressioni}
\`E possibile, durante il debug, valutare il valore di alcune espressioni (porzioni di codice). In particolare, due diverse azioni possono essere eseguite:

\begin{itemize}
\item selezionando un espressione per esempio this.numerator*f.denominator e cliccando su watch \`e possibile vedere  il valore dell'espressione;
\item si possono anche aggiungere nuove espressioni cliccando su add e aggiungendo una data espressione;
\item si possono anche aggiungere altre espressioni (this.numerator*10).
\end{itemize}

 Nota che e' anche possibile cambiare il valore delle variabili mentre il codice e' in esecuzione.


\end{document}
